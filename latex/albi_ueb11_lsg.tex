\documentclass{homework}
\usepackage{marvosym}
\usepackage{hyperref}
\usepackage{color}
\usepackage{caption}
\usepackage{subcaption}
\usepackage{float}

\course{Algorithmische Bioinformatik}
\semester{Wintersemester 2012 / 2013}
\no{11}
\date{Montag, dem 14. Januar 2013}
\author{Stefan Meißner (4279113) und Niels Hoppe (4356370)}
\tutorial{Dienstag 08:00 - 10:00}
\tutor{Alena van Bömmel (Übungsgruppe 3)}

\begin{document}
\maketitle
\begin{enumerate} 

\aufgabe{Statistische Verteilungen}{30}
\begin{enumerate}
\item 
Vulkane brechen zufällig (je nach Vulkan in einem unterschiedlichen Zeitraum) aus. 
Um die Ausbrüche auf den Zeitraum von einem Jahr zu beziehen, bietet sich die Poisson-Verteilung an.
\item 
Bei Hochleistungssport ist davon auszugehen, dass alle Athleten ungefähr die gleiche Leistung bringen.
Die Sieger werden nur einige wenige Zentimeter weiter und die Letzten nur einige wenige Zentimeter weniger springen als der Rest.
Daher kann hier von einer Normalverteilung ausgegangen werden.
\item
Wikipedia: \textit{Da die Exponentialverteilung auch als Lebensdauerverteilung verwendet wird, ist es möglich,
damit zusammenhängende Größen wie Überlebenswahrscheinlichkeit,
die Restlebensdauer und die Ausfallrate mit Hilfe der Verteilungsfunktion anzugeben.}
\end{enumerate}

\aufgabe{Gibbs Sampler}{30}

\aufgabe{Genstrukturen}{30}

\aufgabe{Sequencing by Hybridisation}{30}
\begin{enumerate}
\item
Das Spektrum der Sequenz ist die Menge aller in ihr enthaltenen Teilsequenzen der Länge 3:
$$S = \{\texttt{ATC}, \texttt{TCG}, \texttt{CGT}, \texttt{GTC}, \texttt{CGA}, \texttt{GAT}\}$$

Erstellt man aus diesem Spektrum einen Overlap-Graphen, so zeigt sich, dass in diesem kein Hammilton-Pfad existiert.
Auch wenn man den Euler-Trick anwendet, bleibt das Problem bestehen, da in dem resultierenden Graphen kein Euler-Pfad existiert.

\item
Wir erstellen zuerst einen Overlap-Graphen und verwenden anschließend den Euler-Trick,
um die ursprüngliche Sequenz zu rekonstruieren.

\begin{figure}[H]
\setlength{\unitlength}{1.5cm}
\centering

\begin{subfigure}{0.5\linewidth}
\centering
\begin{picture}(3,3)(0,0)
\put(1,0){\texttt{CGT}$^{3}$}
\put(2,0){\texttt{TCG}$^{5}$}
\put(3,1){\texttt{ACG}$^{2}$}
\put(3,2){\texttt{TAC}$^{1}$}
\put(2,3){\texttt{GTC}$^{4}$}

\put(1,0){\vector(1,1){1}}		% CGT, GTC
\put(2,0){\vector(1,1){1}}		% TCG, CGT
\put(3,1){\vector(1,1){1}}		% ACG, CGT
\put(3,2){\vector(1,1){1}}		% TAC, ACG
\put(2,3){\vector(1,1){1}}		% GTC, TCG
\end{picture}

\caption{Overlap-Graph}
\label{fig:31a}
\end{subfigure}%
\begin{subfigure}{0.5\linewidth}
\centering

\begin{picture}(3,3)(0,0)
\put(1,0){\texttt{CG}$^{3,6}$}
\put(2,0){\texttt{GT}$^{4}$}
\put(3,1){\texttt{TC}$^{5}$}
\put(3,2){\texttt{AC}$^{2}$}
\put(2,3){\texttt{TA}$^{1}$}

\put(1,0){\vector(1,1){1}}		% CG, GT
\put(2,0){\vector(1,1){1}}		% GT, TC
\put(3,1){\vector(1,1){1}}		% TC, CG
\put(3,2){\vector(1,1){1}}		% AC, CG
\put(2,3){\vector(1,1){1}}		% TA, AC
\end{picture}

\caption{Euler-Trick}
\label{fig:31b}
\end{subfigure}

\caption{Sequencing by Hybridisation}
\end{figure}

Aus beiden Graphen lassen sich \texttt{TACCGCACGT} und \texttt{TACGCACCGT} als zusammengesetzte Sequenzen ablesen.
\end{enumerate}

\aufgabe{Lowess Normalisierung}{40}
\begin{enumerate}
\item
\item
\item Der Kernel ermöglicht eine Gewichtung der Punkte, die zur Normalisierung herangezogen werden.
Es kann zum Beispiel eine Normalverteilung als Gewichtung gewählt werden, um solche Punkte, die nahe am zu normalisierenden Punkt liegen,
stärker zu gewichten, als solche, die eine große Distanz zu ihm haben.
\end{enumerate}

\aufgabe{RNA-Struktur}{30}
\begin{enumerate}
\item
\item
Die Struktur C kann so nicht vom Zucker-Algorithmus bestimmt werden,
da sie der Bedingung \ldots % TODO
widerspricht. Das zeigt sich deutlich im Kreisdiagramm:

\end{enumerate}

\aufgabe{Clustering}{40}
\begin{enumerate}
\item ((AB)((CD)(EF)))
\item (((AB)(CD))(EF))
\item $x \in B$
\end{enumerate}

\end{enumerate}
\end{document}
