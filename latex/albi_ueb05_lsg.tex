\documentclass{homework}
\usepackage{marvosym}
\usepackage{hyperref}

\course{Algorithmische Bioinformatik}
\semester{Wintersemester 2012 / 2013}
\no{5}
\date{Montag, dem 19. November 2012}
\author{Stefan Meißner (4279113) und Niels Hoppe (4356370)}
\tutorial{Dienstag 08:00 - 10:00}
\tutor{Alena van Bömmel (Übungsgruppe 3)}

\begin{document}
\maketitle
\begin{enumerate} 

\aufgabe{Matrix Diagonalisierung}{60}
\begin{enumerate}
\item
Eigenwerte aus $det(A - \lambda I)$ bestimmen:

\begin{eqnarray*}
det(A - \lambda I) & = & \left|\begin{array}{ccc}
2 - \lambda & 1 & 1\\
-1 & 2 - \lambda & 1\\
1 & 1 & 2 - \lambda
\end{array}\right|\\
& = & 1 \cdot \left|\begin{array}{cc}
1 & 1\\ 1 & 2 - \lambda
\end{array}\right| - (2 - \lambda) \cdot \left|\begin{array}{cc}
2 - \lambda & 1\\ 1 & 2 - \lambda
\end{array}\right| - 1 \cdot \left|\begin{array}{cc}
2 - \lambda & 1\\ 1 & 1
\end{array}\right|\\
& = & (2 - \lambda - 1) - (2 - \lambda) ((2 - \lambda)^2 - 1) - (2 - \lambda - 1)\\
& = & -(\lambda - 3)(\lambda-2)(\lambda-1)
\end{eqnarray*}

Als Lösung dafür ergeben sich die Eigenwerte $\lambda_1 = 3$, $\lambda_2 = 2$ und $\lambda_3 = 1$.

Eigenvektoren $\vec{v_i}$ aus $A \vec{v_i} = \lambda_i \vec{v_i}$ bestimmen:

\begin{itemize}
\item[$(\lambda_1 = 3)$]
$$
\left.\begin{array}{rcl}
2x + y + z & = & 3x\\
-x + 2y + z & = & 3y\\
x + y + 2z & = & 3z
\end{array}\right\}
\vec{v_1} = (1, 0, 1)
$$

\item[$(\lambda_2 = 2)$]
$$
\left.\begin{array}{rcl}
2x + y + z & = & 2x\\
-x + 2y + z & = & 2y\\
x + y + 2z & = & 2z
\end{array}\right\}
\vec{v_2} = (1, -1, 1)
$$

\item[$(\lambda_3 = 1)$]
$$
\left.\begin{array}{rcl}
2x + y + z & = & 1x\\
-x + 2y + z & = & 1y\\
x + y + 2z & = & 1z
\end{array}\right\}
\vec{v_3} = (0, -1, 1)
$$
\end{itemize}

\item $D$ und $T$ ergeben sich aus den Eigenwerten und Eigenvektoren als
$$D = \begin{pmatrix}3 & 0 & 0\\ 0 & 2 & 0\\ 0 & 0 & 1\end{pmatrix}
\quad \text{ und } \quad
T = \begin{pmatrix}1 & 1 & 0\\ 0 & -1 & -1\\ 1 & 1 & 1\end{pmatrix}.$$

Das Inverse $T^{-1}$ berechnen wir durch paralleles Umformen mit der Einheitsmatrix $I$,
denn es gilt $(T | I) \Leftrightarrow (I | T^{-1})$:

$$\left(\begin{array}{ccc|ccc}
1 & 1 & 0 & 1 & 0 & 0\\
0 & -1 & -1 & 0 & 1 & 0\\
1 & 1 & 1 & 0 & 0 & 1
\end{array}\right) \quad \text{III} = \text{III} - \text{I}$$

$$\left(\begin{array}{ccc|ccc}
1 & 1 & 0 & 1 & 0 & 0\\
0 & -1 & -1 & 0 & 1 & 0\\
0 & 0 & 1 & -1 & 0 & 1
\end{array}\right) \quad \text{II} = (-1)(\text{II} + \text{I})$$

$$\left(\begin{array}{ccc|ccc}
1 & 1 & 0 & 1 & 0 & 0\\
0 & 1 & 0 & 1 & -1 & 1\\
0 & 0 & 1 & -1 & 0 & 1
\end{array}\right) \quad \text{I} = \text{I} - \text{II}$$

$$\left(\begin{array}{ccc|ccc}
1 & 0 & 0 & 0 & 1 & 1\\
0 & 1 & 0 & 1 & -1 & 1\\
0 & 0 & 1 & -1 & 0 & 1
\end{array}\right)$$

Daraus ergibt sich $T^{-1}$ als
$$T^{-1} = \begin{pmatrix}0 & 1 & 1\\1 & -1 & -1\\ -1 & 0 & 1\end{pmatrix}.$$

\item
\end{enumerate}

\aufgabe{Profile HMMs}{80}
\begin{enumerate}
\item
\item
\item
\item
\item
\end{enumerate}

\aufgabe{Baum-Welch Algorithmus}{100}

1. Iteration, 1. Sequenz

$$a_{2,1} = a{2,2} = 0,4 \quad e_2(x) = e_2(y) = 0,5$$
$$F = \begin{pmatrix}
1 & 0 & 0 & 0\\
0 & 0,1 & 0,25 & 0\\
0 & 0,12 & 0,065 & 0\\
0 & 0,0688 & 0,031 & 0\\
0 & 0 & 0 & 0,0108
\end{pmatrix}
\quad
B = \begin{pmatrix}
0,0108 & 0 & 0 & 0\\
0 & 0,03224 & 0,03027 & 0\\
0 & 0,0596 & 0,056 & 0\\
0 & 0,11 & 0,104 & 0\\
0 & 0,2 & 0,2 & 0\\
0 & 0 & 0 & 1
\end{pmatrix}$$

1. Iteration, 2. Sequenz

$$F = \begin{pmatrix}
1 & 0 & 0 & 0\\
0 & 0,4 & 0,25 & 0\\
0 & 0,06 & 0,11 & 0\\
0 & 0,0148 & 0,031 & 0\\
0 & 0,01584 & 0,00842 & 0\\
0 & 0 & 0 & 0,00485
\end{pmatrix}
\quad
B = \begin{pmatrix}
0,004852 & 0 & 0 & 0\\
0 & 0,0071 & 0,008048 & 0\\
0 & 0,0266 & 0,0296 & 0\\
0 & 0,11 & 0,104 & 0\\
0 & 0,2 & 0,2 & 0\\
0 & 0 & 0 & 1
\end{pmatrix}$$

\end{enumerate}
\end{document}


















