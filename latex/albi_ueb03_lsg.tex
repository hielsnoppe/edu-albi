\documentclass{homework}
\usepackage{marvosym}
%\usepackage{xyling}

\course{Algorithmische Bioinformatik}
\semester{Wintersemester 2012 / 2013}
\no{3}
\date{Montag, dem 5. November 2012}
\author{Stefan Meißner (4279113) und Niels Hoppe (4356370)}
\tutorial{Dienstag 08:00 - 10:00}
\tutor{Alena van Bömmel (Übungsgruppe 3)}

\begin{document}
\maketitle
\begin{enumerate} 

\aufgabe{Distanzmatrizen}{15 + 15}
\begin{enumerate}
\item Das Kriterium einer aditiven Metrik ist erfüllt, wenn es eine Belegung der
Variablen $x, y, u$ und $v$ mit $a, b, c$ und $d$ gibt, sodass gilt:
$$d(x,y) + d(u,v) \leq d(x,u) + d(y,v) = d(x,v) + d(y,u)$$

Das Kriterium einer Ultrametrik ist erfüllt, wenn es eine Belegung der Variablen
$x, y, z$ aus jedem 3-Tupel aus $a, b, c$ und $d$ gibt, sodass gilt:
$$d(x,y) \leq d(x,z) = d(y,z)$$

Außerdem ist jede Ultrametrik additiv, sodass eine nicht additive Metrik auch
keine Ultrametrik sein kann.

\begin{enumerate}
\item Das Kriterium einer additiven Metrik ist erfüllt, da gilt:
\begin{eqnarray*}
d(a,b) + d(c,d) \leq & d(a,c) + d(b,d) & = d(a,d) + d(b,c)\\
9 + 2 \leq & 9 + 5 & = 9 + 5
\end{eqnarray*}
Das Kriterium einer Ultrametrik ist erfüllt, da gilt:
\begin{description}
\item[$(a,b,c)$] \begin{eqnarray*}
d(b,c) \leq & d(b,a) & = d(c,a)\\
5 \leq & 9 & = 9
\end{eqnarray*}
\item[$(a,b,d)$] \begin{eqnarray*}
d(b,d) \leq & d(b,a) & = d(d,a)\\
5 \leq & 9 & = 9
\end{eqnarray*}
\item[$(a,c,d)$] \begin{eqnarray*}
d(c,d) \leq & d(c,a) & = d(d,a)\\
2 \leq & 9 & = 9
\end{eqnarray*}
\item[$(b,c,d)$] \begin{eqnarray*}
d(c,d) \leq & d(c,b) & = d(d,b)\\
2 \leq & 5 & = 5
\end{eqnarray*}
\end{description}

\item Das Kriterium einer additiven Metrik ist erfüllt, da gilt:
\begin{eqnarray*}
d(a,b) + d(c,d) \leq & d(a,c) + d(b,d) & = d(a,d) + d(b,c)\\
5 + 2 \leq & 7 + 9 & = 8 + 8
\end{eqnarray*}
Das Kriterium einer Ultrametrik ist nicht erfüllt, da kein 3-Tupel zwei gleiche
Distanzen enthält und der zweite Teil der Ungleichung damit nicht erfüllbar ist.

\end{enumerate}
\item \textbf{Bonus:}
\end{enumerate}

\aufgabe{Phylogenetische Bäume}{30}
\begin{enumerate}
\item
\item
\item
\end{enumerate}

\aufgabe{Implementierung Clustering-Verfahren}{60}
\begin{enumerate}
\item
\item
\item
\end{enumerate}

\aufgabe{Maximum-Likelihood-Schätzung}{60}
\begin{enumerate}
\item
\item
\item
\item
\item
\end{enumerate}

\end{enumerate}
\end{document}
