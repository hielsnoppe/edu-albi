\documentclass{homework}
\usepackage{marvosym}
\usepackage{hyperref}
\usepackage{color}

\course{Algorithmische Bioinformatik}
\semester{Wintersemester 2012 / 2013}
\no{7}
\date{Montag, dem 3. Dezember 2012}
\author{Stefan Meißner (4279113) und Niels Hoppe (4356370)}
\tutorial{Dienstag 08:00 - 10:00}
\tutor{Alena van Bömmel (Übungsgruppe 3)}

\begin{document}
\maketitle
\begin{enumerate} 

\aufgabe{Gene finding}{100}

Abgabe am 10. Dezember 2012

%\begin{enumerate}
%\item
%\item
%\item
%\item
%\item 
%\end{enumerate}

\aufgabe{Coding capacity}{60}
Für ein Codon $ABC$ mit einer absoluten Häufigkeit von $M_{ABC}$ auf $K$
gezählte Codons gilt
$$F_{ABC} = \ln f_{ABC} = \ln \frac{M_{ABC}}{K}.$$

Die Scores $i \in {1,2,3}: H_i$ für ein Fenster $w$ berechnen sich als
$$H_i = \sum_{ABC \in w} F_{ABC}.$$

Daraus berechnen sich die Wahrscheinlichkeiten $i \in {1,2,3}: P_i$ dafür, dass
der Leserahmen $i$ in dem Fenster ein kodierender Abschnitt der Sequenz ist als
$$P_i = \frac{\exp H_i}{\exp H_1 + \exp H_2 + \exp H_3}.$$

$$\log \frac{P_i}{1-P_i}$$

Diese Formeln setzen wir um in folgendem Programm. Dabei ist $M$ ein
\textit{dictionary} mit Codons $ABC$ als Schlüssel und Werten $M_{ABC}$.

\begin{lstlisting}[language=python]
def MF(M, K):
	F = {}
	for codon, m in M.iteritems():
		if m == 0.0: m = 1.0
		F[codon] = math.log(m / float(K))
	return F

def winscore(seq, F):
	return sum([F[seq[n:n+3]] for n in range(0, len(seq), 3)])

def analyze(seq, winsize):
	data = []
	winsize = 3 * winsize	# window size in nucletides
	H, P = [0.0, 0.0, 0.0], [0.0, 0.0, 0.0]
	F = MF(M, 1000)			# prepared frequencies

	step = 0
	win = seq[:winsize]		# current window
	
	for n in seq[winsize:]:
		rfn = step % 3		# reading frame number is 0, 1 or 2
		H[rfn] = winscore(win, F)

		if rfn == 2:
			A = sum(H) / float(len(H))
			H = map(lambda Hi: math.exp(Hi - A), H)
			P = map(lambda Hi: Hi / sum(H), H)
			P = map(lambda Pi: math.log10(Pi / (1.0 - Pi)), P)
			data.append((step, P[0], P[1], P[2]))

		step = step + 1;
		win = win[1:] + n
	return data
\end{lstlisting}

\begin{center}
\setlength{\unitlength}{0.3mm}
\begin{picture}(209,80)(0,40)
\put(0,0){\vector(1,0){627}}
\put(0,0){\vector(0,1){40}}
\put(0,0){\vector(0,-1){40}}
\put(2,9.29984891158){\circle*{0.1}}
\put(2,-11.4304446727){\circle*{0.1}}
\put(2,-9.37799820378){\circle*{0.1}}
\put(5,9.29984891158){\circle*{0.1}}
\put(5,-11.4304446727){\circle*{0.1}}
\put(5,-9.37799820378){\circle*{0.1}}
\put(8,9.29984891158){\circle*{0.1}}
\put(8,-11.4304446727){\circle*{0.1}}
\put(8,-9.37799820378){\circle*{0.1}}
\put(11,9.29984891158){\circle*{0.1}}
\put(11,-11.4304446727){\circle*{0.1}}
\put(11,-9.37799820378){\circle*{0.1}}
\put(14,9.29984891158){\circle*{0.1}}
\put(14,-11.4304446727){\circle*{0.1}}
\put(14,-9.37799820378){\circle*{0.1}}
\put(17,9.29984891158){\circle*{0.1}}
\put(17,-11.4304446727){\circle*{0.1}}
\put(17,-9.37799820378){\circle*{0.1}}
\put(20,9.29984891158){\circle*{0.1}}
\put(20,-11.4304446727){\circle*{0.1}}
\put(20,-9.37799820378){\circle*{0.1}}
\put(23,9.29984891158){\circle*{0.1}}
\put(23,-11.4304446727){\circle*{0.1}}
\put(23,-9.37799820378){\circle*{0.1}}
\put(26,9.29984891158){\circle*{0.1}}
\put(26,-11.4304446727){\circle*{0.1}}
\put(26,-9.37799820378){\circle*{0.1}}
\put(29,9.29984891158){\circle*{0.1}}
\put(29,-11.4304446727){\circle*{0.1}}
\put(29,-9.37799820378){\circle*{0.1}}
\put(32,9.29984891158){\circle*{0.1}}
\put(32,-11.4304446727){\circle*{0.1}}
\put(32,-9.37799820378){\circle*{0.1}}
\put(35,9.29984891158){\circle*{0.1}}
\put(35,-11.4304446727){\circle*{0.1}}
\put(35,-9.37799820378){\circle*{0.1}}
\put(38,9.29984891158){\circle*{0.1}}
\put(38,-11.4304446727){\circle*{0.1}}
\put(38,-9.37799820378){\circle*{0.1}}
\put(41,9.29984891158){\circle*{0.1}}
\put(41,-11.4304446727){\circle*{0.1}}
\put(41,-9.37799820378){\circle*{0.1}}
\put(44,9.29984891158){\circle*{0.1}}
\put(44,-11.4304446727){\circle*{0.1}}
\put(44,-9.37799820378){\circle*{0.1}}
\put(47,9.29984891158){\circle*{0.1}}
\put(47,-11.4304446727){\circle*{0.1}}
\put(47,-9.37799820378){\circle*{0.1}}
\put(50,9.29984891158){\circle*{0.1}}
\put(50,-11.4304446727){\circle*{0.1}}
\put(50,-9.37799820378){\circle*{0.1}}
\put(53,9.29984891158){\circle*{0.1}}
\put(53,-11.4304446727){\circle*{0.1}}
\put(53,-9.37799820378){\circle*{0.1}}
\put(56,9.29984891158){\circle*{0.1}}
\put(56,-11.4304446727){\circle*{0.1}}
\put(56,-9.37799820378){\circle*{0.1}}
\put(59,9.29984891158){\circle*{0.1}}
\put(59,-11.4304446727){\circle*{0.1}}
\put(59,-9.37799820378){\circle*{0.1}}
\put(62,9.29984891158){\circle*{0.1}}
\put(62,-11.4304446727){\circle*{0.1}}
\put(62,-9.37799820378){\circle*{0.1}}
\put(65,9.29984891158){\circle*{0.1}}
\put(65,-11.4304446727){\circle*{0.1}}
\put(65,-9.37799820378){\circle*{0.1}}
\put(68,9.29984891158){\circle*{0.1}}
\put(68,-11.4304446727){\circle*{0.1}}
\put(68,-9.37799820378){\circle*{0.1}}
\put(71,9.29984891158){\circle*{0.1}}
\put(71,-11.4304446727){\circle*{0.1}}
\put(71,-9.37799820378){\circle*{0.1}}
\put(74,9.29984891158){\circle*{0.1}}
\put(74,-11.4304446727){\circle*{0.1}}
\put(74,-9.37799820378){\circle*{0.1}}
\put(77,9.29984891158){\circle*{0.1}}
\put(77,-11.4304446727){\circle*{0.1}}
\put(77,-9.37799820378){\circle*{0.1}}
\put(80,9.29984891158){\circle*{0.1}}
\put(80,-11.4304446727){\circle*{0.1}}
\put(80,-9.37799820378){\circle*{0.1}}
\put(83,9.29984891158){\circle*{0.1}}
\put(83,-11.4304446727){\circle*{0.1}}
\put(83,-9.37799820378){\circle*{0.1}}
\put(86,9.29984891158){\circle*{0.1}}
\put(86,-11.4304446727){\circle*{0.1}}
\put(86,-9.37799820378){\circle*{0.1}}
\put(89,9.29984891158){\circle*{0.1}}
\put(89,-11.4304446727){\circle*{0.1}}
\put(89,-9.37799820378){\circle*{0.1}}
\put(92,9.51607530918){\circle*{0.1}}
\put(92,-10.7086260457){\circle*{0.1}}
\put(92,-9.76985138){\circle*{0.1}}
\put(95,7.31320593072){\circle*{0.1}}
\put(95,-7.31765234931){\circle*{0.1}}
\put(95,-11.8974171543){\circle*{0.1}}
\put(98,6.77943957243){\circle*{0.1}}
\put(98,-6.78431634583){\circle*{0.1}}
\put(98,-11.2839443214){\circle*{0.1}}
\put(101,4.89162408645){\circle*{0.1}}
\put(101,-4.89184954308){\circle*{0.1}}
\put(101,-12.0694501484){\circle*{0.1}}
\put(104,5.97222640417){\circle*{0.1}}
\put(104,-5.97225944142){\circle*{0.1}}
\put(104,-14.8136564004){\circle*{0.1}}
\put(107,6.26238827424){\circle*{0.1}}
\put(107,-6.26241794751){\circle*{0.1}}
\put(107,-15.1965847293){\circle*{0.1}}
\put(110,4.80553498157){\circle*{0.1}}
\put(110,-4.80556808869){\circle*{0.1}}
\put(110,-13.6501955696){\circle*{0.1}}
\put(113,4.00653979592){\circle*{0.1}}
\put(113,-4.0066139617){\circle*{0.1}}
\put(113,-12.1609533816){\circle*{0.1}}
\put(116,1.50460440164){\circle*{0.1}}
\put(116,-1.5046135419){\circle*{0.1}}
\put(116,-11.7432586576){\circle*{0.1}}
\put(119,-0.187670134589){\circle*{0.1}}
\put(119,0.187669647168){\circle*{0.1}}
\put(119,-13.7160570803){\circle*{0.1}}
\put(122,1.99156752224){\circle*{0.1}}
\put(122,-1.99156754927){\circle*{0.1}}
\put(122,-17.1725259573){\circle*{0.1}}
\put(125,-1.65607849456){\circle*{0.1}}
\put(125,1.65607842947){\circle*{0.1}}
\put(125,-16.1472825753){\circle*{0.1}}
\put(128,-0.614591111089){\circle*{0.1}}
\put(128,0.61459047256){\circle*{0.1}}
\put(128,-13.5779091372){\circle*{0.1}}
\put(131,-3.76062448416){\circle*{0.1}}
\put(131,3.76061493387){\circle*{0.1}}
\put(131,-13.7009503446){\circle*{0.1}}
\put(134,-1.05965360759){\circle*{0.1}}
\put(134,1.05964766803){\circle*{0.1}}
\put(134,-11.839169272){\circle*{0.1}}
\put(137,0.0647382377626){\circle*{0.1}}
\put(137,-0.0647504857965){\circle*{0.1}}
\put(137,-10.906819001){\circle*{0.1}}
\put(140,0.485328437376){\circle*{0.1}}
\put(140,-0.485626660321){\circle*{0.1}}
\put(140,-8.19956901354){\circle*{0.1}}
\put(143,1.57953259866){\circle*{0.1}}
\put(143,-1.58176894715){\circle*{0.1}}
\put(143,-7.01984546424){\circle*{0.1}}
\put(146,2.38411817324){\circle*{0.1}}
\put(146,-2.39990812218){\circle*{0.1}}
\put(146,-5.97922489752){\circle*{0.1}}
\put(149,2.39701626605){\circle*{0.1}}
\put(149,-2.41850286896){\circle*{0.1}}
\put(149,-5.7252191795){\circle*{0.1}}
\put(152,1.37880042098){\circle*{0.1}}
\put(152,-1.42242381119){\circle*{0.1}}
\put(152,-4.30855432297){\circle*{0.1}}
\put(155,0.715655896895){\circle*{0.1}}
\put(155,-0.943065979902){\circle*{0.1}}
\put(155,-2.51266845982){\circle*{0.1}}
\put(158,-1.04097361583){\circle*{0.1}}
\put(158,0.91131760875){\circle*{0.1}}
\put(158,-3.09284648072){\circle*{0.1}}
\put(161,-3.72848510422){\circle*{0.1}}
\put(161,3.68359869811){\circle*{0.1}}
\put(161,-6.30293835242){\circle*{0.1}}
\put(164,-5.10761398522){\circle*{0.1}}
\put(164,4.96469193433){\circle*{0.1}}
\put(164,-6.60745117624){\circle*{0.1}}
\put(167,-6.98622741546){\circle*{0.1}}
\put(167,2.59775044295){\circle*{0.1}}
\put(167,-2.60389384711){\circle*{0.1}}
\put(170,-2.25193822607){\circle*{0.1}}
\put(170,-0.878988602776){\circle*{0.1}}
\put(170,0.590849787587){\circle*{0.1}}
\put(173,-2.17422282368){\circle*{0.1}}
\put(173,-0.394616585711){\circle*{0.1}}
\put(173,0.125395466908){\circle*{0.1}}
\put(176,-1.48986785871){\circle*{0.1}}
\put(176,0.567592924874){\circle*{0.1}}
\put(176,-1.26099833002){\circle*{0.1}}
\put(179,-1.69076657637){\circle*{0.1}}
\put(179,1.37132641347){\circle*{0.1}}
\put(179,-2.63283677292){\circle*{0.1}}
\put(182,-1.65673364282){\circle*{0.1}}
\put(182,1.22961973332){\circle*{0.1}}
\put(182,-2.30063124003){\circle*{0.1}}
\put(185,-1.02243766186){\circle*{0.1}}
\put(185,-2.14437535049){\circle*{0.1}}
\put(185,0.680225416373){\circle*{0.1}}
\put(188,-0.365248880864){\circle*{0.1}}
\put(188,-1.04644617337){\circle*{0.1}}
\put(188,-0.451205199232){\circle*{0.1}}
\put(191,0.154999659924){\circle*{0.1}}
\put(191,-3.44476239222){\circle*{0.1}}
\put(191,-0.220380326799){\circle*{0.1}}
\put(194,-3.16351947774){\circle*{0.1}}
\put(194,-5.73842424254){\circle*{0.1}}
\put(194,3.11756797857){\circle*{0.1}}
\put(197,-6.49286698384){\circle*{0.1}}
\put(197,-6.70992927785){\circle*{0.1}}
\put(197,5.99214380205){\circle*{0.1}}
\put(200,-7.75903259217){\circle*{0.1}}
\put(200,-9.24128251107){\circle*{0.1}}
\put(200,7.61413040795){\circle*{0.1}}
\put(203,-8.04294479469){\circle*{0.1}}
\put(203,-11.092741215){\circle*{0.1}}
\put(203,8.01738292467){\circle*{0.1}}
\put(206,-8.01295786853){\circle*{0.1}}
\put(206,-11.4523226548){\circle*{0.1}}
\put(206,7.99654787068){\circle*{0.1}}
\put(209,-9.89253916761){\circle*{0.1}}
\put(209,-15.0307912255){\circle*{0.1}}
\put(209,9.89019972113){\circle*{0.1}}
\put(212,-12.8783933361){\circle*{0.1}}
\put(212,-17.5319266449){\circle*{0.1}}
\put(212,12.8743098912){\circle*{0.1}}
\put(215,-11.150467379){\circle*{0.1}}
\put(215,-19.2144347918){\circle*{0.1}}
\put(215,11.1503866903){\circle*{0.1}}
\put(218,-11.2648596035){\circle*{0.1}}
\put(218,-19.0316560146){\circle*{0.1}}
\put(218,11.2647460006){\circle*{0.1}}
\put(221,-10.5740717626){\circle*{0.1}}
\put(221,-22.8376930257){\circle*{0.1}}
\put(221,10.5740711214){\circle*{0.1}}
\put(224,-9.95195108831){\circle*{0.1}}
\put(224,-21.5769330273){\circle*{0.1}}
\put(224,9.9519497507){\circle*{0.1}}
\put(227,-10.2468038401){\circle*{0.1}}
\put(227,-23.3911374018){\circle*{0.1}}
\put(227,10.2468036075){\circle*{0.1}}
\put(230,-12.2021353083){\circle*{0.1}}
\put(230,-21.1867895871){\circle*{0.1}}
\put(230,12.2021073519){\circle*{0.1}}
\put(233,-11.0089411172){\circle*{0.1}}
\put(233,-19.4707619912){\circle*{0.1}}
\put(233,11.0088900791){\circle*{0.1}}
\put(236,-13.864126986){\circle*{0.1}}
\put(236,-19.686514267){\circle*{0.1}}
\put(236,13.8630619743){\circle*{0.1}}
\put(239,-13.6277415248){\circle*{0.1}}
\put(239,-19.7009204948){\circle*{0.1}}
\put(239,13.6269434835){\circle*{0.1}}
\put(242,-14.3130182692){\circle*{0.1}}
\put(242,-19.1240293481){\circle*{0.1}}
\put(242,14.3096106072){\circle*{0.1}}
\put(245,-15.032076013){\circle*{0.1}}
\put(245,-19.2476323376){\circle*{0.1}}
\put(245,15.0253253356){\circle*{0.1}}
\put(248,-14.8967893478){\circle*{0.1}}
\put(248,-19.0745527764){\circle*{0.1}}
\put(248,14.8897396747){\circle*{0.1}}
\put(251,-15.4593395327){\circle*{0.1}}
\put(251,-20.3163763966){\circle*{0.1}}
\put(251,15.4561074134){\circle*{0.1}}
\put(254,-16.07407988){\circle*{0.1}}
\put(254,-23.3847460092){\circle*{0.1}}
\put(254,16.0738878269){\circle*{0.1}}
\put(257,-13.4843674928){\circle*{0.1}}
\put(257,-22.0558895214){\circle*{0.1}}
\put(257,13.4843225106){\circle*{0.1}}
\put(260,-14.3381877532){\circle*{0.1}}
\put(260,-20.2519982223){\circle*{0.1}}
\put(260,14.3372290823){\circle*{0.1}}
\put(263,-13.7019773599){\circle*{0.1}}
\put(263,-19.1693216052){\circle*{0.1}}
\put(263,13.7003750673){\circle*{0.1}}
\put(266,-15.4419206563){\circle*{0.1}}
\put(266,-19.3276274579){\circle*{0.1}}
\put(266,15.4320693104){\circle*{0.1}}
\put(269,-16.6634290716){\circle*{0.1}}
\put(269,-20.5796380955){\circle*{0.1}}
\put(269,16.6539158114){\circle*{0.1}}
\put(272,-17.8718985324){\circle*{0.1}}
\put(272,-20.8636360008){\circle*{0.1}}
\put(272,17.8446021705){\circle*{0.1}}
\put(275,-18.126961627){\circle*{0.1}}
\put(275,-19.0417107803){\circle*{0.1}}
\put(275,17.8670412013){\circle*{0.1}}
\put(278,-19.1515204195){\circle*{0.1}}
\put(278,-21.1299986357){\circle*{0.1}}
\put(278,19.0667564029){\circle*{0.1}}
\put(281,-19.0305665927){\circle*{0.1}}
\put(281,-23.5727028942){\circle*{0.1}}
\put(281,19.0259259292){\circle*{0.1}}
\put(284,-19.1104931926){\circle*{0.1}}
\put(284,-22.708327189){\circle*{0.1}}
\put(284,19.0968011465){\circle*{0.1}}
\put(287,-19.1241362317){\circle*{0.1}}
\put(287,-21.6020529491){\circle*{0.1}}
\put(287,19.0754260883){\circle*{0.1}}
\put(290,-18.4714693777){\circle*{0.1}}
\put(290,-18.4331861482){\circle*{0.1}}
\put(290,17.8500569092){\circle*{0.1}}
\put(293,-18.3255007644){\circle*{0.1}}
\put(293,-15.9167617821){\circle*{0.1}}
\put(293,15.864133245){\circle*{0.1}}
\put(296,-19.177378564){\circle*{0.1}}
\put(296,-17.0566319145){\circle*{0.1}}
\put(296,16.984155642){\circle*{0.1}}
\put(299,-18.5348999388){\circle*{0.1}}
\put(299,-16.3374391206){\circle*{0.1}}
\put(299,16.2708607836){\circle*{0.1}}
\put(302,-17.7854437327){\circle*{0.1}}
\put(302,-14.3193136643){\circle*{0.1}}
\put(302,14.3034001941){\circle*{0.1}}
\put(305,-16.7755740674){\circle*{0.1}}
\put(305,-12.736210896){\circle*{0.1}}
\put(305,12.7279492604){\circle*{0.1}}
\put(308,-15.5533862863){\circle*{0.1}}
\put(308,-10.0813198573){\circle*{0.1}}
\put(308,10.0797262161){\circle*{0.1}}
\put(311,-14.1244274537){\circle*{0.1}}
\put(311,-9.62224876143){\circle*{0.1}}
\put(311,9.61739002737){\circle*{0.1}}
\put(314,-15.6486655401){\circle*{0.1}}
\put(314,-11.6359626355){\circle*{0.1}}
\put(314,11.6274447272){\circle*{0.1}}
\put(317,-16.1382564368){\circle*{0.1}}
\put(317,-10.5388331467){\circle*{0.1}}
\put(317,10.537456688){\circle*{0.1}}
\put(320,-15.6244030365){\circle*{0.1}}
\put(320,-12.7796836921){\circle*{0.1}}
\put(320,12.7474455119){\circle*{0.1}}
\put(323,-16.7075963983){\circle*{0.1}}
\put(323,-16.7746093585){\circle*{0.1}}
\put(323,16.1383967688){\circle*{0.1}}
\put(326,-14.508869847){\circle*{0.1}}
\put(326,-16.7591960096){\circle*{0.1}}
\put(326,14.4460842072){\circle*{0.1}}
\put(329,-17.6049297976){\circle*{0.1}}
\put(329,-16.5544411256){\circle*{0.1}}
\put(329,16.327643948){\circle*{0.1}}
\put(332,-14.7551665261){\circle*{0.1}}
\put(332,-16.2684371705){\circle*{0.1}}
\put(332,14.6149934558){\circle*{0.1}}
\put(335,-13.6707987211){\circle*{0.1}}
\put(335,-17.1398104744){\circle*{0.1}}
\put(335,13.6549374789){\circle*{0.1}}
\put(338,-14.4699258017){\circle*{0.1}}
\put(338,-19.1167580147){\circle*{0.1}}
\put(338,14.465810808){\circle*{0.1}}
\put(341,-15.3139985583){\circle*{0.1}}
\put(341,-19.1648927412){\circle*{0.1}}
\put(341,15.3037467153){\circle*{0.1}}
\put(344,-14.4203282412){\circle*{0.1}}
\put(344,-15.8675409445){\circle*{0.1}}
\put(344,14.2699861811){\circle*{0.1}}
\put(347,-14.784442721){\circle*{0.1}}
\put(347,-14.8127023174){\circle*{0.1}}
\put(347,14.1963975693){\circle*{0.1}}
\put(350,-14.0445553624){\circle*{0.1}}
\put(350,-13.3461682348){\circle*{0.1}}
\put(350,13.0249226925){\circle*{0.1}}
\put(353,-16.192886916){\circle*{0.1}}
\put(353,-14.5835741491){\circle*{0.1}}
\put(353,14.4570582402){\circle*{0.1}}
\put(356,-13.7319543386){\circle*{0.1}}
\put(356,-15.3086580284){\circle*{0.1}}
\put(356,13.600946008){\circle*{0.1}}
\put(359,-11.9522518738){\circle*{0.1}}
\put(359,-14.127465896){\circle*{0.1}}
\put(359,11.8840122504){\circle*{0.1}}
\put(362,-9.64780130124){\circle*{0.1}}
\put(362,-15.2769488393){\circle*{0.1}}
\put(362,9.6464710889){\circle*{0.1}}
\put(365,-9.36981507457){\circle*{0.1}}
\put(365,-16.5176163284){\circle*{0.1}}
\put(365,9.36958340249){\circle*{0.1}}
\put(368,-11.0109129548){\circle*{0.1}}
\put(368,-14.7506881141){\circle*{0.1}}
\put(368,10.9992712718){\circle*{0.1}}
\put(371,-11.0888000772){\circle*{0.1}}
\put(371,-11.6986956871){\circle*{0.1}}
\put(371,10.7392199426){\circle*{0.1}}
\put(374,-11.1562492851){\circle*{0.1}}
\put(374,-12.3046858109){\circle*{0.1}}
\put(374,10.9510023569){\circle*{0.1}}
\put(377,-8.17386708255){\circle*{0.1}}
\put(377,-14.804833812){\circle*{0.1}}
\put(377,8.17344703639){\circle*{0.1}}
\put(380,-6.92066121365){\circle*{0.1}}
\put(380,-14.9150300177){\circle*{0.1}}
\put(380,6.92057373362){\circle*{0.1}}
\put(383,-7.79555835732){\circle*{0.1}}
\put(383,-14.1386264556){\circle*{0.1}}
\put(383,7.79497323876){\circle*{0.1}}
\put(386,-5.85451742789){\circle*{0.1}}
\put(386,-11.4898299611){\circle*{0.1}}
\put(386,5.85319353337){\circle*{0.1}}
\put(389,-6.43341729771){\circle*{0.1}}
\put(389,-13.8626664779){\circle*{0.1}}
\put(389,6.4332495429){\circle*{0.1}}
\put(392,-7.73631578994){\circle*{0.1}}
\put(392,-15.8803823962){\circle*{0.1}}
\put(392,7.73624218953){\circle*{0.1}}
\put(395,-9.08824554992){\circle*{0.1}}
\put(395,-15.6777021493){\circle*{0.1}}
\put(395,9.08780499437){\circle*{0.1}}
\put(398,-9.60064452746){\circle*{0.1}}
\put(398,-18.2627730524){\circle*{0.1}}
\put(398,9.60060399961){\circle*{0.1}}
\put(401,-10.5040141085){\circle*{0.1}}
\put(401,-15.3342004894){\circle*{0.1}}
\put(401,10.5006806707){\circle*{0.1}}
\put(404,-8.8383207743){\circle*{0.1}}
\put(404,-14.4573812714){\circle*{0.1}}
\put(404,8.8369749739){\circle*{0.1}}
\put(407,-11.2248019111){\circle*{0.1}}
\put(407,-15.0016069417){\circle*{0.1}}
\put(407,11.2136430158){\circle*{0.1}}
\put(410,-11.5242310784){\circle*{0.1}}
\put(410,-14.4678356256){\circle*{0.1}}
\put(410,11.4954049007){\circle*{0.1}}
\put(413,-11.1462507921){\circle*{0.1}}
\put(413,-11.1637977438){\circle*{0.1}}
\put(413,10.55291767){\circle*{0.1}}
\put(416,-10.7583361121){\circle*{0.1}}
\put(416,-12.3184826535){\circle*{0.1}}
\put(416,10.6249896085){\circle*{0.1}}
\put(419,-11.5403181676){\circle*{0.1}}
\put(419,-12.7650760951){\circle*{0.1}}
\put(419,11.3505861091){\circle*{0.1}}
\put(422,-12.2271359056){\circle*{0.1}}
\put(422,-11.9591736449){\circle*{0.1}}
\put(422,11.4808013391){\circle*{0.1}}
\put(425,-12.5343429329){\circle*{0.1}}
\put(425,-12.1896725212){\circle*{0.1}}
\put(425,11.7429618171){\circle*{0.1}}
\put(428,-12.2423076924){\circle*{0.1}}
\put(428,-10.7969152642){\circle*{0.1}}
\put(428,10.6462825537){\circle*{0.1}}
\put(431,-11.9635313686){\circle*{0.1}}
\put(431,-11.2084585176){\circle*{0.1}}
\put(431,10.904345046){\circle*{0.1}}
\put(434,-12.1650856525){\circle*{0.1}}
\put(434,-14.5058101612){\circle*{0.1}}
\put(434,12.1083075435){\circle*{0.1}}
\put(437,-11.3435850203){\circle*{0.1}}
\put(437,-15.2970837797){\circle*{0.1}}
\put(437,11.3344694073){\circle*{0.1}}
\put(440,-13.3579245077){\circle*{0.1}}
\put(440,-15.0418539646){\circle*{0.1}}
\put(440,13.241154763){\circle*{0.1}}
\put(443,-12.5597661338){\circle*{0.1}}
\put(443,-13.5223834445){\circle*{0.1}}
\put(443,12.3119745766){\circle*{0.1}}
\put(446,-13.688697241){\circle*{0.1}}
\put(446,-14.0662468386){\circle*{0.1}}
\put(446,13.2550577503){\circle*{0.1}}
\put(449,-17.39136682){\circle*{0.1}}
\put(449,-14.3977107508){\circle*{0.1}}
\put(449,14.3704737501){\circle*{0.1}}
\put(452,-18.3422467738){\circle*{0.1}}
\put(452,-12.9642420093){\circle*{0.1}}
\put(452,12.9624663147){\circle*{0.1}}
\put(455,-18.8315698641){\circle*{0.1}}
\put(455,-13.2173665476){\circle*{0.1}}
\put(455,13.2160133086){\circle*{0.1}}
\put(458,-17.4773504338){\circle*{0.1}}
\put(458,-12.8426872868){\circle*{0.1}}
\put(458,12.838514373){\circle*{0.1}}
\put(461,-17.279064844){\circle*{0.1}}
\put(461,-13.3650438356){\circle*{0.1}}
\put(461,13.3555067093){\circle*{0.1}}
\put(464,-16.7371734785){\circle*{0.1}}
\put(464,-13.1023505581){\circle*{0.1}}
\put(464,13.0892250062){\circle*{0.1}}
\put(467,-16.2672728149){\circle*{0.1}}
\put(467,-9.40206101577){\circle*{0.1}}
\put(467,9.40174028139){\circle*{0.1}}
\put(470,-15.9078934992){\circle*{0.1}}
\put(470,-8.96063057024){\circle*{0.1}}
\put(470,8.9603387344){\circle*{0.1}}
\put(473,-14.1255926767){\circle*{0.1}}
\put(473,-10.9787560927){\circle*{0.1}}
\put(473,10.955865158){\circle*{0.1}}
\put(476,-16.5963836619){\circle*{0.1}}
\put(476,-11.7286345953){\circle*{0.1}}
\put(476,11.7254420095){\circle*{0.1}}
\put(479,-16.9625025072){\circle*{0.1}}
\put(479,-10.1408284356){\circle*{0.1}}
\put(479,10.1404912257){\circle*{0.1}}
\put(482,-15.7977911888){\circle*{0.1}}
\put(482,-8.23008441457){\circle*{0.1}}
\put(482,8.22994152786){\circle*{0.1}}
\put(485,-14.7619522053){\circle*{0.1}}
\put(485,-8.20983993272){\circle*{0.1}}
\put(485,8.20937998191){\circle*{0.1}}
\put(488,-14.9981570091){\circle*{0.1}}
\put(488,-8.02703212681){\circle*{0.1}}
\put(488,8.02674816164){\circle*{0.1}}
\put(491,-16.8432862631){\circle*{0.1}}
\put(491,-8.61240877246){\circle*{0.1}}
\put(491,8.61234218356){\circle*{0.1}}
\put(494,-18.5955502339){\circle*{0.1}}
\put(494,-9.3841741129){\circle*{0.1}}
\put(494,9.38415257821){\circle*{0.1}}
\put(497,-15.0969302357){\circle*{0.1}}
\put(497,-8.84811179724){\circle*{0.1}}
\put(497,8.84745975642){\circle*{0.1}}
\put(500,-16.6676477724){\circle*{0.1}}
\put(500,-7.96350096803){\circle*{0.1}}
\put(500,7.96346234718){\circle*{0.1}}
\put(503,-17.031271508){\circle*{0.1}}
\put(503,-9.0103435154){\circle*{0.1}}
\put(503,9.01025872313){\circle*{0.1}}
\put(506,-18.0662837643){\circle*{0.1}}
\put(506,-8.62435020547){\circle*{0.1}}
\put(506,8.6243336902){\circle*{0.1}}
\put(509,-15.6797804309){\circle*{0.1}}
\put(509,-10.2653248657){\circle*{0.1}}
\put(509,10.2636220519){\circle*{0.1}}
\put(512,-18.4000784027){\circle*{0.1}}
\put(512,-11.5395689722){\circle*{0.1}}
\put(512,11.5392465087){\circle*{0.1}}
\put(515,-19.2073297192){\circle*{0.1}}
\put(515,-11.8344217236){\circle*{0.1}}
\put(515,11.8342429446){\circle*{0.1}}
\put(518,-21.1973101704){\circle*{0.1}}
\put(518,-10.7594475499){\circle*{0.1}}
\put(518,10.7594423033){\circle*{0.1}}
\put(521,-21.3620148981){\circle*{0.1}}
\put(521,-10.4639912491){\circle*{0.1}}
\put(521,10.4639881602){\circle*{0.1}}
\put(524,-21.3353636762){\circle*{0.1}}
\put(524,-9.44726301762){\circle*{0.1}}
\put(524,9.44726202957){\circle*{0.1}}
\put(527,-21.7827301924){\circle*{0.1}}
\put(527,-8.38725668056){\circle*{0.1}}
\put(527,8.38725650632){\circle*{0.1}}
\put(530,-18.5698111607){\circle*{0.1}}
\put(530,-7.65212199325){\circle*{0.1}}
\put(530,7.65211897261){\circle*{0.1}}
\put(533,-17.3496683408){\circle*{0.1}}
\put(533,-9.42428825115){\circle*{0.1}}
\put(533,9.42419360184){\circle*{0.1}}
\put(536,-18.2126738467){\circle*{0.1}}
\put(536,-9.2791869157){\circle*{0.1}}
\put(536,9.27915726169){\circle*{0.1}}
\put(539,-16.3975848915){\circle*{0.1}}
\put(539,-9.76997092197){\circle*{0.1}}
\put(539,9.76954930971){\circle*{0.1}}
\put(542,-16.6402049298){\circle*{0.1}}
\put(542,-9.09932214937){\circle*{0.1}}
\put(542,9.09917479583){\circle*{0.1}}
\put(545,-16.0945743583){\circle*{0.1}}
\put(545,-6.98091302755){\circle*{0.1}}
\put(545,6.98088891416){\circle*{0.1}}
\put(548,-17.0233231699){\circle*{0.1}}
\put(548,-7.75014487436){\circle*{0.1}}
\put(548,7.75012481413){\circle*{0.1}}
\put(551,-18.0814698654){\circle*{0.1}}
\put(551,-10.2818760297){\circle*{0.1}}
\put(551,10.281766635){\circle*{0.1}}
\put(554,-18.4752276652){\circle*{0.1}}
\put(554,-11.9395639466){\circle*{0.1}}
\put(554,11.9390952777){\circle*{0.1}}
\put(557,-20.9939846644){\circle*{0.1}}
\put(557,-12.5142337277){\circle*{0.1}}
\put(557,12.5141837328){\circle*{0.1}}
\put(560,-21.865773976){\circle*{0.1}}
\put(560,-14.7664115364){\circle*{0.1}}
\put(560,14.7661665904){\circle*{0.1}}
\put(563,-23.2593935214){\circle*{0.1}}
\put(563,-14.7550572706){\circle*{0.1}}
\put(563,14.7550086702){\circle*{0.1}}
\put(566,-22.4973760877){\circle*{0.1}}
\put(566,-17.6407121561){\circle*{0.1}}
\put(566,17.6374785927){\circle*{0.1}}
\put(569,-21.8270075431){\circle*{0.1}}
\put(569,-18.6396806359){\circle*{0.1}}
\put(569,18.6178195033){\circle*{0.1}}
\put(572,-24.7753307624){\circle*{0.1}}
\put(572,-18.0702910468){\circle*{0.1}}
\put(572,18.0699054784){\circle*{0.1}}
\put(575,-27.171980304){\circle*{0.1}}
\put(575,-19.2885796626){\circle*{0.1}}
\put(575,19.2884804134){\circle*{0.1}}
\put(578,-26.3607643655){\circle*{0.1}}
\put(578,-17.4166301893){\circle*{0.1}}
\put(578,17.4166009172){\circle*{0.1}}
\put(581,-23.1130864361){\circle*{0.1}}
\put(581,-19.7598171884){\circle*{0.1}}
\put(581,19.7417186259){\circle*{0.1}}
\put(584,-24.3728802994){\circle*{0.1}}
\put(584,-19.3690834617){\circle*{0.1}}
\put(584,19.366353262){\circle*{0.1}}
\put(587,-25.3539778765){\circle*{0.1}}
\put(587,-21.438675161){\circle*{0.1}}
\put(587,21.4291549639){\circle*{0.1}}
\put(590,-22.7146010806){\circle*{0.1}}
\put(590,-21.4294075551){\circle*{0.1}}
\put(590,21.251210046){\circle*{0.1}}
\put(593,-22.9289508337){\circle*{0.1}}
\put(593,-21.7071446636){\circle*{0.1}}
\put(593,21.5168271703){\circle*{0.1}}
\put(596,-22.5978677023){\circle*{0.1}}
\put(596,-20.2950757107){\circle*{0.1}}
\put(596,20.2358478489){\circle*{0.1}}
\put(599,-22.3935246966){\circle*{0.1}}
\put(599,-19.6446587639){\circle*{0.1}}
\put(599,19.6087366352){\circle*{0.1}}
\put(602,-20.166206729){\circle*{0.1}}
\put(602,-20.952421732){\circle*{0.1}}
\put(602,19.8711780597){\circle*{0.1}}
\put(605,-20.5010117375){\circle*{0.1}}
\put(605,-20.1868247171){\circle*{0.1}}
\put(605,19.7277290222){\circle*{0.1}}
\put(608,-17.9808689615){\circle*{0.1}}
\put(608,-20.8239603619){\circle*{0.1}}
\put(608,17.9485714498){\circle*{0.1}}
\put(611,-18.1587806602){\circle*{0.1}}
\put(611,-21.7146192024){\circle*{0.1}}
\put(611,18.1444159041){\circle*{0.1}}
\put(614,-18.5488164886){\circle*{0.1}}
\put(614,-19.0056411976){\circle*{0.1}}
\put(614,18.1454759399){\circle*{0.1}}
\put(617,-16.9417835088){\circle*{0.1}}
\put(617,-18.2658787171){\circle*{0.1}}
\put(617,16.7706693127){\circle*{0.1}}
\put(620,-18.182446083){\circle*{0.1}}
\put(620,-19.5065412896){\circle*{0.1}}
\put(620,18.0113318565){\circle*{0.1}}
\put(623,-18.0588502003){\circle*{0.1}}
\put(623,-18.2826598271){\circle*{0.1}}
\put(623,17.5615062008){\circle*{0.1}}
\put(626,-17.3102623981){\circle*{0.1}}
\put(626,-15.645723784){\circle*{0.1}}
\put(626,15.526490953){\circle*{0.1}}
\end{picture}

\end{center}

\begin{enumerate}
\item
\item
\end{enumerate}

\aufgabe{RNA Sekundärstruktur}{40}

Geht schneller von Hand, zeichne ich dann ein.

\begin{enumerate}
\item
\item
\item
\end{enumerate}

\end{enumerate}
\end{document}
