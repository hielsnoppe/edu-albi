\documentclass{homework}
\usepackage{marvosym}
\usepackage{hyperref}
\usepackage{color}
\usepackage{caption}
\usepackage{subcaption}
\usepackage{float}

\course{Algorithmische Bioinformatik}
\semester{Wintersemester 2012 / 2013}
\no{9}
\date{Montag, dem 17. Dezember 2012}
\author{Stefan Meißner (4279113) und Niels Hoppe (4356370)}
\tutorial{Dienstag 08:00 - 10:00}
\tutor{Alena van Bömmel (Übungsgruppe 3)}

\begin{document}
\maketitle
\begin{enumerate} 

\aufgabe{Overlap-Graph}{}

% 1) Overlap phase -> alle Paare vergleichen; Overlap Graph
% 2) Layout phase -> Position von jedem Read
% 3) Consensus phase -> Multialignment, Final sequence

\begin{enumerate}
\item
\item
\item Aus dem MST ergibt sich folgendes Layout:

\begin{figure}[h]
\setlength{\unitlength}{0.05mm}
\centering

\begin{picture}(1400,800)(0,0)
\put(0,000){\vector(1,0){500}}
\put(50,100){\vector(1,0){500}}
\put(750,200){\vector(-1,0){500}}
\put(1100,300){\vector(-1,0){500}}
\put(700,400){\vector(1,0){500}}
\put(1020,500){\vector(1,0){500}}
\put(1630,600){\vector(-1,0){500}}
\put(1870,700){\vector(-1,0){500}}
\end{picture}
\caption{Layout}
\label{fig:Layout}
\end{figure}

\item
\end{enumerate}

\aufgabe{Sequencing by Hybridisation}{}

Habe ich auf Papier fertig, füge ich dann ein.

\aufgabe{Euler-Zyklus/Hammilton}{}

Es gibt einen Euler-Pfad (s. Abb. \ref{fig:Euler-Pfad}), der jedoch nicht zyklisch ist.
Ein Euler-Zyklus ist nicht möglich, da der Graph ungerichtet ist und es ungerade Knotengrade bei $(0,0)$ und $(1,0)$ gibt.
Ein Hammilton-Zyklus lässt sich jedoch einfach finden (s. Abb. \ref{fig:Hammilton-Zyklus}).

\begin{figure}[h]
\setlength{\unitlength}{2.0cm}
\centering

\begin{subfigure}[b]{0.49\linewidth}
\centering
\begin{picture}(1,2)
\put(0,0){\circle*{0.1}}
\put(0,1){\circle*{0.1}}
\put(1,0){\circle*{0.1}}
\put(1,1){\circle*{0.1}}
\put(0.5,1.75){\circle*{0.1}}
\put(0.5,0.5){\circle*{0.1}}
\thicklines
\put(0,0){\vector(0,1){1}}
\put(0,1){\vector(2,3){0.5}}
\put(0.5,1.75){\vector(2,-3){0.5}}
\put(1,1){\vector(0,-1){1}}
\put(1,0){\vector(-1,0){1}}
\put(0,0){\vector(1,1){0.5}}
\put(0.5,0.5){\vector(1,1){0.5}}
\put(1,1){\vector(-1,0){1}}
\put(0,1){\vector(1,-1){0.5}}
\put(0.5,0.5){\vector(1,-1){0.5}}
\end{picture}
\caption{Euler-Pfad}
\label{fig:Euler-Pfad}
\end{subfigure}
\begin{subfigure}[b]{0.49\linewidth}
\centering
\begin{picture}(1,2)
\put(0,0){\circle*{0.1}}
\put(0,1){\circle*{0.1}}
\put(1,0){\circle*{0.1}}
\put(1,1){\circle*{0.1}}
\put(0.5,1.75){\circle*{0.1}}
\put(0.5,0.5){\circle*{0.1}}
\thicklines
\put(0,0){\vector(0,1){1}}
\put(0,1){\vector(2,3){0.5}}
\put(0.5,1.75){\vector(2,-3){0.5}}
\put(1,1){\vector(0,-1){1}}
\put(1,0){\vector(-1,1){0.5}}
\put(0.5,0.5){\vector(-1,-1){0.5}}
\end{picture}
\caption{Hammilton-Zyklus}
\label{fig:Hammilton-Zyklus}
\end{subfigure}

\caption{Das Haus des Nikolaus}
\end{figure}

\aufgabe{Poissonprozess}{}

% Poissonprozess mit Intensität \lambda > 0; \{X(t): E \geq 0\}
% 1) X(0) = 0
% 2) Independent increment: t_0 = 0 < t_1 < t_2 < \ldots < t_n
% [X(t_1)-X(t_0)];[X(t_2)-X(t_1)]; \ldots [X(t_n)-X(t_{n-1})]
% unabhängige ZV
% 3) Homogenity in time: s \geq 0, t \geq 0 ist die ZV
% [X(s+t)-X(s)] \tilde P_0(t \cdot \lambda), d.h.
% P\{X(s+t)-X(s)=k\} = \frac{(\lambda t)^k \cdot e^{-\lambda t}}{k!}
% k=0,1,2,\ldots
% P(X_t(\frac{1}{2}) = 1, X(\frac{5}{2} = 5)
% = P(X(\frac{1}{2})=1, X(\frac{5}{2} - X(\frac{1}{2})=4)

\begin{enumerate}
\item
\item
\item
\end{enumerate}


\end{enumerate}
\end{document}
