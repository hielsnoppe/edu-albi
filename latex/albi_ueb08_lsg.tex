\documentclass{homework}
\usepackage{marvosym}
\usepackage{hyperref}
\usepackage{color}

\course{Algorithmische Bioinformatik}
\semester{Wintersemester 2012 / 2013}
\no{8}
\date{Montag, dem 10. Dezember 2012}
\author{Stefan Meißner (4279113) und Niels Hoppe (4356370)}
\tutorial{Dienstag 08:00 - 10:00}
\tutor{Alena van Bömmel (Übungsgruppe 3)}

\begin{document}
\maketitle
\begin{enumerate} 

\aufgabe{}{}

Nussinov

w(i,j) = score, max. # Basenpaare für Subseqeunz x_i, ..., x_j
\delta (i, j) = \begin{cases}
1 & \text{ if } i \cdot j\\
0 & \text{ else }
\end{cases}

Initialisierung
w(i, i) = 0; w(i, i-1) = 0; i = 2, ..., L

Rekursion
alle Subsequenzen der Länge 2 ... L
w(i, j) = \max \begin{cases}
w(i+1, j)								& (1)\\
w(i, j-1)								& (2)\\
w(i+1, j-1) + \delta(i, j)				& (3)\\
\max_{i < k < j} w(i, k) + w(k+1, j)	& (4)
\end{cases}

Traceback

push(1, L)

pop(i, j)
if j \geq i: continue
elif w(i+1,j) = w(i, j): push(i+1, j)
elif w(i, i-1) = W[i, j]: push(i, j-1)
elif W[i+1, j-1] + delta(i, j) = W[i, j]:
	push(i+1, j-1)
	print(i, j)
else:
	for k in range(i+1, j-1):
		if W[i, k] + W[k+1, j] = W[i, j]:
			push(i, k)
			push(k+1, j)
			break



Information content IC von (P(X)M)_{PSSM}

self-information - \log_2(P_{i,j})
(avg) expected information -P_{i,j} \log_2(P{i,j}
IC entropy - \sum_{i,j} P{i,j} \cdot \log_2(P_{i,j}) \Rightarrow (i)
relative entropy \sum_{i,j} P_{i,j} \cdot \log_2(\frac{P_{i,j}}{P_B})
f_{i,j](a,b) = H_{i,j} = - \sum_{a,b \in \mathcal{A}} f_{i,j}(a,b) \cdot \log_2 \frac{f_{i,j}(a,b)}{f_i(a) \cdot f_j(b)}
Häufigkeit von Paar (a,b) in Spalten i und j


\end{enumerate}
\end{document}
