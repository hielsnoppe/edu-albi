\documentclass{homework}
\usepackage{marvosym}
\usepackage{hyperref}
\usepackage{color}
\usepackage{caption}
\usepackage{subcaption}
\usepackage{float}

\course{Algorithmische Bioinformatik}
\semester{Wintersemester 2012 / 2013}
\no{14}
\date{Montag, dem 4. Februar 2013}
\author{Stefan Meißner (4279113) und Niels Hoppe (4356370)}
\tutorial{Dienstag 08:00 - 10:00}
\tutor{Alena van Bömmel (Übungsgruppe 3)}

\begin{document}
\maketitle
\begin{enumerate} 

\aufgabe{Isotopenverteilung}{50 + 100}
\begin{enumerate}
\item
Wir berechnen die Isotopenverteilung für $C_{60}$ mit Hilfe der Bernoulli-Verteilung.
Dafür verwenden wir $n = 60$ und $p = 0.01$ als Wahrscheinlichkeit für ein $^{13}C$.
$k$ gibt die Anzahl vorkommender $^{13}C$ an.
Daraus ergibt sich die Isotopenverteilung in Tabelle \ref{tab:a49i}.

\begin{table}[H]
\centering
\begin{tabular}{clc}
$\#^{13}C$	& $Pr$	& mass\\\hline
0	& 0.54716	& 720\\
1	& 0.33161	& 721\\
2	& 0.09881	& 722\\
3	& 0.0193	& 723\\
4	& 0.00278	& 724\\
5	& 0.00031	& 725\\
6	& 0.00003	& 726\\
\ldots	& $\leq$ 0.00001	& $\geq$ 727\\
\end{tabular}
\caption{Isotopenverteilung für $C_{60}$}
\label{tab:a49i}
\end{table}

\item 
\item 
\end{enumerate}

\aufgabe{Proteinidentifizierung}{100}
\begin{enumerate}
\item (\ldots.0875.2.dat)
Zur Identifizierung haben wir zuerst die NCBInr-Datenbank mit allen Taxa durchsucht.
Um die Signifikanz zu verbessern haben wir dann verschiedene Kombinationen der Parameter für die Massentoleranzen probiert,
bis wir eine relativ gute Signifikanz, allerdings noch keine Identifikation erhielten.
Mit diesen Parametern ($pmt = \pm 0.9 Da$ und $fmt = \pm 0.6 Da$) haben wir anschließen die SwissProt-Datebank durchsucht und dabei einen Treffer bei den Säugetieren gefunden.
Diesen Treffer konnten wir anschließend auch in der NCBInr-Datenbank unter Einschränkung auf Säugetiere finden.

Identifiziert wurde das Peptid \texttt{K.DNPQTHYYAVAVVK.K} aus dem Protein \textbf{Serotransferrin} mit einem Score von \textbf{46}.
\item (\ldots.1449.2.dta)
Unter Beibehaltung der Parameter für die Massentoleranzen haben wir mit dem gleichen Vorgehen das Peptid
\texttt{K.VIPAADLSEQISTAGTEASGTGNMK.F} aus dem Protein \textbf{Glycogen phosphorylase} mit einem Score von \textbf{41} identifiziert.
\item (\ldots.1883.2.dta)
Auch mit kleineren Modifikationen der Parameter konnten wir keine eindeutige Identifizierung vornehmen.
Den besten Treffer mit einem Score von \textbf{30} brachte das Peptid \texttt{VYVEELKPTPEGDLEILLQK}.
\end{enumerate}

\aufgabe{Feature-Based Alignment}{80}

\end{enumerate}
\end{document}
