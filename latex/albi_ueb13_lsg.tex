\documentclass{homework}
\usepackage{marvosym}
\usepackage{hyperref}
\usepackage{color}
\usepackage{caption}
\usepackage{subcaption}
\usepackage{float}

\course{Algorithmische Bioinformatik}
\semester{Wintersemester 2012 / 2013}
\no{13}
\date{Montag, dem 28. Januar 2013}
\author{Stefan Meißner (4279113) und Niels Hoppe (4356370)}
\tutorial{Dienstag 08:00 - 10:00}
\tutor{Alena van Bömmel (Übungsgruppe 3)}

\begin{document}
\maketitle
\begin{enumerate} 

\aufgabe{ChIP-Seq}{100}
\begin{enumerate}
\item 
\item 
\item 
\item 
\item 
\item 
\item 
\end{enumerate}

\aufgabe{Proteomics}{}

\begin{enumerate}
\item
Eine genaue Bestimmung des Verhältnisses der Masse zur Ladung von Ionen und Ionengemischen wie z.B. Peptiden,
die aus Proteinen durch enzymatische Verdauung gewonnen werden.
Mit dem entstehenden Spektrum kann dann eine Proteinidentifikation durchgeführt,
oder auch eine komplette Proteinsequenz berechnet werden.
\item
Ein ganzes Protein ist zur Sequenzierung schlichtweg zu groß.
Zerteilt man es durch eine enzymatische Verdauung in Peptide,
so lassen sich diese Peptide nicht einfach zu einer Sequenz zusammenfügen,
da sie durch die spezifischen Schnittstellen des verwendeten Enzyms keine Überlapps besitzen.
Stattdessen werden die Peptide erneut fragmentiert und durch eine weitere Massenspektrometrie die Peptidsequenz bestimmt.
\item
Der Nobelpreis wurde für zwei unterschiedliche Leistungen vergeben.
Fenn und Tanaka erhielten ihn dafür, dass es ihnen gelang,
ein komplettes Protein zu ionisieren und massenspektrometrisch zu untersuchen.
Dazu ist die geschickte Wahl von Trägermatrix und Wellenlänge und Intensität des Lasers nötig.
Wüthrichs Leistung steht nicht in direkter Verbindung zu Proteomics.
\item
Ja, das ist möglich.
In der Praxis ist dafür jedoch einerseits ein sehr genaues Massenspektrometer nötig,
um z.B. die Aminosäuren K und Q unterscheiden zu können und die Aminosäuren Leucin und Isoleucin unterscheiden sich in ihrer Masse gar nicht.
\item Die $a_m$, $b_m$ und $c_m$ Ionen sind Burchteile der Peptide.
Sie entstehen durch die Fragmentierung und unterscheiden sich darin,
dass die Peptide an unterschiedlichen Stellen auseinander brechen.
\item
Die $x_{(n-m)}$, $y_{(n-m)}$ und $z_{(n-m)}$ Ionen korrelieren jeweils mit den $a_m$, $b_m$ bzw. $c_m$ Ionen insofern,
als dass sie das jeweils verbleibende Bruchstück darstellen,
das das zugehörige $a_m$, $b_m$ bzw. $c_m$ Ion zum vollständigen Peptid ergänzt.
\item \begin{description}
\item[Abbildung a]
zeigt ein Chromatogramm, das durch Aufteilung der Peptide z.B. in einer Säule,
die die Peptide je nach ihrer Wechselwirkung mit dem Material der Säule
in unterschiedlichen Zeiten (x-Achse, Retentionszeit in Minuten) passieren, entsteht.
Die Intensität (y-Achse) gibt an, wie viele Ladungen pro Sekunde den Detektor passieren.
\item[Abbildung b]
zeigt das Spektrum der Peptidmischung, die zu der in Abbildung a hervorgehobenen Zeitspanne von der Säule eluiert.
Die x-Achse zeigt dabei das Verhältnis Masse zu Ladung (Dalton pro Elementarladung),
während die y-Achse die jeweilige Intensität als Anzahl von Einschlägen auf dem Detektor darstellt.
\item[Abbildung c]
zeigt das Fragmentspektrum der aus dem in Abbildung b hervorgehobenen Peptid gewonnenen Fragmenten
mit aufgrund der beobachteten Massendifferenzen zugeordneten Aminosäuresequenz.
Die Achsen entsprechend Abbildung b.
\end{description}
\item
Die absolute Quantifizierung gibt die absolute Menge des untersuchten Proteins bzw. der jeweiligen Peptide an,
während die relative Quantifizierung das Mengenverhältnis zwischen Peptiden eines Proteins
in unterschiedlichen Zuständen (z.B. krank/gesund, behandelt/unbeandelt, \ldots) ermittelt.
\end{enumerate}

\end{enumerate}
\end{document}
