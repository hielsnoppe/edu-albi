\documentclass{homework}
\usepackage{marvosym}
\usepackage{hyperref}
\usepackage{color}
\usepackage{caption}
\usepackage{subcaption}
\usepackage{float}

\course{Algorithmische Bioinformatik}
\semester{Wintersemester 2012 / 2013}
\no{9}
\date{Montag, dem 17. Dezember 2012}
\author{Stefan Meißner (4279113) und Niels Hoppe (4356370)}
\tutorial{Dienstag 08:00 - 10:00}
\tutor{Alena van Bömmel (Übungsgruppe 3)}

\begin{document}
\maketitle
\begin{enumerate} 

\aufgabe{Overlap-Graph}{\qquad}

% 1) Overlap phase -> alle Paare vergleichen; Overlap Graph
% 2) Layout phase -> Position von jedem Read
% 3) Consensus phase -> Multialignment, Final sequence

\begin{enumerate}
\item Siehe (b)
\item Wir erhalten einen gewichteten Overlap-Graphen und berechnen darin einen minimum spanning tree:

\begin{figure}[H]
\setlength{\unitlength}{1.5cm}
\centering

\begin{subfigure}{0.5\linewidth}
\centering

\begin{picture}(3,3)(0,0)
\put(1,0){$f_1$}
\put(2,0){$f_2$}
\put(3,1){$f_3$}
\put(3,2){$f_4$}
\put(2,3){$f_5$}
\put(1,3){$f_6$}
\put(0,2){$f_7$}
\put(0,1){$f_8$}

\tiny
\put(1,0){\line(1,0){1}} \put(1.5,0){$-40$}			% 1,2
\put(1,0){\line(0,1){3}} \put(1,1.5){$-450$}		% 1,6
\put(1,0){\line(-1,2){1}} \put(0.5,1){$-250$}		% 1,7
\put(2,0){\line(1,1){1}} \put(2.5,0.5){$-150$}		% 2,3
\put(2,0){\line(1,2){1}} \put(2.5,1){$-260$}		% 2,4
\put(3,1){\line(0,1){1}} \put(3,1.5){$-390$}		% 3,4
\put(3,1){\line(-1,2){1}} \put(2.5,2){$-80$}		% 3,5
\put(3,1){\line(-1,0){3}} \put(1.5,1){$-180$}		% 3,8
\put(3,2){\line(-3,-1){3}} \put(1.5,1.5){$-70$}		% 4,8
\put(2,3){\line(-2,-1){2}} \put(1,2.5){$-150$}		% 5,7
\put(2,3){\line(-1,-1){2}} \put(1,2){$-400$}		% 5,8
\put(1,3){\line(-1,-1){1}} \put(0.5,2.5){$-300$}	% 6,7
\put(0,2){\line(0,-1){1}} \put(0,1.5){$-50$}		% 7,8
\end{picture}

\caption{Overlap-Graph}
\label{fig:30i}
\end{subfigure}%
\begin{subfigure}{0.5\linewidth}
\centering

\begin{picture}(3,3)(0,0)
\put(1,0){$f_1$}
\put(2,0){$f_2$}
\put(3,1){$f_3$}
\put(3,2){$f_4$}
\put(2,3){$f_5$}
\put(1,3){$f_6$}
\put(0,2){$f_7$}
\put(0,1){$f_8$}

\tiny
\put(1,0){\line(0,1){3}} \put(1,1.5){$-450$}		% 1,6
\put(2,0){\line(1,2){1}} \put(2.5,1){$-260$}		% 2,4
\put(3,1){\line(0,1){1}} \put(3,1.5){$-390$}		% 3,4
\put(3,1){\line(-1,0){3}} \put(1.5,1){$-180$}		% 3,8
\put(2,3){\line(-2,-1){2}} \put(1,2.5){$-150$}		% 5,7
\put(2,3){\line(-1,-1){2}} \put(1,2){$-400$}		% 5,8
\put(1,3){\line(-1,-1){1}} \put(0.5,2.5){$-300$}	% 6,7
\end{picture}

\caption{Minimum spanning tree}
\label{fig:30ii}
\end{subfigure}

\caption{Darstellung der Reads als Graphen}
\end{figure}

\item Aus dem minimum spanning tree ergibt sich folgendes Layout:

\begin{figure}[H]
\setlength{\unitlength}{0.05mm}
\centering

\begin{picture}(1870,800)(0,0)
\footnotesize
\put(0,000){\vector(1,0){500}} \put(600,000){$f_1: (0,500)$}
\put(50,100){\vector(1,0){500}} \put(650,100){$f_6: (50,550)$}
\put(750,200){\vector(-1,0){500}} \put(850,200){$f_7: (750,250)$}
\put(1100,300){\vector(-1,0){500}} \put(1200,300){$f_5: (1100,600)$}
\put(700,400){\vector(1,0){500}} \put(1300,400){$f_8: (700,1200)$}
\put(1020,500){\vector(1,0){500}} \put(520,500){$f_3: (1020,1520)$}
\put(1630,600){\vector(-1,0){500}} \put(630,600){$f_4: (1630,1130)$}
\put(1870,700){\vector(-1,0){500}} \put(870,700){$f_2: (1870,1370)$}
\end{picture}

\caption{Layout und Koordinaten der Reads}
\label{fig:30iii}
\end{figure}

\item Zu überprüfen bleiben die Kanten, die nicht im minimum spanning tree enthalten sind:

\begin{description}
\item[$(f_1, f_2)$:] $500 - 1370 = -870 \neq -40$
\item[$(f_1, f_7)$:] $250 - 500 = -250$
\item[$(f_2, f_3)$:] $1370 - 1520 = -250$
\item[$(f_3, f_5)$:] $600 - 1520 = -920 \neq -80$
\item[$(f_4, f_8)$:] $1130 - 1200 = -70$
\item[$(f_7, f_8)$:] $700 - 750 = -50$
\end{description}

Es gibt also Inkonsistenzen bei $(f_1, f_2)$ und $(f_3, f_5)$.
Diese könnten zum Beispiel auf Sequenzierfehler oder lange Repeats zurückzuführen sein.

\end{enumerate}

\aufgabe{Sequencing by Hybridisation}{\qquad}

Wir erstellen zuerst einen Overlap-Graphen und verwenden anschließend den Euler-Trick,
um die ursprüngliche Sequenz zu rekonstruieren.

\begin{figure}[H]
\setlength{\unitlength}{1.5cm}
\centering

\begin{subfigure}{0.5\linewidth}
\centering
\begin{picture}(3,3)(0,0)
\put(1,0){\texttt{GCA}}
\put(2,0){\texttt{CGC}}
\put(3,1){\texttt{CGT}}
\put(3,2){\texttt{CCG}}
\put(2,3){\texttt{CAC}}
\put(1,3){\texttt{ACG}}
\put(0,2){\texttt{ACC}}
\put(0,1){\texttt{TAC}}

\put(1,0){\vector(1,3){1}}
\put(2,0){\vector(-1,0){1}}
\put(3,2){\vector(0,-1){1}}
\put(2,3){\vector(-1,0){1}}
\put(2,3){\vector(-2,-1){2}}
\put(1,3){\vector(1,-3){1}}
\put(1,3){\vector(1,-1){2}}
\put(0,2){\vector(1,0){3}}
\put(0,1){\vector(0,1){1}}
\put(0,1){\vector(1,2){1}}
\end{picture}

\caption{Overlap-Graph}
\label{fig:31a}
\end{subfigure}%
\begin{subfigure}{0.5\linewidth}
\centering

\begin{picture}(3,3)(0,0)
\put(1,0){\texttt{GC}}
\put(2,0){\texttt{CA}}
\put(3,1){\texttt{CG}}
\put(3,2){\texttt{GT}}
\put(2,3){\texttt{CC}}
\put(1,3){\texttt{AC}}
\put(0,2){\texttt{TA}}

\put(1,0){\vector(1,0){1}}
\put(2,0){\vector(-1,3){1}}
\put(3,1){\vector(-2,-1){2}}
\put(3,1){\vector(0,1){1}}
\put(2,3){\vector(1,-2){1}}
\put(1,3){\vector(1,0){1}}
\put(1,3){\vector(1,-1){2}}
\put(0,2){\vector(1,1){1}}
\end{picture}

\caption{Euler-Trick}
\label{fig:31b}
\end{subfigure}

\caption{Sequencing by Hybridisation}
\end{figure}

Aus beiden Graphen lässt sich \texttt{TACCGCACGT} als zusammengesetzte Sequenz ablesen.


\aufgabe{Euler-Zyklus/Hammilton}{\qquad}

Es gibt einen Euler-Pfad (s. Abb. \ref{fig:32a}), der jedoch nicht zyklisch ist.
Ein Euler-Zyklus ist nicht möglich, da der Graph ungerichtet ist und es ungerade Knotengrade bei $(0,0)$ und $(1,0)$ gibt.
Ein Hammilton-Zyklus lässt sich jedoch einfach finden (s. Abb. \ref{fig:32b}).

\begin{figure}[H]
\setlength{\unitlength}{2.0cm}
\centering

\begin{subfigure}{0.5\linewidth}
\centering

\begin{picture}(1,2)
\put(0,0){\circle*{0.1}}
\put(0,1){\circle*{0.1}}
\put(1,0){\circle*{0.1}}
\put(1,1){\circle*{0.1}}
\put(0.5,1.75){\circle*{0.1}}
\put(0.5,0.5){\circle*{0.1}}
\thicklines
\put(0,0){\vector(0,1){1}}
\put(0,1){\vector(2,3){0.5}}
\put(0.5,1.75){\vector(2,-3){0.5}}
\put(1,1){\vector(0,-1){1}}
\put(1,0){\vector(-1,0){1}}
\put(0,0){\vector(1,1){0.5}}
\put(0.5,0.5){\vector(1,1){0.5}}
\put(1,1){\vector(-1,0){1}}
\put(0,1){\vector(1,-1){0.5}}
\put(0.5,0.5){\vector(1,-1){0.5}}
\end{picture}

\caption{Euler-Pfad}
\label{fig:32a}
\end{subfigure}%
\begin{subfigure}{0.5\linewidth}
\centering

\begin{picture}(1,2)
\put(0,0){\circle*{0.1}}
\put(0,1){\circle*{0.1}}
\put(1,0){\circle*{0.1}}
\put(1,1){\circle*{0.1}}
\put(0.5,1.75){\circle*{0.1}}
\put(0.5,0.5){\circle*{0.1}}
\thicklines
\put(0,0){\vector(0,1){1}}
\put(0,1){\vector(2,3){0.5}}
\put(0.5,1.75){\vector(2,-3){0.5}}
\put(1,1){\vector(0,-1){1}}
\put(1,0){\vector(-1,1){0.5}}
\put(0.5,0.5){\vector(-1,-1){0.5}}
\end{picture}

\caption{Hammilton-Zyklus}
\label{fig:32b}
\end{subfigure}

\caption{Das Haus vom Nikolaus}
\end{figure}

\aufgabe{Poissonprozess}{\qquad}

% Poissonprozess mit Intensität \lambda > 0; \{X(t): E \geq 0\}
% 1) X(0) = 0
% 2) Independent increment: t_0 = 0 < t_1 < t_2 < \ldots < t_n
% [X(t_1)-X(t_0)];[X(t_2)-X(t_1)]; \ldots [X(t_n)-X(t_{n-1})]
% unabhängige ZV
% 3) Homogenity in time: s \geq 0, t \geq 0 ist die ZV
% [X(s+t)-X(s)] \tilde P_0(t \cdot \lambda), d.h.
% P\{X(s+t)-X(s)=k\} = \frac{(\lambda t)^k \cdot e^{-\lambda t}}{k!}
% k=0,1,2,\ldots
% P(X_t(\frac{1}{2}) = 1, X(\frac{5}{2} = 5)
% = P(X(\frac{1}{2})=1, X(\frac{5}{2} - X(\frac{1}{2})=4)

\begin{enumerate}
\item \begin{eqnarray*}
P_i(k=1, \lambda = \frac{4}{h}, t=0.5h)
& = & e^{-\lambda t} \cdot \frac{(\lambda t)^k}{k!}\\
& = & e^{-4 \cdot 0,5} \cdot \frac{(4 \cdot 0.5)^1}{1!}\\
& = & 0.271
\end{eqnarray*}
\item \begin{eqnarray*}
P_{ii}(k=4, \lambda = \frac{4}{h}, t=2h)
& = & e^{-4 \cdot 2} \cdot \frac{(4 \cdot 2)^4}{4!}\\
& = & 0.057\\
P & = & P_i \cdot P_{ii} = 0.015\\
\end{eqnarray*}
\item \begin{eqnarray*}
P_{iii}(k=5, \lambda = \frac{4}{h}, t=5.5h)
& = & e^{-4 \cdot 5,5} \cdot \frac{(4 \cdot 5.5)^5}{5!}\\
& = & 0.00001\\
P & = & P_i \cdot P_{ii} \cdot P_{iii} = 1.8 \cdot 10^{-7}\\
\end{eqnarray*}
\end{enumerate}


\end{enumerate}
\end{document}
