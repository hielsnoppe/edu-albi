\documentclass{homework}
\usepackage{marvosym}
\usepackage{hyperref}
\usepackage{color}

\course{Algorithmische Bioinformatik}
\semester{Wintersemester 2012 / 2013}
\no{7}
\date{Montag, dem 3. Dezember 2012}
\author{Stefan Meißner (4279113) und Niels Hoppe (4356370)}
\tutorial{Dienstag 08:00 - 10:00}
\tutor{Alena van Bömmel (Übungsgruppe 3)}

\begin{document}
\maketitle
\begin{enumerate} 

\aufgabe{Gene finding}{100}

Abgabe am 10. Dezember 2012

%\begin{enumerate}
%\item
%\item
%\item
%\item
%\item 
%\end{enumerate}

\aufgabe{Coding capacity}{60}
Für ein Codon $ABC$ mit einer absoluten Häufigkeit von $M_{ABC}$ auf $K$
gezählte Codons gilt
$$F_{ABC} = \ln f_{ABC} = \ln \frac{M_{ABC}}{K}.$$

Die Scores $i \in {1,2,3}: H_i$ für ein Fenster $w$ berechnen sich als
$$H_i = \sum_{ABC \in w} F_{ABC}.$$

Daraus berechnen sich die Wahrscheinlichkeiten $i \in {1,2,3}: P_i$ dafür, dass
der Leserahmen $i$ in dem Fenster ein kodierender Abschnitt der Sequenz ist als
$$P_i = \frac{\exp H_i}{\exp H_1 + \exp H_2 + \exp H_3}.$$

%$$\log \frac{P_i}{1-P_i}$$

Diese Formeln setzen wir um in folgendem Programm. Dabei ist $M$ ein
\textit{dictionary} mit Codons $ABC$ als Schlüssel und Werten $M_{ABC}$.

\begin{lstlisting}[language=python]
def MF(M, K):
	F = {}
	for codon, m in M.iteritems():
		if m == 0.0: m = 1.0
		F[codon] = math.log(m / float(K))
	return F

def winscore(seq, F):
	return sum([F[seq[n:n+3]] for n in range(0, len(seq), 3)])

def analyze(seq, winsize):
	data = []
	winsize = 3 * winsize	# window size in nucletides
	H, P = [0.0, 0.0, 0.0], [0.0, 0.0, 0.0]
	F = MF(M, 1000)			# prepared frequencies

	step = 0
	win = seq[:winsize]		# current window
	
	for n in seq[winsize:]:
		rfn = step % 3		# reading frame number is 0, 1 or 2
		H[rfn] = winscore(win, F)

		if rfn == 2:
			A = sum(H) / float(len(H))
			H = map(lambda Hi: math.exp(Hi - A), H)
			P = map(lambda Hi: Hi / sum(H), H)
			P = map(lambda Pi: math.log10(Pi / (1.0 - Pi)), P)
			data.append((step, P[0], P[1], P[2]))

		step = step + 1;
		win = win[1:] + n
	return data
\end{lstlisting}

\begin{enumerate}
\item Der Wertebereich reichte von Werten kleiner $10^{-10}$ bis hin zu Werten
größer $10^{10}$. Eine geeignete Darstellung wäre nur mit Hilfe einer
logarithmischen Skala möglich gewesen und wäre dann wiederum identisch mit der
Darstellung in (b).

\item In der folgenden Abbildung ist der erste Leserahmen rot, der zweite grün
und der dritte blau dargestellt.

\begin{center}
\setlength{\unitlength}{0.3mm}
\begin{picture}(69.6666666667,150)(0,-75)
\put(0,0){\vector(1,0){209.0}}
\put(0,0){\vector(0,1){75}}
\put(0,0){\vector(0,-1){75}}
\color{red}
\put(0.666666666667,23.7901882729){\circle*{0.1}}
\color{green}
\put(0.666666666667,-26.7715651142){\circle*{0.1}}
\color{blue}
\put(0.666666666667,-24.42462845){\circle*{0.1}}
\color{red}
\put(1.66666666667,18.2830148268){\circle*{0.1}}
\color{green}
\put(1.66666666667,-18.2941308733){\circle*{0.1}}
\color{blue}
\put(1.66666666667,-29.7435428857){\circle*{0.1}}
\color{red}
\put(2.66666666667,16.9485989311){\circle*{0.1}}
\color{green}
\put(2.66666666667,-16.9607908646){\circle*{0.1}}
\color{blue}
\put(2.66666666667,-28.2098608035){\circle*{0.1}}
\color{red}
\put(3.66666666667,12.2290602161){\circle*{0.1}}
\color{green}
\put(3.66666666667,-12.2296238577){\circle*{0.1}}
\color{blue}
\put(3.66666666667,-30.1736253709){\circle*{0.1}}
\color{red}
\put(4.66666666667,14.9305660104){\circle*{0.1}}
\color{green}
\put(4.66666666667,-14.9306486036){\circle*{0.1}}
\color{blue}
\put(4.66666666667,-37.0341410009){\circle*{0.1}}
\color{red}
\put(5.66666666667,15.6559706856){\circle*{0.1}}
\color{green}
\put(5.66666666667,-15.6560448688){\circle*{0.1}}
\color{blue}
\put(5.66666666667,-37.9914618231){\circle*{0.1}}
\color{red}
\put(6.66666666667,12.0138374539){\circle*{0.1}}
\color{green}
\put(6.66666666667,-12.0139202217){\circle*{0.1}}
\color{blue}
\put(6.66666666667,-34.125488924){\circle*{0.1}}
\color{red}
\put(7.66666666667,10.0163494898){\circle*{0.1}}
\color{green}
\put(7.66666666667,-10.0165349042){\circle*{0.1}}
\color{blue}
\put(7.66666666667,-30.402383454){\circle*{0.1}}
\color{red}
\put(8.66666666667,3.7615110041){\circle*{0.1}}
\color{green}
\put(8.66666666667,-3.76153385474){\circle*{0.1}}
\color{blue}
\put(8.66666666667,-29.3581466441){\circle*{0.1}}
\color{red}
\put(9.66666666667,-0.469175336472){\circle*{0.1}}
\color{green}
\put(9.66666666667,0.469174117919){\circle*{0.1}}
\color{blue}
\put(9.66666666667,-34.2901427007){\circle*{0.1}}
\color{red}
\put(10.6666666667,4.9789188056){\circle*{0.1}}
\color{green}
\put(10.6666666667,-4.97891887318){\circle*{0.1}}
\color{blue}
\put(10.6666666667,-42.9313148934){\circle*{0.1}}
\color{red}
\put(11.6666666667,-4.14019623641){\circle*{0.1}}
\color{green}
\put(11.6666666667,4.14019607368){\circle*{0.1}}
\color{blue}
\put(11.6666666667,-40.3682064383){\circle*{0.1}}
\color{red}
\put(12.6666666667,-1.53647777772){\circle*{0.1}}
\color{green}
\put(12.6666666667,1.5364761814){\circle*{0.1}}
\color{blue}
\put(12.6666666667,-33.9447728431){\circle*{0.1}}
\color{red}
\put(13.6666666667,-9.40156121039){\circle*{0.1}}
\color{green}
\put(13.6666666667,9.40153733468){\circle*{0.1}}
\color{blue}
\put(13.6666666667,-34.2523758616){\circle*{0.1}}
\color{red}
\put(14.6666666667,-2.64913401898){\circle*{0.1}}
\color{green}
\put(14.6666666667,2.64911917007){\circle*{0.1}}
\color{blue}
\put(14.6666666667,-29.5979231799){\circle*{0.1}}
\color{red}
\put(15.6666666667,0.161845594406){\circle*{0.1}}
\color{green}
\put(15.6666666667,-0.161876214491){\circle*{0.1}}
\color{blue}
\put(15.6666666667,-27.2670475025){\circle*{0.1}}
\color{red}
\put(16.6666666667,1.21332109344){\circle*{0.1}}
\color{green}
\put(16.6666666667,-1.2140666508){\circle*{0.1}}
\color{blue}
\put(16.6666666667,-20.4989225338){\circle*{0.1}}
\color{red}
\put(17.6666666667,3.94883149666){\circle*{0.1}}
\color{green}
\put(17.6666666667,-3.95442236786){\circle*{0.1}}
\color{blue}
\put(17.6666666667,-17.5496136606){\circle*{0.1}}
\color{red}
\put(18.6666666667,5.96029543309){\circle*{0.1}}
\color{green}
\put(18.6666666667,-5.99977030546){\circle*{0.1}}
\color{blue}
\put(18.6666666667,-14.9480622438){\circle*{0.1}}
\color{red}
\put(19.6666666667,5.99254066512){\circle*{0.1}}
\color{green}
\put(19.6666666667,-6.04625717241){\circle*{0.1}}
\color{blue}
\put(19.6666666667,-14.3130479487){\circle*{0.1}}
\color{red}
\put(20.6666666667,3.44700105244){\circle*{0.1}}
\color{green}
\put(20.6666666667,-3.55605952798){\circle*{0.1}}
\color{blue}
\put(20.6666666667,-10.7713858074){\circle*{0.1}}
\color{red}
\put(21.6666666667,1.78913974224){\circle*{0.1}}
\color{green}
\put(21.6666666667,-2.35766494975){\circle*{0.1}}
\color{blue}
\put(21.6666666667,-6.28167114956){\circle*{0.1}}
\color{red}
\put(22.6666666667,-2.60243403958){\circle*{0.1}}
\color{green}
\put(22.6666666667,2.27829402187){\circle*{0.1}}
\color{blue}
\put(22.6666666667,-7.73211620181){\circle*{0.1}}
\color{red}
\put(23.6666666667,-9.32121276055){\circle*{0.1}}
\color{green}
\put(23.6666666667,9.20899674527){\circle*{0.1}}
\color{blue}
\put(23.6666666667,-15.7573458811){\circle*{0.1}}
\color{red}
\put(24.6666666667,-12.7690349631){\circle*{0.1}}
\color{green}
\put(24.6666666667,12.4117298358){\circle*{0.1}}
\color{blue}
\put(24.6666666667,-16.5186279406){\circle*{0.1}}
\color{red}
\put(25.6666666667,-17.4655685387){\circle*{0.1}}
\color{green}
\put(25.6666666667,6.49437610737){\circle*{0.1}}
\color{blue}
\put(25.6666666667,-6.50973461778){\circle*{0.1}}
\color{red}
\put(26.6666666667,-5.62984556517){\circle*{0.1}}
\color{green}
\put(26.6666666667,-2.19747150694){\circle*{0.1}}
\color{blue}
\put(26.6666666667,1.47712446897){\circle*{0.1}}
\color{red}
\put(27.6666666667,-5.43555705921){\circle*{0.1}}
\color{green}
\put(27.6666666667,-0.986541464278){\circle*{0.1}}
\color{blue}
\put(27.6666666667,0.31348866727){\circle*{0.1}}
\color{red}
\put(28.6666666667,-3.72466964677){\circle*{0.1}}
\color{green}
\put(28.6666666667,1.41898231219){\circle*{0.1}}
\color{blue}
\put(28.6666666667,-3.15249582505){\circle*{0.1}}
\color{red}
\put(29.6666666667,-4.22691644092){\circle*{0.1}}
\color{green}
\put(29.6666666667,3.42831603367){\circle*{0.1}}
\color{blue}
\put(29.6666666667,-6.58209193231){\circle*{0.1}}
\color{red}
\put(30.6666666667,-4.14183410706){\circle*{0.1}}
\color{green}
\put(30.6666666667,3.0740493333){\circle*{0.1}}
\color{blue}
\put(30.6666666667,-5.75157810008){\circle*{0.1}}
\color{red}
\put(31.6666666667,-2.55609415465){\circle*{0.1}}
\color{green}
\put(31.6666666667,-5.36093837623){\circle*{0.1}}
\color{blue}
\put(31.6666666667,1.70056354093){\circle*{0.1}}
\color{red}
\put(32.6666666667,-0.91312220216){\circle*{0.1}}
\color{green}
\put(32.6666666667,-2.61611543343){\circle*{0.1}}
\color{blue}
\put(32.6666666667,-1.12801299808){\circle*{0.1}}
\color{red}
\put(33.6666666667,0.387499149811){\circle*{0.1}}
\color{green}
\put(33.6666666667,-8.61190598055){\circle*{0.1}}
\color{blue}
\put(33.6666666667,-0.550950816997){\circle*{0.1}}
\color{red}
\put(34.6666666667,-7.90879869435){\circle*{0.1}}
\color{green}
\put(34.6666666667,-14.3460606063){\circle*{0.1}}
\color{blue}
\put(34.6666666667,7.79391994643){\circle*{0.1}}
\color{red}
\put(35.6666666667,-16.2321674596){\circle*{0.1}}
\color{green}
\put(35.6666666667,-16.7748231946){\circle*{0.1}}
\color{blue}
\put(35.6666666667,14.9803595051){\circle*{0.1}}
\color{red}
\put(36.6666666667,-19.3975814804){\circle*{0.1}}
\color{green}
\put(36.6666666667,-23.1032062777){\circle*{0.1}}
\color{blue}
\put(36.6666666667,19.0353260199){\circle*{0.1}}
\color{red}
\put(37.6666666667,-20.1073619867){\circle*{0.1}}
\color{green}
\put(37.6666666667,-27.7318530376){\circle*{0.1}}
\color{blue}
\put(37.6666666667,20.0434573117){\circle*{0.1}}
\color{red}
\put(38.6666666667,-20.0323946713){\circle*{0.1}}
\color{green}
\put(38.6666666667,-28.6308066371){\circle*{0.1}}
\color{blue}
\put(38.6666666667,19.9913696767){\circle*{0.1}}
\color{red}
\put(39.6666666667,-24.731347919){\circle*{0.1}}
\color{green}
\put(39.6666666667,-37.5769780637){\circle*{0.1}}
\color{blue}
\put(39.6666666667,24.7254993028){\circle*{0.1}}
\color{red}
\put(40.6666666667,-32.1959833403){\circle*{0.1}}
\color{green}
\put(40.6666666667,-43.8298166121){\circle*{0.1}}
\color{blue}
\put(40.6666666667,32.1857747279){\circle*{0.1}}
\color{red}
\put(41.6666666667,-27.8761684476){\circle*{0.1}}
\color{green}
\put(41.6666666667,-48.0360869796){\circle*{0.1}}
\color{blue}
\put(41.6666666667,27.8759667258){\circle*{0.1}}
\color{red}
\put(42.6666666667,-28.1621490089){\circle*{0.1}}
\color{green}
\put(42.6666666667,-47.5791400365){\circle*{0.1}}
\color{blue}
\put(42.6666666667,28.1618650014){\circle*{0.1}}
\color{red}
\put(43.6666666667,-26.4351794065){\circle*{0.1}}
\color{green}
\put(43.6666666667,-57.0942325642){\circle*{0.1}}
\color{blue}
\put(43.6666666667,26.4351778034){\circle*{0.1}}
\color{red}
\put(44.6666666667,-24.8798777208){\circle*{0.1}}
\color{green}
\put(44.6666666667,-53.9423325682){\circle*{0.1}}
\color{blue}
\put(44.6666666667,24.8798743767){\circle*{0.1}}
\color{red}
\put(45.6666666667,-25.6170096003){\circle*{0.1}}
\color{green}
\put(45.6666666667,-58.4778435044){\circle*{0.1}}
\color{blue}
\put(45.6666666667,25.6170090187){\circle*{0.1}}
\color{red}
\put(46.6666666667,-30.5053382707){\circle*{0.1}}
\color{green}
\put(46.6666666667,-52.9669739678){\circle*{0.1}}
\color{blue}
\put(46.6666666667,30.5052683798){\circle*{0.1}}
\color{red}
\put(47.6666666667,-27.5223527929){\circle*{0.1}}
\color{green}
\put(47.6666666667,-48.6769049781){\circle*{0.1}}
\color{blue}
\put(47.6666666667,27.5222251979){\circle*{0.1}}
\color{red}
\put(48.6666666667,-34.6603174651){\circle*{0.1}}
\color{green}
\put(48.6666666667,-49.2162856676){\circle*{0.1}}
\color{blue}
\put(48.6666666667,34.6576549357){\circle*{0.1}}
\color{red}
\put(49.6666666667,-34.0693538121){\circle*{0.1}}
\color{green}
\put(49.6666666667,-49.2523012369){\circle*{0.1}}
\color{blue}
\put(49.6666666667,34.0673587087){\circle*{0.1}}
\color{red}
\put(50.6666666667,-35.7825456731){\circle*{0.1}}
\color{green}
\put(50.6666666667,-47.8100733702){\circle*{0.1}}
\color{blue}
\put(50.6666666667,35.7740265179){\circle*{0.1}}
\color{red}
\put(51.6666666667,-37.5801900326){\circle*{0.1}}
\color{green}
\put(51.6666666667,-48.119080844){\circle*{0.1}}
\color{blue}
\put(51.6666666667,37.5633133391){\circle*{0.1}}
\color{red}
\put(52.6666666667,-37.2419733695){\circle*{0.1}}
\color{green}
\put(52.6666666667,-47.6863819411){\circle*{0.1}}
\color{blue}
\put(52.6666666667,37.2243491868){\circle*{0.1}}
\color{red}
\put(53.6666666667,-38.6483488317){\circle*{0.1}}
\color{green}
\put(53.6666666667,-50.7909409915){\circle*{0.1}}
\color{blue}
\put(53.6666666667,38.6402685335){\circle*{0.1}}
\color{red}
\put(54.6666666667,-40.1851997){\circle*{0.1}}
\color{green}
\put(54.6666666667,-58.4618650231){\circle*{0.1}}
\color{blue}
\put(54.6666666667,40.1847195673){\circle*{0.1}}
\color{red}
\put(55.6666666667,-33.710918732){\circle*{0.1}}
\color{green}
\put(55.6666666667,-55.1397238034){\circle*{0.1}}
\color{blue}
\put(55.6666666667,33.7108062766){\circle*{0.1}}
\color{red}
\put(56.6666666667,-35.8454693829){\circle*{0.1}}
\color{green}
\put(56.6666666667,-50.6299955557){\circle*{0.1}}
\color{blue}
\put(56.6666666667,35.8430727058){\circle*{0.1}}
\color{red}
\put(57.6666666667,-34.2549433999){\circle*{0.1}}
\color{green}
\put(57.6666666667,-47.923304013){\circle*{0.1}}
\color{blue}
\put(57.6666666667,34.2509376683){\circle*{0.1}}
\color{red}
\put(58.6666666667,-38.6048016408){\circle*{0.1}}
\color{green}
\put(58.6666666667,-48.3190686447){\circle*{0.1}}
\color{blue}
\put(58.6666666667,38.580173276){\circle*{0.1}}
\color{red}
\put(59.6666666667,-41.658572679){\circle*{0.1}}
\color{green}
\put(59.6666666667,-51.4490952387){\circle*{0.1}}
\color{blue}
\put(59.6666666667,41.6347895286){\circle*{0.1}}
\color{red}
\put(60.6666666667,-44.6797463309){\circle*{0.1}}
\color{green}
\put(60.6666666667,-52.159090002){\circle*{0.1}}
\color{blue}
\put(60.6666666667,44.6115054261){\circle*{0.1}}
\color{red}
\put(61.6666666667,-45.3174040674){\circle*{0.1}}
\color{green}
\put(61.6666666667,-47.6042769509){\circle*{0.1}}
\color{blue}
\put(61.6666666667,44.6676030034){\circle*{0.1}}
\color{red}
\put(62.6666666667,-47.8788010487){\circle*{0.1}}
\color{green}
\put(62.6666666667,-52.8249965894){\circle*{0.1}}
\color{blue}
\put(62.6666666667,47.6668910073){\circle*{0.1}}
\color{red}
\put(63.6666666667,-47.5764164818){\circle*{0.1}}
\color{green}
\put(63.6666666667,-58.9317572356){\circle*{0.1}}
\color{blue}
\put(63.6666666667,47.564814823){\circle*{0.1}}
\color{red}
\put(64.6666666667,-47.7762329815){\circle*{0.1}}
\color{green}
\put(64.6666666667,-56.7708179725){\circle*{0.1}}
\color{blue}
\put(64.6666666667,47.7420028662){\circle*{0.1}}
\color{red}
\put(65.6666666667,-47.8103405792){\circle*{0.1}}
\color{green}
\put(65.6666666667,-54.0051323729){\circle*{0.1}}
\color{blue}
\put(65.6666666667,47.6885652208){\circle*{0.1}}
\color{red}
\put(66.6666666667,-46.1786734443){\circle*{0.1}}
\color{green}
\put(66.6666666667,-46.0829653706){\circle*{0.1}}
\color{blue}
\put(66.6666666667,44.625142273){\circle*{0.1}}
\color{red}
\put(67.6666666667,-45.8137519109){\circle*{0.1}}
\color{green}
\put(67.6666666667,-39.7919044552){\circle*{0.1}}
\color{blue}
\put(67.6666666667,39.6603331124){\circle*{0.1}}
\color{red}
\put(68.6666666667,-47.9434464101){\circle*{0.1}}
\color{green}
\put(68.6666666667,-42.6415797863){\circle*{0.1}}
\color{blue}
\put(68.6666666667,42.4603891049){\circle*{0.1}}
\color{red}
\put(69.6666666667,-46.3372498469){\circle*{0.1}}
\color{green}
\put(69.6666666667,-40.8435978016){\circle*{0.1}}
\color{blue}
\put(69.6666666667,40.6771519589){\circle*{0.1}}
\color{red}
\put(70.6666666667,-44.4636093318){\circle*{0.1}}
\color{green}
\put(70.6666666667,-35.7982841607){\circle*{0.1}}
\color{blue}
\put(70.6666666667,35.7585004853){\circle*{0.1}}
\color{red}
\put(71.6666666667,-41.9389351686){\circle*{0.1}}
\color{green}
\put(71.6666666667,-31.84052724){\circle*{0.1}}
\color{blue}
\put(71.6666666667,31.819873151){\circle*{0.1}}
\color{red}
\put(72.6666666667,-38.8834657157){\circle*{0.1}}
\color{green}
\put(72.6666666667,-25.2032996431){\circle*{0.1}}
\color{blue}
\put(72.6666666667,25.1993155403){\circle*{0.1}}
\color{red}
\put(73.6666666667,-35.3110686342){\circle*{0.1}}
\color{green}
\put(73.6666666667,-24.0556219036){\circle*{0.1}}
\color{blue}
\put(73.6666666667,24.0434750684){\circle*{0.1}}
\color{red}
\put(74.6666666667,-39.1216638504){\circle*{0.1}}
\color{green}
\put(74.6666666667,-29.0899065888){\circle*{0.1}}
\color{blue}
\put(74.6666666667,29.0686118181){\circle*{0.1}}
\color{red}
\put(75.6666666667,-40.345641092){\circle*{0.1}}
\color{green}
\put(75.6666666667,-26.3470828668){\circle*{0.1}}
\color{blue}
\put(75.6666666667,26.3436417201){\circle*{0.1}}
\color{red}
\put(76.6666666667,-39.0610075912){\circle*{0.1}}
\color{green}
\put(76.6666666667,-31.9492092302){\circle*{0.1}}
\color{blue}
\put(76.6666666667,31.8686137796){\circle*{0.1}}
\color{red}
\put(77.6666666667,-41.7689909957){\circle*{0.1}}
\color{green}
\put(77.6666666667,-41.9365233962){\circle*{0.1}}
\color{blue}
\put(77.6666666667,40.345991922){\circle*{0.1}}
\color{red}
\put(78.6666666667,-36.2721746176){\circle*{0.1}}
\color{green}
\put(78.6666666667,-41.8979900239){\circle*{0.1}}
\color{blue}
\put(78.6666666667,36.1152105179){\circle*{0.1}}
\color{red}
\put(79.6666666667,-44.012324494){\circle*{0.1}}
\color{green}
\put(79.6666666667,-41.386102814){\circle*{0.1}}
\color{blue}
\put(79.6666666667,40.81910987){\circle*{0.1}}
\color{red}
\put(80.6666666667,-36.8879163153){\circle*{0.1}}
\color{green}
\put(80.6666666667,-40.6710929263){\circle*{0.1}}
\color{blue}
\put(80.6666666667,36.5374836395){\circle*{0.1}}
\color{red}
\put(81.6666666667,-34.1769968028){\circle*{0.1}}
\color{green}
\put(81.6666666667,-42.849526186){\circle*{0.1}}
\color{blue}
\put(81.6666666667,34.1373436972){\circle*{0.1}}
\color{red}
\put(82.6666666667,-36.1748145043){\circle*{0.1}}
\color{green}
\put(82.6666666667,-47.7918950367){\circle*{0.1}}
\color{blue}
\put(82.6666666667,36.1645270201){\circle*{0.1}}
\color{red}
\put(83.6666666667,-38.2849963958){\circle*{0.1}}
\color{green}
\put(83.6666666667,-47.9122318531){\circle*{0.1}}
\color{blue}
\put(83.6666666667,38.2593667884){\circle*{0.1}}
\color{red}
\put(84.6666666667,-36.0508206031){\circle*{0.1}}
\color{green}
\put(84.6666666667,-39.6688523612){\circle*{0.1}}
\color{blue}
\put(84.6666666667,35.6749654528){\circle*{0.1}}
\color{red}
\put(85.6666666667,-36.9611068026){\circle*{0.1}}
\color{green}
\put(85.6666666667,-37.0317557935){\circle*{0.1}}
\color{blue}
\put(85.6666666667,35.4909939232){\circle*{0.1}}
\color{red}
\put(86.6666666667,-35.111388406){\circle*{0.1}}
\color{green}
\put(86.6666666667,-33.3654205871){\circle*{0.1}}
\color{blue}
\put(86.6666666667,32.5623067313){\circle*{0.1}}
\color{red}
\put(87.6666666667,-40.48221729){\circle*{0.1}}
\color{green}
\put(87.6666666667,-36.4589353727){\circle*{0.1}}
\color{blue}
\put(87.6666666667,36.1426456005){\circle*{0.1}}
\color{red}
\put(88.6666666667,-34.3298858464){\circle*{0.1}}
\color{green}
\put(88.6666666667,-38.2716450711){\circle*{0.1}}
\color{blue}
\put(88.6666666667,34.0023650201){\circle*{0.1}}
\color{red}
\put(89.6666666667,-29.8806296846){\circle*{0.1}}
\color{green}
\put(89.6666666667,-35.3186647401){\circle*{0.1}}
\color{blue}
\put(89.6666666667,29.7100306259){\circle*{0.1}}
\color{red}
\put(90.6666666667,-24.1195032531){\circle*{0.1}}
\color{green}
\put(90.6666666667,-38.1923720983){\circle*{0.1}}
\color{blue}
\put(90.6666666667,24.1161777223){\circle*{0.1}}
\color{red}
\put(91.6666666667,-23.4245376864){\circle*{0.1}}
\color{green}
\put(91.6666666667,-41.294040821){\circle*{0.1}}
\color{blue}
\put(91.6666666667,23.4239585062){\circle*{0.1}}
\color{red}
\put(92.6666666667,-27.5272823869){\circle*{0.1}}
\color{green}
\put(92.6666666667,-36.8767202853){\circle*{0.1}}
\color{blue}
\put(92.6666666667,27.4981781794){\circle*{0.1}}
\color{red}
\put(93.6666666667,-27.722000193){\circle*{0.1}}
\color{green}
\put(93.6666666667,-29.2467392177){\circle*{0.1}}
\color{blue}
\put(93.6666666667,26.8480498564){\circle*{0.1}}
\color{red}
\put(94.6666666667,-27.8906232128){\circle*{0.1}}
\color{green}
\put(94.6666666667,-30.7617145272){\circle*{0.1}}
\color{blue}
\put(94.6666666667,27.3775058923){\circle*{0.1}}
\color{red}
\put(95.6666666667,-20.4346677064){\circle*{0.1}}
\color{green}
\put(95.6666666667,-37.01208453){\circle*{0.1}}
\color{blue}
\put(95.6666666667,20.433617591){\circle*{0.1}}
\color{red}
\put(96.6666666667,-17.3016530341){\circle*{0.1}}
\color{green}
\put(96.6666666667,-37.2875750442){\circle*{0.1}}
\color{blue}
\put(96.6666666667,17.3014343341){\circle*{0.1}}
\color{red}
\put(97.6666666667,-19.4888958933){\circle*{0.1}}
\color{green}
\put(97.6666666667,-35.3465661391){\circle*{0.1}}
\color{blue}
\put(97.6666666667,19.4874330969){\circle*{0.1}}
\color{red}
\put(98.6666666667,-14.6362935697){\circle*{0.1}}
\color{green}
\put(98.6666666667,-28.7245749027){\circle*{0.1}}
\color{blue}
\put(98.6666666667,14.6329838334){\circle*{0.1}}
\color{red}
\put(99.6666666667,-16.0835432443){\circle*{0.1}}
\color{green}
\put(99.6666666667,-34.6566661948){\circle*{0.1}}
\color{blue}
\put(99.6666666667,16.0831238572){\circle*{0.1}}
\color{red}
\put(100.666666667,-19.3407894749){\circle*{0.1}}
\color{green}
\put(100.666666667,-39.7009559906){\circle*{0.1}}
\color{blue}
\put(100.666666667,19.3406054738){\circle*{0.1}}
\color{red}
\put(101.666666667,-22.7206138748){\circle*{0.1}}
\color{green}
\put(101.666666667,-39.1942553734){\circle*{0.1}}
\color{blue}
\put(101.666666667,22.7195124859){\circle*{0.1}}
\color{red}
\put(102.666666667,-24.0016113187){\circle*{0.1}}
\color{green}
\put(102.666666667,-45.656932631){\circle*{0.1}}
\color{blue}
\put(102.666666667,24.001509999){\circle*{0.1}}
\color{red}
\put(103.666666667,-26.2600352712){\circle*{0.1}}
\color{green}
\put(103.666666667,-38.3355012234){\circle*{0.1}}
\color{blue}
\put(103.666666667,26.2517016767){\circle*{0.1}}
\color{red}
\put(104.666666667,-22.0958019357){\circle*{0.1}}
\color{green}
\put(104.666666667,-36.1434531784){\circle*{0.1}}
\color{blue}
\put(104.666666667,22.0924374348){\circle*{0.1}}
\color{red}
\put(105.666666667,-28.0620047776){\circle*{0.1}}
\color{green}
\put(105.666666667,-37.5040173542){\circle*{0.1}}
\color{blue}
\put(105.666666667,28.0341075394){\circle*{0.1}}
\color{red}
\put(106.666666667,-28.8105776961){\circle*{0.1}}
\color{green}
\put(106.666666667,-36.169589064){\circle*{0.1}}
\color{blue}
\put(106.666666667,28.7385122517){\circle*{0.1}}
\color{red}
\put(107.666666667,-27.8656269801){\circle*{0.1}}
\color{green}
\put(107.666666667,-27.9094943596){\circle*{0.1}}
\color{blue}
\put(107.666666667,26.382294175){\circle*{0.1}}
\color{red}
\put(108.666666667,-26.8958402803){\circle*{0.1}}
\color{green}
\put(108.666666667,-30.7962066337){\circle*{0.1}}
\color{blue}
\put(108.666666667,26.5624740211){\circle*{0.1}}
\color{red}
\put(109.666666667,-28.8507954189){\circle*{0.1}}
\color{green}
\put(109.666666667,-31.9126902377){\circle*{0.1}}
\color{blue}
\put(109.666666667,28.3764652726){\circle*{0.1}}
\color{red}
\put(110.666666667,-30.567839764){\circle*{0.1}}
\color{green}
\put(110.666666667,-29.8979341124){\circle*{0.1}}
\color{blue}
\put(110.666666667,28.7020033476){\circle*{0.1}}
\color{red}
\put(111.666666667,-31.3358573323){\circle*{0.1}}
\color{green}
\put(111.666666667,-30.4741813029){\circle*{0.1}}
\color{blue}
\put(111.666666667,29.3574045427){\circle*{0.1}}
\color{red}
\put(112.666666667,-30.6057692311){\circle*{0.1}}
\color{green}
\put(112.666666667,-26.9922881606){\circle*{0.1}}
\color{blue}
\put(112.666666667,26.6157063842){\circle*{0.1}}
\color{red}
\put(113.666666667,-29.9088284215){\circle*{0.1}}
\color{green}
\put(113.666666667,-28.0211462939){\circle*{0.1}}
\color{blue}
\put(113.666666667,27.2608626149){\circle*{0.1}}
\color{red}
\put(114.666666667,-30.4127141312){\circle*{0.1}}
\color{green}
\put(114.666666667,-36.2645254029){\circle*{0.1}}
\color{blue}
\put(114.666666667,30.2707688588){\circle*{0.1}}
\color{red}
\put(115.666666667,-28.3589625508){\circle*{0.1}}
\color{green}
\put(115.666666667,-38.2427094494){\circle*{0.1}}
\color{blue}
\put(115.666666667,28.3361735182){\circle*{0.1}}
\color{red}
\put(116.666666667,-33.3948112691){\circle*{0.1}}
\color{green}
\put(116.666666667,-37.6046349115){\circle*{0.1}}
\color{blue}
\put(116.666666667,33.1028869074){\circle*{0.1}}
\color{red}
\put(117.666666667,-31.3994153345){\circle*{0.1}}
\color{green}
\put(117.666666667,-33.8059586112){\circle*{0.1}}
\color{blue}
\put(117.666666667,30.7799364415){\circle*{0.1}}
\color{red}
\put(118.666666667,-34.2217431024){\circle*{0.1}}
\color{green}
\put(118.666666667,-35.1656170965){\circle*{0.1}}
\color{blue}
\put(118.666666667,33.1376443759){\circle*{0.1}}
\color{red}
\put(119.666666667,-43.47841705){\circle*{0.1}}
\color{green}
\put(119.666666667,-35.9942768769){\circle*{0.1}}
\color{blue}
\put(119.666666667,35.9261843753){\circle*{0.1}}
\color{red}
\put(120.666666667,-45.8556169344){\circle*{0.1}}
\color{green}
\put(120.666666667,-32.4106050233){\circle*{0.1}}
\color{blue}
\put(120.666666667,32.4061657869){\circle*{0.1}}
\color{red}
\put(121.666666667,-47.0789246603){\circle*{0.1}}
\color{green}
\put(121.666666667,-33.043416369){\circle*{0.1}}
\color{blue}
\put(121.666666667,33.0400332715){\circle*{0.1}}
\color{red}
\put(122.666666667,-43.6933760845){\circle*{0.1}}
\color{green}
\put(122.666666667,-32.1067182169){\circle*{0.1}}
\color{blue}
\put(122.666666667,32.0962859324){\circle*{0.1}}
\color{red}
\put(123.666666667,-43.19766211){\circle*{0.1}}
\color{green}
\put(123.666666667,-33.412609589){\circle*{0.1}}
\color{blue}
\put(123.666666667,33.3887667732){\circle*{0.1}}
\color{red}
\put(124.666666667,-41.8429336963){\circle*{0.1}}
\color{green}
\put(124.666666667,-32.7558763953){\circle*{0.1}}
\color{blue}
\put(124.666666667,32.7230625154){\circle*{0.1}}
\color{red}
\put(125.666666667,-40.6681820373){\circle*{0.1}}
\color{green}
\put(125.666666667,-23.5051525394){\circle*{0.1}}
\color{blue}
\put(125.666666667,23.5043507035){\circle*{0.1}}
\color{red}
\put(126.666666667,-39.7697337481){\circle*{0.1}}
\color{green}
\put(126.666666667,-22.4015764256){\circle*{0.1}}
\color{blue}
\put(126.666666667,22.400846836){\circle*{0.1}}
\color{red}
\put(127.666666667,-35.3139816917){\circle*{0.1}}
\color{green}
\put(127.666666667,-27.4468902319){\circle*{0.1}}
\color{blue}
\put(127.666666667,27.389662895){\circle*{0.1}}
\color{red}
\put(128.666666667,-41.4909591547){\circle*{0.1}}
\color{green}
\put(128.666666667,-29.3215864881){\circle*{0.1}}
\color{blue}
\put(128.666666667,29.3136050238){\circle*{0.1}}
\color{red}
\put(129.666666667,-42.4062562679){\circle*{0.1}}
\color{green}
\put(129.666666667,-25.3520710891){\circle*{0.1}}
\color{blue}
\put(129.666666667,25.3512280643){\circle*{0.1}}
\color{red}
\put(130.666666667,-39.494477972){\circle*{0.1}}
\color{green}
\put(130.666666667,-20.5752110364){\circle*{0.1}}
\color{blue}
\put(130.666666667,20.5748538197){\circle*{0.1}}
\color{red}
\put(131.666666667,-36.9048805134){\circle*{0.1}}
\color{green}
\put(131.666666667,-20.5245998318){\circle*{0.1}}
\color{blue}
\put(131.666666667,20.5234499548){\circle*{0.1}}
\color{red}
\put(132.666666667,-37.4953925228){\circle*{0.1}}
\color{green}
\put(132.666666667,-20.067580317){\circle*{0.1}}
\color{blue}
\put(132.666666667,20.0668704041){\circle*{0.1}}
\color{red}
\put(133.666666667,-42.1082156578){\circle*{0.1}}
\color{green}
\put(133.666666667,-21.5310219312){\circle*{0.1}}
\color{blue}
\put(133.666666667,21.5308554589){\circle*{0.1}}
\color{red}
\put(134.666666667,-46.4888755847){\circle*{0.1}}
\color{green}
\put(134.666666667,-23.4604352822){\circle*{0.1}}
\color{blue}
\put(134.666666667,23.4603814455){\circle*{0.1}}
\color{red}
\put(135.666666667,-37.7423255893){\circle*{0.1}}
\color{green}
\put(135.666666667,-22.1202794931){\circle*{0.1}}
\color{blue}
\put(135.666666667,22.1186493911){\circle*{0.1}}
\color{red}
\put(136.666666667,-41.6691194309){\circle*{0.1}}
\color{green}
\put(136.666666667,-19.9087524201){\circle*{0.1}}
\color{blue}
\put(136.666666667,19.908655868){\circle*{0.1}}
\color{red}
\put(137.666666667,-42.57817877){\circle*{0.1}}
\color{green}
\put(137.666666667,-22.5258587885){\circle*{0.1}}
\color{blue}
\put(137.666666667,22.5256468078){\circle*{0.1}}
\color{red}
\put(138.666666667,-45.1657094107){\circle*{0.1}}
\color{green}
\put(138.666666667,-21.5608755137){\circle*{0.1}}
\color{blue}
\put(138.666666667,21.5608342255){\circle*{0.1}}
\color{red}
\put(139.666666667,-39.1994510773){\circle*{0.1}}
\color{green}
\put(139.666666667,-25.6633121642){\circle*{0.1}}
\color{blue}
\put(139.666666667,25.6590551298){\circle*{0.1}}
\color{red}
\put(140.666666667,-46.0001960068){\circle*{0.1}}
\color{green}
\put(140.666666667,-28.8489224304){\circle*{0.1}}
\color{blue}
\put(140.666666667,28.8481162719){\circle*{0.1}}
\color{red}
\put(141.666666667,-48.0183242981){\circle*{0.1}}
\color{green}
\put(141.666666667,-29.5860543091){\circle*{0.1}}
\color{blue}
\put(141.666666667,29.5856073615){\circle*{0.1}}
\color{red}
\put(142.666666667,-52.993275426){\circle*{0.1}}
\color{green}
\put(142.666666667,-26.8986188749){\circle*{0.1}}
\color{blue}
\put(142.666666667,26.8986057582){\circle*{0.1}}
\color{red}
\put(143.666666667,-53.4050372453){\circle*{0.1}}
\color{green}
\put(143.666666667,-26.1599781227){\circle*{0.1}}
\color{blue}
\put(143.666666667,26.1599704004){\circle*{0.1}}
\color{red}
\put(144.666666667,-53.3384091905){\circle*{0.1}}
\color{green}
\put(144.666666667,-23.6181575441){\circle*{0.1}}
\color{blue}
\put(144.666666667,23.6181550739){\circle*{0.1}}
\color{red}
\put(145.666666667,-54.4568254811){\circle*{0.1}}
\color{green}
\put(145.666666667,-20.9681417014){\circle*{0.1}}
\color{blue}
\put(145.666666667,20.9681412658){\circle*{0.1}}
\color{red}
\put(146.666666667,-46.4245279017){\circle*{0.1}}
\color{green}
\put(146.666666667,-19.1303049831){\circle*{0.1}}
\color{blue}
\put(146.666666667,19.1302974315){\circle*{0.1}}
\color{red}
\put(147.666666667,-43.3741708521){\circle*{0.1}}
\color{green}
\put(147.666666667,-23.5607206279){\circle*{0.1}}
\color{blue}
\put(147.666666667,23.5604840046){\circle*{0.1}}
\color{red}
\put(148.666666667,-45.5316846166){\circle*{0.1}}
\color{green}
\put(148.666666667,-23.1979672892){\circle*{0.1}}
\color{blue}
\put(148.666666667,23.1978931542){\circle*{0.1}}
\color{red}
\put(149.666666667,-40.9939622288){\circle*{0.1}}
\color{green}
\put(149.666666667,-24.4249273049){\circle*{0.1}}
\color{blue}
\put(149.666666667,24.4238732743){\circle*{0.1}}
\color{red}
\put(150.666666667,-41.6005123244){\circle*{0.1}}
\color{green}
\put(150.666666667,-22.7483053734){\circle*{0.1}}
\color{blue}
\put(150.666666667,22.7479369896){\circle*{0.1}}
\color{red}
\put(151.666666667,-40.2364358958){\circle*{0.1}}
\color{green}
\put(151.666666667,-17.4522825689){\circle*{0.1}}
\color{blue}
\put(151.666666667,17.4522222854){\circle*{0.1}}
\color{red}
\put(152.666666667,-42.5583079246){\circle*{0.1}}
\color{green}
\put(152.666666667,-19.3753621859){\circle*{0.1}}
\color{blue}
\put(152.666666667,19.3753120353){\circle*{0.1}}
\color{red}
\put(153.666666667,-45.2036746634){\circle*{0.1}}
\color{green}
\put(153.666666667,-25.7046900742){\circle*{0.1}}
\color{blue}
\put(153.666666667,25.7044165874){\circle*{0.1}}
\color{red}
\put(154.666666667,-46.188069163){\circle*{0.1}}
\color{green}
\put(154.666666667,-29.8489098666){\circle*{0.1}}
\color{blue}
\put(154.666666667,29.8477381942){\circle*{0.1}}
\color{red}
\put(155.666666667,-52.484961661){\circle*{0.1}}
\color{green}
\put(155.666666667,-31.2855843192){\circle*{0.1}}
\color{blue}
\put(155.666666667,31.285459332){\circle*{0.1}}
\color{red}
\put(156.666666667,-54.6644349399){\circle*{0.1}}
\color{green}
\put(156.666666667,-36.916028841){\circle*{0.1}}
\color{blue}
\put(156.666666667,36.9154164761){\circle*{0.1}}
\color{red}
\put(157.666666667,-58.1484838035){\circle*{0.1}}
\color{green}
\put(157.666666667,-36.8876431765){\circle*{0.1}}
\color{blue}
\put(157.666666667,36.8875216754){\circle*{0.1}}
\color{red}
\put(158.666666667,-56.2434402194){\circle*{0.1}}
\color{green}
\put(158.666666667,-44.1017803904){\circle*{0.1}}
\color{blue}
\put(158.666666667,44.0936964818){\circle*{0.1}}
\color{red}
\put(159.666666667,-54.5675188576){\circle*{0.1}}
\color{green}
\put(159.666666667,-46.5992015899){\circle*{0.1}}
\color{blue}
\put(159.666666667,46.5445487581){\circle*{0.1}}
\color{red}
\put(160.666666667,-61.938326906){\circle*{0.1}}
\color{green}
\put(160.666666667,-45.1757276169){\circle*{0.1}}
\color{blue}
\put(160.666666667,45.174763696){\circle*{0.1}}
\color{red}
\put(161.666666667,-67.9299507599){\circle*{0.1}}
\color{green}
\put(161.666666667,-48.2214491565){\circle*{0.1}}
\color{blue}
\put(161.666666667,48.2212010335){\circle*{0.1}}
\color{red}
\put(162.666666667,-65.9019109137){\circle*{0.1}}
\color{green}
\put(162.666666667,-43.5415754732){\circle*{0.1}}
\color{blue}
\put(162.666666667,43.541502293){\circle*{0.1}}
\color{red}
\put(163.666666667,-57.7827160901){\circle*{0.1}}
\color{green}
\put(163.666666667,-49.3995429711){\circle*{0.1}}
\color{blue}
\put(163.666666667,49.3542965649){\circle*{0.1}}
\color{red}
\put(164.666666667,-60.9322007486){\circle*{0.1}}
\color{green}
\put(164.666666667,-48.4227086542){\circle*{0.1}}
\color{blue}
\put(164.666666667,48.4158831551){\circle*{0.1}}
\color{red}
\put(165.666666667,-63.3849446913){\circle*{0.1}}
\color{green}
\put(165.666666667,-53.5966879025){\circle*{0.1}}
\color{blue}
\put(165.666666667,53.5728874098){\circle*{0.1}}
\color{red}
\put(166.666666667,-56.7865027014){\circle*{0.1}}
\color{green}
\put(166.666666667,-53.5735188878){\circle*{0.1}}
\color{blue}
\put(166.666666667,53.128025115){\circle*{0.1}}
\color{red}
\put(167.666666667,-57.3223770842){\circle*{0.1}}
\color{green}
\put(167.666666667,-54.267861659){\circle*{0.1}}
\color{blue}
\put(167.666666667,53.7920679257){\circle*{0.1}}
\color{red}
\put(168.666666667,-56.4946692557){\circle*{0.1}}
\color{green}
\put(168.666666667,-50.7376892767){\circle*{0.1}}
\color{blue}
\put(168.666666667,50.5896196224){\circle*{0.1}}
\color{red}
\put(169.666666667,-55.9838117416){\circle*{0.1}}
\color{green}
\put(169.666666667,-49.1116469096){\circle*{0.1}}
\color{blue}
\put(169.666666667,49.0218415879){\circle*{0.1}}
\color{red}
\put(170.666666667,-50.4155168225){\circle*{0.1}}
\color{green}
\put(170.666666667,-52.3810543299){\circle*{0.1}}
\color{blue}
\put(170.666666667,49.6779451493){\circle*{0.1}}
\color{red}
\put(171.666666667,-51.2525293438){\circle*{0.1}}
\color{green}
\put(171.666666667,-50.4670617928){\circle*{0.1}}
\color{blue}
\put(171.666666667,49.3193225555){\circle*{0.1}}
\color{red}
\put(172.666666667,-44.9521724038){\circle*{0.1}}
\color{green}
\put(172.666666667,-52.0599009048){\circle*{0.1}}
\color{blue}
\put(172.666666667,44.8714286245){\circle*{0.1}}
\color{red}
\put(173.666666667,-45.3969516504){\circle*{0.1}}
\color{green}
\put(173.666666667,-54.2865480059){\circle*{0.1}}
\color{blue}
\put(173.666666667,45.3610397604){\circle*{0.1}}
\color{red}
\put(174.666666667,-46.3720412216){\circle*{0.1}}
\color{green}
\put(174.666666667,-47.5141029939){\circle*{0.1}}
\color{blue}
\put(174.666666667,45.3636898496){\circle*{0.1}}
\color{red}
\put(175.666666667,-42.3544587719){\circle*{0.1}}
\color{green}
\put(175.666666667,-45.6646967927){\circle*{0.1}}
\color{blue}
\put(175.666666667,41.9266732817){\circle*{0.1}}
\color{red}
\put(176.666666667,-45.4561152076){\circle*{0.1}}
\color{green}
\put(176.666666667,-48.766353224){\circle*{0.1}}
\color{blue}
\put(176.666666667,45.0283296412){\circle*{0.1}}
\color{red}
\put(177.666666667,-45.1471255009){\circle*{0.1}}
\color{green}
\put(177.666666667,-45.7066495678){\circle*{0.1}}
\color{blue}
\put(177.666666667,43.903765502){\circle*{0.1}}
\color{red}
\put(178.666666667,-43.2756559954){\circle*{0.1}}
\color{green}
\put(178.666666667,-39.1143094601){\circle*{0.1}}
\color{blue}
\put(178.666666667,38.8162273825){\circle*{0.1}}
\color{red}
\put(179.666666667,-43.0437270306){\circle*{0.1}}
\color{green}
\put(179.666666667,-40.9084023289){\circle*{0.1}}
\color{blue}
\put(179.666666667,40.2183822979){\circle*{0.1}}
\color{red}
\put(180.666666667,-42.1290849938){\circle*{0.1}}
\color{green}
\put(180.666666667,-46.4488084968){\circle*{0.1}}
\color{blue}
\put(180.666666667,41.8506832098){\circle*{0.1}}
\color{red}
\put(181.666666667,-43.7274455221){\circle*{0.1}}
\color{green}
\put(181.666666667,-47.2615519213){\circle*{0.1}}
\color{blue}
\put(181.666666667,43.3380329113){\circle*{0.1}}
\color{red}
\put(182.666666667,-43.7677399515){\circle*{0.1}}
\color{green}
\put(182.666666667,-49.1755444582){\circle*{0.1}}
\color{blue}
\put(182.666666667,43.5948423203){\circle*{0.1}}
\color{red}
\put(183.666666667,-45.12675577){\circle*{0.1}}
\color{green}
\put(183.666666667,-46.2296846116){\circle*{0.1}}
\color{blue}
\put(183.666666667,44.1037855461){\circle*{0.1}}
\color{red}
\put(184.666666667,-43.475158913){\circle*{0.1}}
\color{green}
\put(184.666666667,-44.2881280202){\circle*{0.1}}
\color{blue}
\put(184.666666667,42.3386681036){\circle*{0.1}}
\color{red}
\put(185.666666667,-35.3812972383){\circle*{0.1}}
\color{green}
\put(185.666666667,-42.1426925661){\circle*{0.1}}
\color{blue}
\put(185.666666667,35.286891041){\circle*{0.1}}
\color{red}
\put(186.666666667,-33.9287786097){\circle*{0.1}}
\color{green}
\put(186.666666667,-41.914503073){\circle*{0.1}}
\color{blue}
\put(186.666666667,33.8745565423){\circle*{0.1}}
\color{red}
\put(187.666666667,-31.6963475648){\circle*{0.1}}
\color{green}
\put(187.666666667,-38.0361697956){\circle*{0.1}}
\color{blue}
\put(187.666666667,31.5822361337){\circle*{0.1}}
\color{red}
\put(188.666666667,-33.2930487397){\circle*{0.1}}
\color{green}
\put(188.666666667,-40.6172783589){\circle*{0.1}}
\color{blue}
\put(188.666666667,33.2198393021){\circle*{0.1}}
\color{red}
\put(189.666666667,-30.9908476618){\circle*{0.1}}
\color{green}
\put(189.666666667,-35.7300597522){\circle*{0.1}}
\color{blue}
\put(189.666666667,30.7588385798){\circle*{0.1}}
\color{red}
\put(190.666666667,-29.134413716){\circle*{0.1}}
\color{green}
\put(190.666666667,-37.5212218869){\circle*{0.1}}
\color{blue}
\put(190.666666667,29.0892417545){\circle*{0.1}}
\color{red}
\put(191.666666667,-25.1424442139){\circle*{0.1}}
\color{green}
\put(191.666666667,-36.1242908911){\circle*{0.1}}
\color{blue}
\put(191.666666667,25.1286716394){\circle*{0.1}}
\color{red}
\put(192.666666667,-30.4375023103){\circle*{0.1}}
\color{green}
\put(192.666666667,-41.43568823){\circle*{0.1}}
\color{blue}
\put(192.666666667,30.4238328873){\circle*{0.1}}
\color{red}
\put(193.666666667,-34.1633702366){\circle*{0.1}}
\color{green}
\put(193.666666667,-37.9329548111){\circle*{0.1}}
\color{blue}
\put(193.666666667,33.8109064506){\circle*{0.1}}
\color{red}
\put(194.666666667,-29.5216997733){\circle*{0.1}}
\color{green}
\put(194.666666667,-35.4481998289){\circle*{0.1}}
\color{blue}
\put(194.666666667,29.3844051321){\circle*{0.1}}
\color{red}
\put(195.666666667,-29.1319180442){\circle*{0.1}}
\color{green}
\put(195.666666667,-31.8935760964){\circle*{0.1}}
\color{blue}
\put(195.666666667,28.595308317){\circle*{0.1}}
\color{red}
\put(196.666666667,-27.7178739293){\circle*{0.1}}
\color{green}
\put(196.666666667,-32.9093167731){\circle*{0.1}}
\color{blue}
\put(196.666666667,27.5276324556){\circle*{0.1}}
\color{red}
\put(197.666666667,-25.874764568){\circle*{0.1}}
\color{green}
\put(197.666666667,-28.1145225988){\circle*{0.1}}
\color{blue}
\put(197.666666667,25.2126725873){\circle*{0.1}}
\color{red}
\put(198.666666667,-21.9884115188){\circle*{0.1}}
\color{green}
\put(198.666666667,-22.9712451437){\circle*{0.1}}
\color{blue}
\put(198.666666667,20.9194737896){\circle*{0.1}}
\color{red}
\put(199.666666667,-22.7976476588){\circle*{0.1}}
\color{green}
\put(199.666666667,-21.7586435702){\circle*{0.1}}
\color{blue}
\put(199.666666667,20.7113628949){\circle*{0.1}}
\color{red}
\put(200.666666667,-23.707538764){\circle*{0.1}}
\color{green}
\put(200.666666667,-15.3921640393){\circle*{0.1}}
\color{blue}
\put(200.666666667,15.3454205196){\circle*{0.1}}
\color{red}
\put(201.666666667,-25.1521330497){\circle*{0.1}}
\color{green}
\put(201.666666667,-12.0597655322){\circle*{0.1}}
\color{blue}
\put(201.666666667,12.0545040014){\circle*{0.1}}
\color{red}
\put(202.666666667,-24.6603181115){\circle*{0.1}}
\color{green}
\put(202.666666667,-14.1657329787){\circle*{0.1}}
\color{blue}
\put(202.666666667,14.1484593605){\circle*{0.1}}
\color{red}
\put(203.666666667,-17.9593634368){\circle*{0.1}}
\color{green}
\put(203.666666667,-6.50001893381){\circle*{0.1}}
\color{blue}
\put(203.666666667,6.48782489834){\circle*{0.1}}
\color{red}
\put(204.666666667,-16.3950860477){\circle*{0.1}}
\color{green}
\put(204.666666667,-9.38382839779){\circle*{0.1}}
\color{blue}
\put(204.666666667,9.29727548834){\circle*{0.1}}
\color{red}
\put(205.666666667,-17.6249975046){\circle*{0.1}}
\color{green}
\put(205.666666667,-14.0926706){\circle*{0.1}}
\color{blue}
\put(205.666666667,13.7018815206){\circle*{0.1}}
\color{red}
\put(206.666666667,-17.1388127832){\circle*{0.1}}
\color{green}
\put(206.666666667,-13.2550861097){\circle*{0.1}}
\color{blue}
\put(206.666666667,12.917955626){\circle*{0.1}}
\color{red}
\put(207.666666667,-18.1253079603){\circle*{0.1}}
\color{green}
\put(207.666666667,-18.8370537584){\circle*{0.1}}
\color{blue}
\put(207.666666667,16.9465680045){\circle*{0.1}}
\color{red}
\put(208.666666667,-18.4000466953){\circle*{0.1}}
\color{green}
\put(208.666666667,-27.7124716819){\circle*{0.1}}
\color{blue}
\put(208.666666667,18.3704334449){\circle*{0.1}}
\end{picture}

\end{center}
\end{enumerate}

\aufgabe{RNA Sekundärstruktur}{40}

Handschriftlich auf der Rückseite.

%\begin{enumerate}
%\item
%\item
%\item
%\end{enumerate}

\end{enumerate}
\end{document}
