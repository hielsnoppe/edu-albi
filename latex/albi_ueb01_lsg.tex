\documentclass{homework}
\usepackage{marvosym}
\course{Algorithmische Bioinformatik}
\semester{Wintersemester 2012 / 2013}
\no{1}
\date{Montag, dem 22. Oktober 2012}
\author{Stefan Meißner (4279113) und Niels Hoppe (4356370)}
\tutorial{Dienstag 08:00 - 10:00 und 12:00 - 14:00}
\tutor{Alena van Bömmel und ?}

\begin{document}
\maketitle
\begin{enumerate} 

\aufgabe{Sequenzstatistik}{20}

Für Sequenzen der Länge sieben bestehend aus Zeichen des vierelementigen Alphabetes $\{A,C,G,T\}$ gibt es $4^7$ mögliche Permutationen.
Davon sind alle Sequenzen der Form \texttt{**CTGAC}, \texttt{*CTGAC*} oder \texttt{CTGAC**} Treffer.
Da diese drei Muster disjunkt sind und jeweils zwei Wildcards enthalten,
errechnet sich dafür eine Trefferzahl von $3 \cdot 4^2 = 48$.
Insgesamt berechnet sich also eine Trefferwahrscheinlichkeit von $\frac{48}{4^7} = \frac{3}{1024} \approx 2,929 \cdot 10^{-3}$.
Bei $10.000$ Sequenzen sind also $29,296875$ exakte Treffer zu erwarten.

\textbf{Bonus (20 Punkte):} Wählen wir als Beispiel die Sequenz \texttt{ACACA},
so erhalten wir als Muster für Treffer \texttt{**ACACA}, \texttt{*ACACA*} und \texttt{ACACA**}.
Dabei zeigt sich, dass das erste und das letzte Muster nicht disjunkt sind.
Wir müssen dies bei der Berechnung der Trefferzahl mit dem Inklusions-Exklusions-Prinzip berücksichtigen.
Es gibt dafür also $3 \cdot 4^2 - 1 = 47$ Treffer, da die Sequenz \texttt{ACACACA} sonst doppelt gezählt würde.
Entsprechend verändert sich der Erwartungswert zu $28,68652344$ Treffern auf $10.000$ Sequenzen.

\aufgabe{Reguläre Ausdrücke in Python}{40}

\lstinputlisting[language=python]{albi_ueb01_a2.py}

Beispielaufruf:

\begin{verbatim}
$ python3 albi_ueb01_a2.py -i orf_trans.fasta -o output.file
\end{verbatim}

\aufgabe{Bestimmung einer Proteinfamilie}{50}

makeblastdb Aufruf:
\begin{verbatim}
$ makeblastdb -in output.file -input_type fasta -dbtype prot
-out result.db -title "hefe"
\end{verbatim}

FAL1.fsa:
\begin{verbatim}
>YDR021W 
MSFDREEDQKLKFKTSKKLKVSSTFESMNLKDDLLRGIYSYGFEAPSSIQSRAITQIISG 
KDVIAQAQSGTGKTATFTIGLLQAIDLRKKDLQALILSPTRELASQIGQVVKNLGDYMNV 
NAFAITGGKTLKDDLKKMQKHGCQAVSGTPGRVLDMIKKQMLQTRNVQMLVLDEADELLS 
ETLGFKQQIYDIFAKLPKNCQVVVVSATMNKDILEVTRKFMNDPVKILVKRDEISLEGIK 
QYVVNVDKEEWKFDTLCDIYDSLTITQCVIFCNTKKKVDWLSQRLIQSNFAVVSMHGDMK 
QEERDKVMNDFRTGHSRVLISTDVWARGIDVQQVSLVINYDLPEIIENYIHRIGRSGRFG 
RKGVAINFITKADLAKLREIEKFYSIKINPMPANFAELS*
\end{verbatim}

blastp Query:
\begin{verbatim}
$ blastp -db result.db -query FAL1.fsa -out "results.out"
\end{verbatim}

$results.out$ enthält nun die Scores zu allen Homologen von FAL1.
Werden alle betrachtet (auch welche mit geringen Score und FAL1 selbst) gibt es 
\begin{verbatim}
$ wc -l homologen.fal1 
53 homologen.fal1
\end{verbatim}
in der DB.\\
Die Namen der Homologen müssen nun für den Input des Python Scripts zeilenweise in eine Datei geschrieben werden.
Die $protein\_properties.tab$ kann unverändert genutzt werden.
Um das Ergebnis in einer Ausgabedatei zu erhalten, einfach den StdOut pipen, beispiel:

\begin{verbatim}
$ python3 albi_ueb01_a2.py 
		-i homologen.fal1 
		-p protein_properties.tab 
		> result_aufg3.txt
\end{verbatim}

\lstinputlisting[language=python]{albi_ueb01_a3.py}

\aufgabe{Dynamische Programmierung}{50}
\begin{enumerate}
\item[1.] Dynamische Programmierung berechnet eine Gesamtlösung für ein Problem,
indem sie es rekursiv in einfachere Teilprobleme teilt,
deren jeweilige Lösung sich aus anderen Teilergebnissen leicht berechnen lässt.
Um die wiederholte Berechnung von Teilergebnissen zu vermeiden, werden alle Teilergebnisse z.B. in einer Matrix gespeichert.
Die Gesamtlösung ist je nach Problemstellung häufig ein Maximum oder Minimum in der Ergebnismatrix.
Der optimale Lösungsweg wird anschließend durch Rückverfolgung (backtrace) der Teilergebnisse,
die zu der optimalen Lösung geführt haben, ermittelt.

\item[2.] Die Matrix wurde durch den Smith-Waterman-Algorithmus erzeugt.
\item[3.] Die verwendete gap penalty beträgt $d = 1$, die Scoring Funktion ist
$$s(x_i, y_j) = \begin{cases}
2	& \text{ if } x_i = y_j\\
-1	& \text{ else}
\end{cases}.$$

\item[4.] Die restlichen Felder der Matrix ergeben sich durch den Algorithmus wie folgt:

\begin{center}
\begin{tabular}{c|c@{}c@{}c@{}c@{}c@{}c@{}c@{}c@{}c@{}c@{}c@{}c@{}c@{}c@{}c@{}c@{}c@{}c@{}c@{}c@{}c@{}c@{}c}
& && A && A && G && C && C && T && T && G && C && A && A\\\hline
\ldots & & & & & & $\quad$ & & & & & & & & & & & & & & & &\\
& $\uparrow$ & $\nwarrow$ & & & $\uparrow$ & & $\uparrow$ & $\nwarrow$ & & $\nwarrow$ & & & & & & & & $\nwarrow$ & & & & & $\uparrow$\\
C & 0 && 0 && 1 && 4 && 7 && 7 & $\leftarrow$ & 6 & $\leftarrow$ & 5 & $\leftarrow$ & 4 && 6 & $\leftarrow$ & 5 && 6\\
& $\uparrow$ & $\nwarrow$ & & $\nwarrow$ & & & $\uparrow$ & $\nwarrow$ & & $\nwarrow$ & & & & & & & & $\nwarrow$ & & $\nwarrow$ & & & $\uparrow$\\
C & 0 && 0 && 0 && 3 && 6 && \textbf{9} & $\leftarrow$ & 8 & $\leftarrow$ & 7 & $\leftarrow$ & 6 && 6 && 5 && 5\\
& $\uparrow$ & $\nwarrow$ & & $\nwarrow$ & & & $\uparrow$ & & $\uparrow$ & & $\uparrow$ & $\nwarrow$ & & $\nwarrow$ & & $\nwarrow$ & & $\nwarrow$ & & $\nwarrow$ & & $\nwarrow$ &\\
A & 0 && 2 && 2 && 2 && 5 && 8 && 8 && 7 && 6 && 5 && 8 && 7
\end{tabular}
\end{center}

\item[5.] Die optimalen lokalen Alignments ergeben sich durch Backtracing der Felder mit Wert 9.
Es sind dies:

\begin{verbatim}
AAGCCTTGCAA               AA-GCCTTGCAA
  GAC--GCAAGGCCA     GACGCAAGGCCA
\end{verbatim}

\end{enumerate}


\end{enumerate}
\end{document}
