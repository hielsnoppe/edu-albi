\documentclass{homework}
\usepackage{marvosym}
\usepackage{hyperref}

\course{Algorithmische Bioinformatik}
\semester{Wintersemester 2012 / 2013}
\no{4}
\date{Montag, dem 12. November 2012}
\author{Stefan Meißner (4279113) und Niels Hoppe (4356370)}
\tutorial{Dienstag 08:00 - 10:00}
\tutor{Alena van Bömmel (Übungsgruppe 3)}

\begin{document}
\maketitle
\begin{enumerate} 

\aufgabe{Multiples Alignment}{100}
\begin{enumerate}
\item Bei dem Datensatz handelt es sich um das Protein \textit{ubiquitin}.
\item Der Aufruf \textit{t\_coffee BBS20030.tfa} erzeugt u.a. die Datei \textit{BBS20030.aln}. Hier der Auszug der ersten 5 Zeilen:
\begin{verbatim}
1awd_       YKVTLK--TP-S-G--EETIECPEDTYILD----AAEEAGLDLPYSCRAG
FER1_ANASP  FKVTLI--NEAE-G-TKHEIEVPDDEYILD----AAEEQGYDLPFSCRAG
FER_SPIMA   YKVTLI--SEAE-G-INETIDCDDDTYILD----AAEEAGLDLPYSCRAG
FER1_CYAPA  YKVRLI--CEEQ-G-LDTTIECPDDEYILD----AAEEQGIDLPYSCRAG
FER3_CYACA  YKIHLV--NKDQ-G-IDETIECPDDQYILD----AAEEQGLDLPYSCRAG
\end{verbatim}
\item Quelle: \url{http://en.wikipedia.org/wiki/Multiple\_sequence\_alignment}

Das progressive Alignment baut eine MSA durch Kombinieren der paarweisen Alignments, beginnend mit dem ähnlichsten Paar und fortlaufend mit den am weitesten entfernten Paaren, auf. Alle progressiven Alignment-Verfahren erfordern zwei Stufen: eine erste Stufe, in welcher die Beziehungen zwischen den Sequenzen als Baum, genannt \textit{Guide Tree}, repräsentiert wird und einem zweiten Schritt, in dem das MSA durch sequentielles hinzufügen der Sequenzen anhand des Guide Tree aufgebaut wird. Der initiale Guide Tree wird durch eine effiziente Clusteringmethode wie Neighbor-Joining oder UPGMA bestimmt. \\
Progressive Alignments können kein globales Optimum garantieren, da Fehler beim Hinzufügen von Sequenzen zum MSA propagieren. Die Laufzeit ist exponential.

\item Die iterativen Alignment Algorithmen basieren auf den progressiven, jedoch wird hier beim Hinzufügen von Sequenzen zum MSA die bereits alinierten Sequenzen erneut aliniert. Dadurch können propagierende Fehler vermieden werden und  ein global optimales Alignment garantiert werden.

\item
\begin{verbatim}
$ t_coffee -other_pg aln_compare 
-al1 BBS20030.aln -al2 muscle-clustalw_strict.aln 
*****************************************************
seq1       seq2          Sim   [ALL]           Tot  
BBS20030      47         57.5    94.5 [100.0]   [184912]

$ t_coffee -other_pg aln_compare 
-al2 BBS20030.aln -al1 muscle-clustalw_strict.aln 
*****************************************************
seq1       seq2          Sim   [ALL]           Tot  
muscle-clustalw_strict 47         57.5    93.6 [100.0]   [186804]
\end{verbatim}
D.h. es gibt eine Übereinstimmung von 94,5\% der Alignments, wenn der progressive Ansatz mit dem iterativen Ansatz verglichen wird bzw. 93,6\% Übereinstimmung andersrum. 

\item
Erstaunlicherweise, hat das Alignment von T-Coffee eine höhere Übereinstimmung (94\%) als das von Muscle (92,6\%). Eventuell gibt es zu wenig Daten für den Vergleich?

\item
Der Score veringert sich sogar auf 93,6\%.
\end{enumerate}



\aufgabe{Bayes'sche Regel}{40+10}
Bezeichne $T^+$ bzw. $T^-$ ein positives bzw. negatives Testergebnis und $H^+$
bzw. $H^-$ eine tatsächlich bestehende bzw. nicht bestehende Erkrankung an HIV,
so ist gegeben:

$$P(T^+ | H^+) = 99,9\% = 0,999$$
$$P(T^- | H^-) = 99,5\% = 0,995$$

\begin{enumerate}
\item Aus der Risikogruppe ergibt sich:
$$P(H^+) = 0,1\% = 0,001$$

Gesucht ist:
$$P(H^+ | T^+) = \frac{P(T^+ | H^+) \cdot P(H^+)}{P(T^+)}$$

Die Wahrscheinlichkeit, dass ein Test positiv ist, ergibt sich aus:

\begin{eqnarray*}
P(T^+)
& = & P(T^+ \cap H^+) + P(T^+ \cap H^-) \\ 
& = & P(T^+ | H^+) \cdot P(H^+) + P(T^+ | H^-) \cdot P(H^-) \\ 
& = & P(T^+ | H^+) \cdot P(H^+) + (1-P(T^- | H^-)) \cdot (1-P(H^+))\\
& = & 0,999 \cdot 0,001 + (1-0,995)\cdot(1-0,001)\\
& = & 0,005994\\
\end{eqnarray*}

Daher:
$$P(H^+ | T^+) = \frac{0,999\cdot0,001}{0,005994} = 0,167 = 16,7\%$$

\item Aus der Risikogruppe ergibt sich:

$$P(H^+) = 20\% = 0,2$$

Daher:
$$P(T^+) = 0,2038$$
$$P(H^+ | T^+) = \frac{0,999\cdot0,2}{0,2038} = 0,98 = 98\%$$

\item
	$$P(H^+ | 2\cdot T^+) = \frac{P(2\cdot T^+| H^+) \cdot P(H^+)}{P(2\cdot T^+)} $$

\begin{eqnarray*}
P(2\cdot T^+)
& = & P(2\cdot T^+ \cap H^+) + P(2\cdot T^+ \cap H^-) \\ 
& = & P(2\cdot T^+ | H^+) \cdot P(H^+) + P(2\cdot T^+ | H^-) \cdot P(H^-) \\ 
& = & P(2\cdot T^+ | H^+) \cdot P(H^+) + (1-P(2\cdot T^- | H^-)) \cdot (1-P(H^+))\\
& = & 0,999^2 \cdot 0,001 + (1-0,995)^2\cdot(1-0,001)\\
& = & 0,001023
\end{eqnarray*}

Ergibt: \\
$$P(H^+ | 2\cdot T^+) = \frac{0,999^2\cdot0,001}{0,001023} = 0,9756 = 97,56\%$$
\end{enumerate}


\aufgabe{HMM}{40}
\begin{enumerate}
\item $$\Sigma = \{A, C, G, T\} \quad S = \{Begin, 1, 2, 3, 4, End\}$$
%B	& 1		& 2		& 3		& 4		& E\\
$$A = \begin{pmatrix}
0	& 0,5	& 0		& 0,5	& 0		& 0\\
0	& 0		& 1		& 0		& 0		& 0\\
0	& 0		& 0,5	& 0,3	& 0		& 0,2\\
0	& 0		& 0		& 0		& 1		& 0\\
0	& 0,3	& 0		& 0		& 0,5	& 0,2\\
0	& 0		& 0		& 0		& 0		& 0
\end{pmatrix}$$

\item Der Viterbi-Pfad mit einer maximalen Gesamtwahrscheinlichkeit von 0,25\%
ist "`Begin, 3, 4, 4, End"'.

\item 

\end{enumerate}

\aufgabe{Viterbi in the occasionally dishonest casino}{60}

Der erzeugende Quelltext wurde per E-Mail eingesandt.
Es fällt auf, dass der Viterbi-Pfad in (a) und (c) identisch sind. Der Pfad aus
(b) unterscheidet sich jedoch von den übrigen. Außerdem sind in allen Pfaden
Unterschiede zur Vorhersage aus der Datei \textit{casino.txt} festzustellen.

\begin{enumerate}
\item \begin{verbatim}
FFFFFFFFFFFFFFFFFFFFFFFFFFFFFFFFFFFFFFFFFFFFFLLLLLLLLLLLLLLLLLL
LLLLLLLLLLLLLLLLLLLLLLLLLLLLLLLLLLLLLLLLLLLLLLLLLFFFFFFFFFFFFFF
FFFLLLLLLLLLLLFFFFFFFFFFFFFFFFFFFFFFFFFFFFFFFFFFFFFFFLLLLLLLLLL
LLLLLLLLLLLLLLLLLLLLFFFFFFFFFFFFFFFFFFFFFFFFFFFFFFFFFFFFFFFFFFF
FFFFFFFFFFFFFFFLLLLLLLLLLLLLLLLLLLLLLFFFFFFFFFFF
\end{verbatim}
	
\item \begin{verbatim}
FFFFFFFFFFFLLLLLLLLLLLLLLLLLLLLLLFFFFFFFFFFFFFFFFFFFFFFFFFFFFFF
FFFFFFFFFFFFFFFFFFFFFFFFFFFFLLLLLLLLLLLLLLLLLLLLLLLLLLLLLLFFFFF
FFFFFFFFFFFFFFFFFFFFFFFFFFFFFFFFFFLLLLLLLLLLLFFFFFFFFFFFFFFFFFL
LLLLLLLLLLLLLLLLLLLLLLLLLLLLLLLLLLLLLLLLLLLLLLLLLLLLLLLLLLLLLLL
LLLFFFFFFFFFFFFFFFFFFFFFFFFFFFFFFFFFFFFFFFFFFFFF
\end{verbatim}

\item \begin{verbatim}
FFFFFFFFFFFFFFFFFFFFFFFFFFFFFFFFFFFFFFFFFFFFFLLLLLLLLLLLLLLLLLL
LLLLLLLLLLLLLLLLLLLLLLLLLLLLLLLLLLLLLLLLLLLLLLLLLFFFFFFFFFFFFFF
FFFLLLLLLLLLLLFFFFFFFFFFFFFFFFFFFFFFFFFFFFFFFFFFFFFFFLLLLLLLLLL
LLLLLLLLLLLLLLLLLLLLFFFFFFFFFFFFFFFFFFFFFFFFFFFFFFFFFFFFFFFFFFF
FFFFFFFFFFFFFFFLLLLLLLLLLLLLLLLLLLLLLFFFFFFFFFFF
\end{verbatim}

\item[casino.txt] Vorhersage:
\begin{verbatim}
FFFFFFFFFFFFFFFFFFFFFFFFFFFFFFFFFFFFFFFFFFFFFFFFLLLLLLLLLLLLLLL
LLLFFFFFFFFFFFFLLLLLLLLLLLLLLLLLLLLLLLLLLLLLLLLLLFFFFFFFFFFFFFF
FFFFFFFFFFFFFFFFFFFFFFFFFFFFFFFFFFFFFFFFFFFFFFFFFFFFFLLLLLLLLLL
LLLFFFFFFFFFFFFFFFFFFFFFFFFFFFFFFFFFFFFFFFFFFFFFFFFFFFFFFFFFFFF
FFFFFFFFFFFFFFFFFFLLLLLLLLLLLLLLLLLLLFFFFFFFFFFF
\end{verbatim}
Tatsächlich:
\begin{verbatim}
FFFFFFFFFFFFFFFFFFFFFFFFFFFFFFFFFFFFFFFFFFFFFLLLLLLLLLLLLLLLLLL
LLLFFFFFFFFFFFFLLLLLLLLLLLLLLLLFFFLLLLLLLLLLLLLLFFFFFFFFFFFFFFF
FFLLLLLLLLLLLLLFFFFFFFFFFFFFFFFFFFFFFFFFFFFFFFFFFFFFLLLLLLLLLLF
FFFFFFFFFFFFFFFFFFFFFFFFFFFFFFFFFFFFFFFFFFFFFFFFFFFFFFFFFFFFFFF
FFFFFFFFFFFFFFFLLLLLLLLLLLLLLLLLLLLLLFFFFFFFFFFF
\end{verbatim}

\end{enumerate}

\end{enumerate}
\end{document}
