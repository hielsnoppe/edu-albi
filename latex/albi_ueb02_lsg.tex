\documentclass{homework}
\usepackage{marvosym}
\usepackage{xyling}

\course{Algorithmische Bioinformatik}
\semester{Wintersemester 2012 / 2013}
\no{2}
\date{Montag, dem 29. Oktober 2012}
\author{Stefan Meißner (4279113) und Niels Hoppe (4356370)}
\tutorial{Dienstag 08:00 - 10:00}
\tutor{Alena van Bömmel (Übungsgruppe 3)}

\begin{document}
\maketitle
\begin{enumerate} 

\aufgabe{Maximum Parsimony}{40}

\begin{enumerate}

\item

Sei $k$ Vorgänger der Knoten $i$ und $j$, so gilt:

$$R_k=\begin{cases}
R_i \cap R_j & R_i \cap R_j \neq \emptyset\\
R_i \cup R_j & \text{ else}
\end{cases}$$

Merkmal 1 mit score 3

\Treek[-1]{1.5}{
&&&& \K{\{\textbf{C}\}} \B{dll}_{0} \B{ddrrr}_{0} &&&&\\
&& \K{\{A, \textbf{C}, G\}} \B{dl}_{0} \B{ddrr}_{1} &&&&&&\\
& \{A, \textbf{C}\} \B{dl}_{1} \B{dr}_{0} &&&&&& \{\textbf{C}, T\} \B{dl}_{0} \B{dr}_{1} &\\
A & & C & & G & & C & & T\\
a & & b & & e & & c & & d
}

Merkmal 2 mit score 3

\Treek[-1]{1.5}{
&&&& \{\textbf{A}, C\} \B{dll}_{0} \B{ddrrr}_{0} &&&&\\
&& \{\textbf{A}, C, T\} \B{dl}_{0} \B{ddrr}_{1} &&&&&&\\
& \{\textbf{A}, C\} \B{dl}_{0} \B{dr}_{1} &&&&&& \{\textbf{A}, C\} \B{dl}_{1} \B{dr}_{0} &\\
A & & C & & T & & C & & A\\
a & & b & & e & & c & & d
}

Merkmal 3 mit score 1

\Treek[-1]{1.5}{
&&&& \{\textbf{A}\} \B{dll}_{0} \B{ddrrr}_{0} &&&&\\
&& \{\textbf{A}, T\} \B{dl}_{0} \B{ddrr}_{1} &&&&&&\\
& \{\textbf{A}\} \B{dl}_{0} \B{dr}_{0} &&&&&& \{\textbf{A}\} \B{dl}_{0} \B{dr}_{0} &\\
A & & A & & T & & A & & A\\
a & & b & & e & & c & & d
}

Aus allen 3 Merkmalen ergibt sich dann ein minimaler Score von $3+3+1=7$.

\item

Sankoff-Algorithmus

Kostenfunktion

$$c(a, b) := \begin{cases}
1 & a \neq b\\
0 & \text{ else}
\end{cases}$$

\begin{enumerate}
\item Blätter
$$S_k(a) = \begin{cases}
0 & (a = X_n^k \rightarrow Sequenz)\\
\infty & \text{ else}
\end{cases}$$

\item innere Knoten, i,j-Kinderknoten, k-Elternknoten
$$S_k(a) = \min_b[S_i(b) + S(a, b)] + \min_b[S_j(b)+S(a, b)]$$
\end{enumerate}


Merkmal 1

\Treek[-1]{1.5}{
&&&& [5,3,4,5] \B{dll} \B{ddrrr} &&&&\\
&& [2,2,2,4] \B{dl} \B{ddrr} &&&&&&\\
& [1,1,2,3] \B{dl} \B{dr} &&&&&& [3,1,2,1] \B{dl} \B{dr} &\\
[0,\infty,\infty,\infty] && [\infty,0,\infty,\infty] && [\infty,\infty,0,\infty] && [\infty,0,\infty,\infty] && [\infty,\infty,\infty,0]\\
a & & b & & e & & c & & d
}

Merkmal 2

\Treek[-1]{1.5}{
&&&& [4,3,5,6] \B{dll} \B{ddrrr} &&&&\\
&& [3,2,3,3] \B{dl} \B{ddrr} &&&&&&\\
& [1,1,2,3] \B{dl} \B{dr} &&&&&& [1,1,2,3] \B{dl} \B{dr} &\\
[0,\infty,\infty,\infty] && [\infty,0,\infty,\infty] && [\infty,\infty,\infty,0] && [\infty,0,\infty,\infty] && [0,\infty,\infty,\infty]\\
a & & b & & e & & c & & d
}

Merkmal 3

\Treek[-1]{1.5}{
&&&& [2,5,5,8] \B{dll} \B{ddrrr} &&&&\\
&& [2,3,3,4] \B{dl} \B{ddrr} &&&&&&\\
& [0,2,2,4] \B{dl} \B{dr} &&&&&& [0,2,2,4] \B{dl} \B{dr} &\\
[0,\infty,\infty,\infty] && [0,\infty,\infty,\infty] && [\infty,\infty,\infty,0] && [0,\infty,\infty,\infty] && [0,\infty,\infty,\infty]\\
a & & b & & e & & c & & d
}


\item 

Die ersten beiden Spalten sind phylogenetisch informativer. Bedingt durch den geringeren Parsimony-Score der Spalte 3 sind dort kaum alternative Baumtopologien möglich.

\end{enumerate}

\aufgabe{Jukes-Cantor}{40}

Aus der Vorlesung ist folgendes bekannt:
\begin{itemize}
	\item $P^{(0)}=Id$
	\item $P^{(s+t)}=P^{(s)}+P^{(t)}$
	\item $\frac{d}{dt}P^{(t)} = Q$ für $t=0$
\end{itemize}
Um jetzt $P^{(t)}$ annähernd zu berechnen, kann $t$ in gleichmässige Teilstücke der Länge $l = t/m$ geteilt werden.
Je größer $m$ ist, desto eher ist
$P^{(l)}=Id$.\\
Daher: $P^{(t)}=(P^{(l)})^m=(Id)^m$\\
Da $m$ hinreichend gross ist, kann die additive Null hinzugefügt werden:\\
$(Id)^m=(Id+Q/m)^m \rightarrow exp(tQ)$

\aufgabe{Jukes-Cantor-Simulation}{40 (+20)}

\begin{enumerate}
\item[a, b)] Der vollständige Quelltext wird per E-Mail abgegeben.
Hier folgen relevante Ausschnitte mit Erklärungen:

%\lstinputlisting[language=python]{../python/albi_ueb02_a7.py}

\begin{lstlisting}[language=python]
def simulation(L, TE, dtE, aE, sE):
	"""
	simulation(100, 9, 5, -7, 6)
	"""
	seq = rseq(L)
	mut = seq
	data = []
	for step in xrange(pow(10, TE - sE)):
		for m in xrange(pow(10, sE - dtE)):
			mut = map(lambda s: mutate(s, aE), mut)
		d = dist(seq, mut)
		e = jcest(d, L)
		data.append((step, d, e));
	return data
\end{lstlisting}

Diese Funktion erzeugt eine Liste von Messergebnissen für eine Simulation mit
folgenden Parametern:
\begin{description}
\item[\texttt{L}] Länge der DNA-Sequenz
\item[\texttt{TE}] Gesamtzeit in $\log_{10} T$
\item[\texttt{dtE}] Dauer eines Mutationszyklus in $\log_{10} \Delta t$
\item[\texttt{aE}] Mutationsrate in $\log_{10} \alpha$
\item[\texttt{sE}] Dauer eines Messintervalles in $\log_{10} step$
\end{description}

Die äußere \texttt{for}-Schleife iteriert über die Anzahl gewünschter
Messergebnisse (Schritte), berechnet durch $\frac{T}{step}$. Die innere
\texttt{for}-Schleife iteriert über die Anzahl Mutationszyklen pro
Messintervall, berechnet durch $\frac{step}{\Delta t}$. Pro Messzyklus wird
jeweils die Hamming-Distanz $d$ zur ursprünglichen Sequenz und der mittels
Jukes-Cantor-Modell berechnete evolutionäre Abstand $e$ bestimmt und jeweils in
die Liste von Messwerten $data$ eingetragen.

Die Parameter für die Funktion \texttt{simulation} sind auch die Parameter für
den Aufruf über die Kommandozeile. Die Ausgabe erfolgt momentan in Form von
\LaTeX-Code, der ein Diagramm wie das folgende erzeugt:

\begin{center}
\setlength{\unitlength}{0.1mm}
\begin{picture}(1000,500)
\put(0,0){\vector(1,0){1000}}
\put(0,0){\vector(0,1){500}}
\put(0,0){\circle*{0.1}}
\put(0,-0.0){\circle*{0.1}}
\put(1,0){\circle*{0.1}}
\put(1,-0.0){\circle*{0.1}}
\put(2,0){\circle*{0.1}}
\put(2,-0.0){\circle*{0.1}}
\put(3,0){\circle*{0.1}}
\put(3,-0.0){\circle*{0.1}}
\put(4,5){\circle*{0.1}}
\put(4,0.0503363262455){\circle*{0.1}}
\put(5,5){\circle*{0.1}}
\put(5,0.0503363262455){\circle*{0.1}}
\put(6,10){\circle*{0.1}}
\put(6,0.101357521455){\circle*{0.1}}
\put(7,10){\circle*{0.1}}
\put(7,0.101357521455){\circle*{0.1}}
\put(8,10){\circle*{0.1}}
\put(8,0.101357521455){\circle*{0.1}}
\put(9,10){\circle*{0.1}}
\put(9,0.101357521455){\circle*{0.1}}
\put(10,10){\circle*{0.1}}
\put(10,0.101357521455){\circle*{0.1}}
\put(11,20){\circle*{0.1}}
\put(11,0.205530886856){\circle*{0.1}}
\put(12,20){\circle*{0.1}}
\put(12,0.205530886856){\circle*{0.1}}
\put(13,25){\circle*{0.1}}
\put(13,0.258723268076){\circle*{0.1}}
\put(14,25){\circle*{0.1}}
\put(14,0.258723268076){\circle*{0.1}}
\put(15,25){\circle*{0.1}}
\put(15,0.258723268076){\circle*{0.1}}
\put(16,25){\circle*{0.1}}
\put(16,0.258723268076){\circle*{0.1}}
\put(17,25){\circle*{0.1}}
\put(17,0.258723268076){\circle*{0.1}}
\put(18,25){\circle*{0.1}}
\put(18,0.258723268076){\circle*{0.1}}
\put(19,25){\circle*{0.1}}
\put(19,0.258723268076){\circle*{0.1}}
\put(20,25){\circle*{0.1}}
\put(20,0.258723268076){\circle*{0.1}}
\put(21,25){\circle*{0.1}}
\put(21,0.258723268076){\circle*{0.1}}
\put(22,25){\circle*{0.1}}
\put(22,0.258723268076){\circle*{0.1}}
\put(23,30){\circle*{0.1}}
\put(23,0.312681033521){\circle*{0.1}}
\put(24,30){\circle*{0.1}}
\put(24,0.312681033521){\circle*{0.1}}
\put(25,35){\circle*{0.1}}
\put(25,0.367426531351){\circle*{0.1}}
\put(26,35){\circle*{0.1}}
\put(26,0.367426531351){\circle*{0.1}}
\put(27,35){\circle*{0.1}}
\put(27,0.367426531351){\circle*{0.1}}
\put(28,35){\circle*{0.1}}
\put(28,0.367426531351){\circle*{0.1}}
\put(29,35){\circle*{0.1}}
\put(29,0.367426531351){\circle*{0.1}}
\put(30,35){\circle*{0.1}}
\put(30,0.367426531351){\circle*{0.1}}
\put(31,35){\circle*{0.1}}
\put(31,0.367426531351){\circle*{0.1}}
\put(32,35){\circle*{0.1}}
\put(32,0.367426531351){\circle*{0.1}}
\put(33,35){\circle*{0.1}}
\put(33,0.367426531351){\circle*{0.1}}
\put(34,35){\circle*{0.1}}
\put(34,0.367426531351){\circle*{0.1}}
\put(35,35){\circle*{0.1}}
\put(35,0.367426531351){\circle*{0.1}}
\put(36,35){\circle*{0.1}}
\put(36,0.367426531351){\circle*{0.1}}
\put(37,40){\circle*{0.1}}
\put(37,0.422983103045){\circle*{0.1}}
\put(38,45){\circle*{0.1}}
\put(38,0.479375143162){\circle*{0.1}}
\put(39,45){\circle*{0.1}}
\put(39,0.479375143162){\circle*{0.1}}
\put(40,45){\circle*{0.1}}
\put(40,0.479375143162){\circle*{0.1}}
\put(41,45){\circle*{0.1}}
\put(41,0.479375143162){\circle*{0.1}}
\put(42,45){\circle*{0.1}}
\put(42,0.479375143162){\circle*{0.1}}
\put(43,45){\circle*{0.1}}
\put(43,0.479375143162){\circle*{0.1}}
\put(44,50){\circle*{0.1}}
\put(44,0.536628163653){\circle*{0.1}}
\put(45,50){\circle*{0.1}}
\put(45,0.536628163653){\circle*{0.1}}
\put(46,55){\circle*{0.1}}
\put(46,0.594768863162){\circle*{0.1}}
\put(47,55){\circle*{0.1}}
\put(47,0.594768863162){\circle*{0.1}}
\put(48,55){\circle*{0.1}}
\put(48,0.594768863162){\circle*{0.1}}
\put(49,55){\circle*{0.1}}
\put(49,0.594768863162){\circle*{0.1}}
\put(50,55){\circle*{0.1}}
\put(50,0.594768863162){\circle*{0.1}}
\put(51,55){\circle*{0.1}}
\put(51,0.594768863162){\circle*{0.1}}
\put(52,55){\circle*{0.1}}
\put(52,0.594768863162){\circle*{0.1}}
\put(53,60){\circle*{0.1}}
\put(53,0.653825201793){\circle*{0.1}}
\put(54,65){\circle*{0.1}}
\put(54,0.713826481842){\circle*{0.1}}
\put(55,65){\circle*{0.1}}
\put(55,0.713826481842){\circle*{0.1}}
\put(56,65){\circle*{0.1}}
\put(56,0.713826481842){\circle*{0.1}}
\put(57,65){\circle*{0.1}}
\put(57,0.713826481842){\circle*{0.1}}
\put(58,65){\circle*{0.1}}
\put(58,0.713826481842){\circle*{0.1}}
\put(59,65){\circle*{0.1}}
\put(59,0.713826481842){\circle*{0.1}}
\put(60,65){\circle*{0.1}}
\put(60,0.713826481842){\circle*{0.1}}
\put(61,70){\circle*{0.1}}
\put(61,0.774803435111){\circle*{0.1}}
\put(62,70){\circle*{0.1}}
\put(62,0.774803435111){\circle*{0.1}}
\put(63,85){\circle*{0.1}}
\put(63,0.963919136212){\circle*{0.1}}
\put(64,90){\circle*{0.1}}
\put(64,1.02913817138){\circle*{0.1}}
\put(65,95){\circle*{0.1}}
\put(65,1.0955115855){\circle*{0.1}}
\put(66,95){\circle*{0.1}}
\put(66,1.0955115855){\circle*{0.1}}
\put(67,95){\circle*{0.1}}
\put(67,1.0955115855){\circle*{0.1}}
\put(68,95){\circle*{0.1}}
\put(68,1.0955115855){\circle*{0.1}}
\put(69,95){\circle*{0.1}}
\put(69,1.0955115855){\circle*{0.1}}
\put(70,95){\circle*{0.1}}
\put(70,1.0955115855){\circle*{0.1}}
\put(71,95){\circle*{0.1}}
\put(71,1.0955115855){\circle*{0.1}}
\put(72,105){\circle*{0.1}}
\put(72,1.23189025115){\circle*{0.1}}
\put(73,105){\circle*{0.1}}
\put(73,1.23189025115){\circle*{0.1}}
\put(74,105){\circle*{0.1}}
\put(74,1.23189025115){\circle*{0.1}}
\put(75,110){\circle*{0.1}}
\put(75,1.30198574994){\circle*{0.1}}
\put(76,110){\circle*{0.1}}
\put(76,1.30198574994){\circle*{0.1}}
\put(77,110){\circle*{0.1}}
\put(77,1.30198574994){\circle*{0.1}}
\put(78,110){\circle*{0.1}}
\put(78,1.30198574994){\circle*{0.1}}
\put(79,110){\circle*{0.1}}
\put(79,1.30198574994){\circle*{0.1}}
\put(80,110){\circle*{0.1}}
\put(80,1.30198574994){\circle*{0.1}}
\put(81,110){\circle*{0.1}}
\put(81,1.30198574994){\circle*{0.1}}
\put(82,115){\circle*{0.1}}
\put(82,1.37341648108){\circle*{0.1}}
\put(83,115){\circle*{0.1}}
\put(83,1.37341648108){\circle*{0.1}}
\put(84,115){\circle*{0.1}}
\put(84,1.37341648108){\circle*{0.1}}
\put(85,120){\circle*{0.1}}
\put(85,1.44623430304){\circle*{0.1}}
\put(86,120){\circle*{0.1}}
\put(86,1.44623430304){\circle*{0.1}}
\put(87,120){\circle*{0.1}}
\put(87,1.44623430304){\circle*{0.1}}
\put(88,120){\circle*{0.1}}
\put(88,1.44623430304){\circle*{0.1}}
\put(89,120){\circle*{0.1}}
\put(89,1.44623430304){\circle*{0.1}}
\put(90,120){\circle*{0.1}}
\put(90,1.44623430304){\circle*{0.1}}
\put(91,120){\circle*{0.1}}
\put(91,1.44623430304){\circle*{0.1}}
\put(92,120){\circle*{0.1}}
\put(92,1.44623430304){\circle*{0.1}}
\put(93,120){\circle*{0.1}}
\put(93,1.44623430304){\circle*{0.1}}
\put(94,125){\circle*{0.1}}
\put(94,1.52049415541){\circle*{0.1}}
\put(95,125){\circle*{0.1}}
\put(95,1.52049415541){\circle*{0.1}}
\put(96,125){\circle*{0.1}}
\put(96,1.52049415541){\circle*{0.1}}
\put(97,120){\circle*{0.1}}
\put(97,1.44623430304){\circle*{0.1}}
\put(98,120){\circle*{0.1}}
\put(98,1.44623430304){\circle*{0.1}}
\put(99,125){\circle*{0.1}}
\put(99,1.52049415541){\circle*{0.1}}
\put(100,130){\circle*{0.1}}
\put(100,1.59625430785){\circle*{0.1}}
\put(101,130){\circle*{0.1}}
\put(101,1.59625430785){\circle*{0.1}}
\put(102,130){\circle*{0.1}}
\put(102,1.59625430785){\circle*{0.1}}
\put(103,130){\circle*{0.1}}
\put(103,1.59625430785){\circle*{0.1}}
\put(104,130){\circle*{0.1}}
\put(104,1.59625430785){\circle*{0.1}}
\put(105,130){\circle*{0.1}}
\put(105,1.59625430785){\circle*{0.1}}
\put(106,130){\circle*{0.1}}
\put(106,1.59625430785){\circle*{0.1}}
\put(107,130){\circle*{0.1}}
\put(107,1.59625430785){\circle*{0.1}}
\put(108,130){\circle*{0.1}}
\put(108,1.59625430785){\circle*{0.1}}
\put(109,130){\circle*{0.1}}
\put(109,1.59625430785){\circle*{0.1}}
\put(110,130){\circle*{0.1}}
\put(110,1.59625430785){\circle*{0.1}}
\put(111,130){\circle*{0.1}}
\put(111,1.59625430785){\circle*{0.1}}
\put(112,130){\circle*{0.1}}
\put(112,1.59625430785){\circle*{0.1}}
\put(113,130){\circle*{0.1}}
\put(113,1.59625430785){\circle*{0.1}}
\put(114,130){\circle*{0.1}}
\put(114,1.59625430785){\circle*{0.1}}
\put(115,130){\circle*{0.1}}
\put(115,1.59625430785){\circle*{0.1}}
\put(116,130){\circle*{0.1}}
\put(116,1.59625430785){\circle*{0.1}}
\put(117,130){\circle*{0.1}}
\put(117,1.59625430785){\circle*{0.1}}
\put(118,130){\circle*{0.1}}
\put(118,1.59625430785){\circle*{0.1}}
\put(119,130){\circle*{0.1}}
\put(119,1.59625430785){\circle*{0.1}}
\put(120,130){\circle*{0.1}}
\put(120,1.59625430785){\circle*{0.1}}
\put(121,130){\circle*{0.1}}
\put(121,1.59625430785){\circle*{0.1}}
\put(122,140){\circle*{0.1}}
\put(122,1.75252691935){\circle*{0.1}}
\put(123,140){\circle*{0.1}}
\put(123,1.75252691935){\circle*{0.1}}
\put(124,140){\circle*{0.1}}
\put(124,1.75252691935){\circle*{0.1}}
\put(125,140){\circle*{0.1}}
\put(125,1.75252691935){\circle*{0.1}}
\put(126,140){\circle*{0.1}}
\put(126,1.75252691935){\circle*{0.1}}
\put(127,145){\circle*{0.1}}
\put(127,1.83317518893){\circle*{0.1}}
\put(128,145){\circle*{0.1}}
\put(128,1.83317518893){\circle*{0.1}}
\put(129,145){\circle*{0.1}}
\put(129,1.83317518893){\circle*{0.1}}
\put(130,145){\circle*{0.1}}
\put(130,1.83317518893){\circle*{0.1}}
\put(131,145){\circle*{0.1}}
\put(131,1.83317518893){\circle*{0.1}}
\put(132,150){\circle*{0.1}}
\put(132,1.91559608912){\circle*{0.1}}
\put(133,150){\circle*{0.1}}
\put(133,1.91559608912){\circle*{0.1}}
\put(134,150){\circle*{0.1}}
\put(134,1.91559608912){\circle*{0.1}}
\put(135,150){\circle*{0.1}}
\put(135,1.91559608912){\circle*{0.1}}
\put(136,145){\circle*{0.1}}
\put(136,1.83317518893){\circle*{0.1}}
\put(137,145){\circle*{0.1}}
\put(137,1.83317518893){\circle*{0.1}}
\put(138,145){\circle*{0.1}}
\put(138,1.83317518893){\circle*{0.1}}
\put(139,145){\circle*{0.1}}
\put(139,1.83317518893){\circle*{0.1}}
\put(140,145){\circle*{0.1}}
\put(140,1.83317518893){\circle*{0.1}}
\put(141,150){\circle*{0.1}}
\put(141,1.91559608912){\circle*{0.1}}
\put(142,150){\circle*{0.1}}
\put(142,1.91559608912){\circle*{0.1}}
\put(143,150){\circle*{0.1}}
\put(143,1.91559608912){\circle*{0.1}}
\put(144,150){\circle*{0.1}}
\put(144,1.91559608912){\circle*{0.1}}
\put(145,155){\circle*{0.1}}
\put(145,1.99986929857){\circle*{0.1}}
\put(146,155){\circle*{0.1}}
\put(146,1.99986929857){\circle*{0.1}}
\put(147,155){\circle*{0.1}}
\put(147,1.99986929857){\circle*{0.1}}
\put(148,155){\circle*{0.1}}
\put(148,1.99986929857){\circle*{0.1}}
\put(149,155){\circle*{0.1}}
\put(149,1.99986929857){\circle*{0.1}}
\put(150,150){\circle*{0.1}}
\put(150,1.91559608912){\circle*{0.1}}
\put(151,150){\circle*{0.1}}
\put(151,1.91559608912){\circle*{0.1}}
\put(152,155){\circle*{0.1}}
\put(152,1.99986929857){\circle*{0.1}}
\put(153,155){\circle*{0.1}}
\put(153,1.99986929857){\circle*{0.1}}
\put(154,155){\circle*{0.1}}
\put(154,1.99986929857){\circle*{0.1}}
\put(155,155){\circle*{0.1}}
\put(155,1.99986929857){\circle*{0.1}}
\put(156,155){\circle*{0.1}}
\put(156,1.99986929857){\circle*{0.1}}
\put(157,155){\circle*{0.1}}
\put(157,1.99986929857){\circle*{0.1}}
\put(158,155){\circle*{0.1}}
\put(158,1.99986929857){\circle*{0.1}}
\put(159,155){\circle*{0.1}}
\put(159,1.99986929857){\circle*{0.1}}
\put(160,155){\circle*{0.1}}
\put(160,1.99986929857){\circle*{0.1}}
\put(161,155){\circle*{0.1}}
\put(161,1.99986929857){\circle*{0.1}}
\put(162,155){\circle*{0.1}}
\put(162,1.99986929857){\circle*{0.1}}
\put(163,155){\circle*{0.1}}
\put(163,1.99986929857){\circle*{0.1}}
\put(164,155){\circle*{0.1}}
\put(164,1.99986929857){\circle*{0.1}}
\put(165,155){\circle*{0.1}}
\put(165,1.99986929857){\circle*{0.1}}
\put(166,155){\circle*{0.1}}
\put(166,1.99986929857){\circle*{0.1}}
\put(167,155){\circle*{0.1}}
\put(167,1.99986929857){\circle*{0.1}}
\put(168,155){\circle*{0.1}}
\put(168,1.99986929857){\circle*{0.1}}
\put(169,155){\circle*{0.1}}
\put(169,1.99986929857){\circle*{0.1}}
\put(170,155){\circle*{0.1}}
\put(170,1.99986929857){\circle*{0.1}}
\put(171,160){\circle*{0.1}}
\put(171,2.08607999191){\circle*{0.1}}
\put(172,160){\circle*{0.1}}
\put(172,2.08607999191){\circle*{0.1}}
\put(173,160){\circle*{0.1}}
\put(173,2.08607999191){\circle*{0.1}}
\put(174,165){\circle*{0.1}}
\put(174,2.1743193572){\circle*{0.1}}
\put(175,165){\circle*{0.1}}
\put(175,2.1743193572){\circle*{0.1}}
\put(176,165){\circle*{0.1}}
\put(176,2.1743193572){\circle*{0.1}}
\put(177,165){\circle*{0.1}}
\put(177,2.1743193572){\circle*{0.1}}
\put(178,165){\circle*{0.1}}
\put(178,2.1743193572){\circle*{0.1}}
\put(179,165){\circle*{0.1}}
\put(179,2.1743193572){\circle*{0.1}}
\put(180,165){\circle*{0.1}}
\put(180,2.1743193572){\circle*{0.1}}
\put(181,165){\circle*{0.1}}
\put(181,2.1743193572){\circle*{0.1}}
\put(182,165){\circle*{0.1}}
\put(182,2.1743193572){\circle*{0.1}}
\put(183,170){\circle*{0.1}}
\put(183,2.26468517562){\circle*{0.1}}
\put(184,170){\circle*{0.1}}
\put(184,2.26468517562){\circle*{0.1}}
\put(185,170){\circle*{0.1}}
\put(185,2.26468517562){\circle*{0.1}}
\put(186,170){\circle*{0.1}}
\put(186,2.26468517562){\circle*{0.1}}
\put(187,170){\circle*{0.1}}
\put(187,2.26468517562){\circle*{0.1}}
\put(188,170){\circle*{0.1}}
\put(188,2.26468517562){\circle*{0.1}}
\put(189,170){\circle*{0.1}}
\put(189,2.26468517562){\circle*{0.1}}
\put(190,170){\circle*{0.1}}
\put(190,2.26468517562){\circle*{0.1}}
\put(191,175){\circle*{0.1}}
\put(191,2.35728247283){\circle*{0.1}}
\put(192,175){\circle*{0.1}}
\put(192,2.35728247283){\circle*{0.1}}
\put(193,180){\circle*{0.1}}
\put(193,2.45222425277){\circle*{0.1}}
\put(194,180){\circle*{0.1}}
\put(194,2.45222425277){\circle*{0.1}}
\put(195,180){\circle*{0.1}}
\put(195,2.45222425277){\circle*{0.1}}
\put(196,180){\circle*{0.1}}
\put(196,2.45222425277){\circle*{0.1}}
\put(197,180){\circle*{0.1}}
\put(197,2.45222425277){\circle*{0.1}}
\put(198,180){\circle*{0.1}}
\put(198,2.45222425277){\circle*{0.1}}
\put(199,185){\circle*{0.1}}
\put(199,2.54963232679){\circle*{0.1}}
\put(200,180){\circle*{0.1}}
\put(200,2.45222425277){\circle*{0.1}}
\put(201,180){\circle*{0.1}}
\put(201,2.45222425277){\circle*{0.1}}
\put(202,185){\circle*{0.1}}
\put(202,2.54963232679){\circle*{0.1}}
\put(203,185){\circle*{0.1}}
\put(203,2.54963232679){\circle*{0.1}}
\put(204,185){\circle*{0.1}}
\put(204,2.54963232679){\circle*{0.1}}
\put(205,185){\circle*{0.1}}
\put(205,2.54963232679){\circle*{0.1}}
\put(206,190){\circle*{0.1}}
\put(206,2.64963825335){\circle*{0.1}}
\put(207,190){\circle*{0.1}}
\put(207,2.64963825335){\circle*{0.1}}
\put(208,190){\circle*{0.1}}
\put(208,2.64963825335){\circle*{0.1}}
\put(209,195){\circle*{0.1}}
\put(209,2.75238440655){\circle*{0.1}}
\put(210,195){\circle*{0.1}}
\put(210,2.75238440655){\circle*{0.1}}
\put(211,195){\circle*{0.1}}
\put(211,2.75238440655){\circle*{0.1}}
\put(212,195){\circle*{0.1}}
\put(212,2.75238440655){\circle*{0.1}}
\put(213,195){\circle*{0.1}}
\put(213,2.75238440655){\circle*{0.1}}
\put(214,195){\circle*{0.1}}
\put(214,2.75238440655){\circle*{0.1}}
\put(215,200){\circle*{0.1}}
\put(215,2.85802519518){\circle*{0.1}}
\put(216,195){\circle*{0.1}}
\put(216,2.75238440655){\circle*{0.1}}
\put(217,195){\circle*{0.1}}
\put(217,2.75238440655){\circle*{0.1}}
\put(218,195){\circle*{0.1}}
\put(218,2.75238440655){\circle*{0.1}}
\put(219,195){\circle*{0.1}}
\put(219,2.75238440655){\circle*{0.1}}
\put(220,195){\circle*{0.1}}
\put(220,2.75238440655){\circle*{0.1}}
\put(221,195){\circle*{0.1}}
\put(221,2.75238440655){\circle*{0.1}}
\put(222,195){\circle*{0.1}}
\put(222,2.75238440655){\circle*{0.1}}
\put(223,200){\circle*{0.1}}
\put(223,2.85802519518){\circle*{0.1}}
\put(224,200){\circle*{0.1}}
\put(224,2.85802519518){\circle*{0.1}}
\put(225,205){\circle*{0.1}}
\put(225,2.96672845845){\circle*{0.1}}
\put(226,205){\circle*{0.1}}
\put(226,2.96672845845){\circle*{0.1}}
\put(227,205){\circle*{0.1}}
\put(227,2.96672845845){\circle*{0.1}}
\put(228,205){\circle*{0.1}}
\put(228,2.96672845845){\circle*{0.1}}
\put(229,205){\circle*{0.1}}
\put(229,2.96672845845){\circle*{0.1}}
\put(230,205){\circle*{0.1}}
\put(230,2.96672845845){\circle*{0.1}}
\put(231,205){\circle*{0.1}}
\put(231,2.96672845845){\circle*{0.1}}
\put(232,205){\circle*{0.1}}
\put(232,2.96672845845){\circle*{0.1}}
\put(233,215){\circle*{0.1}}
\put(233,3.19407079026){\circle*{0.1}}
\put(234,215){\circle*{0.1}}
\put(234,3.19407079026){\circle*{0.1}}
\put(235,215){\circle*{0.1}}
\put(235,3.19407079026){\circle*{0.1}}
\put(236,220){\circle*{0.1}}
\put(236,3.31312840894){\circle*{0.1}}
\put(237,220){\circle*{0.1}}
\put(237,3.31312840894){\circle*{0.1}}
\put(238,220){\circle*{0.1}}
\put(238,3.31312840894){\circle*{0.1}}
\put(239,220){\circle*{0.1}}
\put(239,3.31312840894){\circle*{0.1}}
\put(240,220){\circle*{0.1}}
\put(240,3.31312840894){\circle*{0.1}}
\put(241,220){\circle*{0.1}}
\put(241,3.31312840894){\circle*{0.1}}
\put(242,225){\circle*{0.1}}
\put(242,3.43609024453){\circle*{0.1}}
\put(243,225){\circle*{0.1}}
\put(243,3.43609024453){\circle*{0.1}}
\put(244,225){\circle*{0.1}}
\put(244,3.43609024453){\circle*{0.1}}
\put(245,225){\circle*{0.1}}
\put(245,3.43609024453){\circle*{0.1}}
\put(246,230){\circle*{0.1}}
\put(246,3.56322106331){\circle*{0.1}}
\put(247,230){\circle*{0.1}}
\put(247,3.56322106331){\circle*{0.1}}
\put(248,235){\circle*{0.1}}
\put(248,3.6948135126){\circle*{0.1}}
\put(249,235){\circle*{0.1}}
\put(249,3.6948135126){\circle*{0.1}}
\put(250,235){\circle*{0.1}}
\put(250,3.6948135126){\circle*{0.1}}
\put(251,235){\circle*{0.1}}
\put(251,3.6948135126){\circle*{0.1}}
\put(252,240){\circle*{0.1}}
\put(252,3.83119217824){\circle*{0.1}}
\put(253,240){\circle*{0.1}}
\put(253,3.83119217824){\circle*{0.1}}
\put(254,240){\circle*{0.1}}
\put(254,3.83119217824){\circle*{0.1}}
\put(255,240){\circle*{0.1}}
\put(255,3.83119217824){\circle*{0.1}}
\put(256,240){\circle*{0.1}}
\put(256,3.83119217824){\circle*{0.1}}
\put(257,240){\circle*{0.1}}
\put(257,3.83119217824){\circle*{0.1}}
\put(258,235){\circle*{0.1}}
\put(258,3.6948135126){\circle*{0.1}}
\put(259,235){\circle*{0.1}}
\put(259,3.6948135126){\circle*{0.1}}
\put(260,235){\circle*{0.1}}
\put(260,3.6948135126){\circle*{0.1}}
\put(261,235){\circle*{0.1}}
\put(261,3.6948135126){\circle*{0.1}}
\put(262,240){\circle*{0.1}}
\put(262,3.83119217824){\circle*{0.1}}
\put(263,245){\circle*{0.1}}
\put(263,3.97271840818){\circle*{0.1}}
\put(264,245){\circle*{0.1}}
\put(264,3.97271840818){\circle*{0.1}}
\put(265,255){\circle*{0.1}}
\put(265,4.27287856196){\circle*{0.1}}
\put(266,255){\circle*{0.1}}
\put(266,4.27287856196){\circle*{0.1}}
\put(267,255){\circle*{0.1}}
\put(267,4.27287856196){\circle*{0.1}}
\put(268,255){\circle*{0.1}}
\put(268,4.27287856196){\circle*{0.1}}
\put(269,255){\circle*{0.1}}
\put(269,4.27287856196){\circle*{0.1}}
\put(270,255){\circle*{0.1}}
\put(270,4.27287856196){\circle*{0.1}}
\put(271,260){\circle*{0.1}}
\put(271,4.43247711603){\circle*{0.1}}
\put(272,260){\circle*{0.1}}
\put(272,4.43247711603){\circle*{0.1}}
\put(273,260){\circle*{0.1}}
\put(273,4.43247711603){\circle*{0.1}}
\put(274,255){\circle*{0.1}}
\put(274,4.27287856196){\circle*{0.1}}
\put(275,255){\circle*{0.1}}
\put(275,4.27287856196){\circle*{0.1}}
\put(276,265){\circle*{0.1}}
\put(276,4.59917122567){\circle*{0.1}}
\put(277,265){\circle*{0.1}}
\put(277,4.59917122567){\circle*{0.1}}
\put(278,260){\circle*{0.1}}
\put(278,4.43247711603){\circle*{0.1}}
\put(279,260){\circle*{0.1}}
\put(279,4.43247711603){\circle*{0.1}}
\put(280,260){\circle*{0.1}}
\put(280,4.43247711603){\circle*{0.1}}
\put(281,260){\circle*{0.1}}
\put(281,4.43247711603){\circle*{0.1}}
\put(282,260){\circle*{0.1}}
\put(282,4.43247711603){\circle*{0.1}}
\put(283,260){\circle*{0.1}}
\put(283,4.43247711603){\circle*{0.1}}
\put(284,260){\circle*{0.1}}
\put(284,4.43247711603){\circle*{0.1}}
\put(285,260){\circle*{0.1}}
\put(285,4.43247711603){\circle*{0.1}}
\put(286,260){\circle*{0.1}}
\put(286,4.43247711603){\circle*{0.1}}
\put(287,260){\circle*{0.1}}
\put(287,4.43247711603){\circle*{0.1}}
\put(288,260){\circle*{0.1}}
\put(288,4.43247711603){\circle*{0.1}}
\put(289,260){\circle*{0.1}}
\put(289,4.43247711603){\circle*{0.1}}
\put(290,260){\circle*{0.1}}
\put(290,4.43247711603){\circle*{0.1}}
\put(291,260){\circle*{0.1}}
\put(291,4.43247711603){\circle*{0.1}}
\put(292,260){\circle*{0.1}}
\put(292,4.43247711603){\circle*{0.1}}
\put(293,260){\circle*{0.1}}
\put(293,4.43247711603){\circle*{0.1}}
\put(294,255){\circle*{0.1}}
\put(294,4.27287856196){\circle*{0.1}}
\put(295,255){\circle*{0.1}}
\put(295,4.27287856196){\circle*{0.1}}
\put(296,255){\circle*{0.1}}
\put(296,4.27287856196){\circle*{0.1}}
\put(297,255){\circle*{0.1}}
\put(297,4.27287856196){\circle*{0.1}}
\put(298,255){\circle*{0.1}}
\put(298,4.27287856196){\circle*{0.1}}
\put(299,255){\circle*{0.1}}
\put(299,4.27287856196){\circle*{0.1}}
\put(300,255){\circle*{0.1}}
\put(300,4.27287856196){\circle*{0.1}}
\put(301,255){\circle*{0.1}}
\put(301,4.27287856196){\circle*{0.1}}
\put(302,255){\circle*{0.1}}
\put(302,4.27287856196){\circle*{0.1}}
\put(303,255){\circle*{0.1}}
\put(303,4.27287856196){\circle*{0.1}}
\put(304,255){\circle*{0.1}}
\put(304,4.27287856196){\circle*{0.1}}
\put(305,255){\circle*{0.1}}
\put(305,4.27287856196){\circle*{0.1}}
\put(306,255){\circle*{0.1}}
\put(306,4.27287856196){\circle*{0.1}}
\put(307,255){\circle*{0.1}}
\put(307,4.27287856196){\circle*{0.1}}
\put(308,255){\circle*{0.1}}
\put(308,4.27287856196){\circle*{0.1}}
\put(309,255){\circle*{0.1}}
\put(309,4.27287856196){\circle*{0.1}}
\put(310,255){\circle*{0.1}}
\put(310,4.27287856196){\circle*{0.1}}
\put(311,260){\circle*{0.1}}
\put(311,4.43247711603){\circle*{0.1}}
\put(312,260){\circle*{0.1}}
\put(312,4.43247711603){\circle*{0.1}}
\put(313,260){\circle*{0.1}}
\put(313,4.43247711603){\circle*{0.1}}
\put(314,260){\circle*{0.1}}
\put(314,4.43247711603){\circle*{0.1}}
\put(315,260){\circle*{0.1}}
\put(315,4.43247711603){\circle*{0.1}}
\put(316,265){\circle*{0.1}}
\put(316,4.59917122567){\circle*{0.1}}
\put(317,260){\circle*{0.1}}
\put(317,4.43247711603){\circle*{0.1}}
\put(318,260){\circle*{0.1}}
\put(318,4.43247711603){\circle*{0.1}}
\put(319,260){\circle*{0.1}}
\put(319,4.43247711603){\circle*{0.1}}
\put(320,260){\circle*{0.1}}
\put(320,4.43247711603){\circle*{0.1}}
\put(321,260){\circle*{0.1}}
\put(321,4.43247711603){\circle*{0.1}}
\put(322,260){\circle*{0.1}}
\put(322,4.43247711603){\circle*{0.1}}
\put(323,260){\circle*{0.1}}
\put(323,4.43247711603){\circle*{0.1}}
\put(324,265){\circle*{0.1}}
\put(324,4.59917122567){\circle*{0.1}}
\put(325,265){\circle*{0.1}}
\put(325,4.59917122567){\circle*{0.1}}
\put(326,265){\circle*{0.1}}
\put(326,4.59917122567){\circle*{0.1}}
\put(327,265){\circle*{0.1}}
\put(327,4.59917122567){\circle*{0.1}}
\put(328,265){\circle*{0.1}}
\put(328,4.59917122567){\circle*{0.1}}
\put(329,265){\circle*{0.1}}
\put(329,4.59917122567){\circle*{0.1}}
\put(330,270){\circle*{0.1}}
\put(330,4.7736212843){\circle*{0.1}}
\put(331,270){\circle*{0.1}}
\put(331,4.7736212843){\circle*{0.1}}
\put(332,270){\circle*{0.1}}
\put(332,4.7736212843){\circle*{0.1}}
\put(333,270){\circle*{0.1}}
\put(333,4.7736212843){\circle*{0.1}}
\put(334,275){\circle*{0.1}}
\put(334,4.95658439993){\circle*{0.1}}
\put(335,275){\circle*{0.1}}
\put(335,4.95658439993){\circle*{0.1}}
\put(336,275){\circle*{0.1}}
\put(336,4.95658439993){\circle*{0.1}}
\put(337,280){\circle*{0.1}}
\put(337,5.14893425389){\circle*{0.1}}
\put(338,280){\circle*{0.1}}
\put(338,5.14893425389){\circle*{0.1}}
\put(339,280){\circle*{0.1}}
\put(339,5.14893425389){\circle*{0.1}}
\put(340,280){\circle*{0.1}}
\put(340,5.14893425389){\circle*{0.1}}
\put(341,280){\circle*{0.1}}
\put(341,5.14893425389){\circle*{0.1}}
\put(342,280){\circle*{0.1}}
\put(342,5.14893425389){\circle*{0.1}}
\put(343,280){\circle*{0.1}}
\put(343,5.14893425389){\circle*{0.1}}
\put(344,280){\circle*{0.1}}
\put(344,5.14893425389){\circle*{0.1}}
\put(345,280){\circle*{0.1}}
\put(345,5.14893425389){\circle*{0.1}}
\put(346,280){\circle*{0.1}}
\put(346,5.14893425389){\circle*{0.1}}
\put(347,280){\circle*{0.1}}
\put(347,5.14893425389){\circle*{0.1}}
\put(348,280){\circle*{0.1}}
\put(348,5.14893425389){\circle*{0.1}}
\put(349,275){\circle*{0.1}}
\put(349,4.95658439993){\circle*{0.1}}
\put(350,275){\circle*{0.1}}
\put(350,4.95658439993){\circle*{0.1}}
\put(351,275){\circle*{0.1}}
\put(351,4.95658439993){\circle*{0.1}}
\put(352,275){\circle*{0.1}}
\put(352,4.95658439993){\circle*{0.1}}
\put(353,275){\circle*{0.1}}
\put(353,4.95658439993){\circle*{0.1}}
\put(354,275){\circle*{0.1}}
\put(354,4.95658439993){\circle*{0.1}}
\put(355,280){\circle*{0.1}}
\put(355,5.14893425389){\circle*{0.1}}
\put(356,280){\circle*{0.1}}
\put(356,5.14893425389){\circle*{0.1}}
\put(357,280){\circle*{0.1}}
\put(357,5.14893425389){\circle*{0.1}}
\put(358,285){\circle*{0.1}}
\put(358,5.35168633365){\circle*{0.1}}
\put(359,280){\circle*{0.1}}
\put(359,5.14893425389){\circle*{0.1}}
\put(360,280){\circle*{0.1}}
\put(360,5.14893425389){\circle*{0.1}}
\put(361,285){\circle*{0.1}}
\put(361,5.35168633365){\circle*{0.1}}
\put(362,285){\circle*{0.1}}
\put(362,5.35168633365){\circle*{0.1}}
\put(363,285){\circle*{0.1}}
\put(363,5.35168633365){\circle*{0.1}}
\put(364,285){\circle*{0.1}}
\put(364,5.35168633365){\circle*{0.1}}
\put(365,285){\circle*{0.1}}
\put(365,5.35168633365){\circle*{0.1}}
\put(366,280){\circle*{0.1}}
\put(366,5.14893425389){\circle*{0.1}}
\put(367,275){\circle*{0.1}}
\put(367,4.95658439993){\circle*{0.1}}
\put(368,280){\circle*{0.1}}
\put(368,5.14893425389){\circle*{0.1}}
\put(369,285){\circle*{0.1}}
\put(369,5.35168633365){\circle*{0.1}}
\put(370,285){\circle*{0.1}}
\put(370,5.35168633365){\circle*{0.1}}
\put(371,290){\circle*{0.1}}
\put(371,5.56603038555){\circle*{0.1}}
\put(372,290){\circle*{0.1}}
\put(372,5.56603038555){\circle*{0.1}}
\put(373,290){\circle*{0.1}}
\put(373,5.56603038555){\circle*{0.1}}
\put(374,290){\circle*{0.1}}
\put(374,5.56603038555){\circle*{0.1}}
\put(375,290){\circle*{0.1}}
\put(375,5.56603038555){\circle*{0.1}}
\put(376,290){\circle*{0.1}}
\put(376,5.56603038555){\circle*{0.1}}
\put(377,290){\circle*{0.1}}
\put(377,5.56603038555){\circle*{0.1}}
\put(378,290){\circle*{0.1}}
\put(378,5.56603038555){\circle*{0.1}}
\put(379,285){\circle*{0.1}}
\put(379,5.35168633365){\circle*{0.1}}
\put(380,280){\circle*{0.1}}
\put(380,5.14893425389){\circle*{0.1}}
\put(381,280){\circle*{0.1}}
\put(381,5.14893425389){\circle*{0.1}}
\put(382,275){\circle*{0.1}}
\put(382,4.95658439993){\circle*{0.1}}
\put(383,275){\circle*{0.1}}
\put(383,4.95658439993){\circle*{0.1}}
\put(384,275){\circle*{0.1}}
\put(384,4.95658439993){\circle*{0.1}}
\put(385,275){\circle*{0.1}}
\put(385,4.95658439993){\circle*{0.1}}
\put(386,275){\circle*{0.1}}
\put(386,4.95658439993){\circle*{0.1}}
\put(387,275){\circle*{0.1}}
\put(387,4.95658439993){\circle*{0.1}}
\put(388,275){\circle*{0.1}}
\put(388,4.95658439993){\circle*{0.1}}
\put(389,275){\circle*{0.1}}
\put(389,4.95658439993){\circle*{0.1}}
\put(390,275){\circle*{0.1}}
\put(390,4.95658439993){\circle*{0.1}}
\put(391,275){\circle*{0.1}}
\put(391,4.95658439993){\circle*{0.1}}
\put(392,275){\circle*{0.1}}
\put(392,4.95658439993){\circle*{0.1}}
\put(393,275){\circle*{0.1}}
\put(393,4.95658439993){\circle*{0.1}}
\put(394,275){\circle*{0.1}}
\put(394,4.95658439993){\circle*{0.1}}
\put(395,275){\circle*{0.1}}
\put(395,4.95658439993){\circle*{0.1}}
\put(396,275){\circle*{0.1}}
\put(396,4.95658439993){\circle*{0.1}}
\put(397,275){\circle*{0.1}}
\put(397,4.95658439993){\circle*{0.1}}
\put(398,275){\circle*{0.1}}
\put(398,4.95658439993){\circle*{0.1}}
\put(399,275){\circle*{0.1}}
\put(399,4.95658439993){\circle*{0.1}}
\put(400,275){\circle*{0.1}}
\put(400,4.95658439993){\circle*{0.1}}
\put(401,275){\circle*{0.1}}
\put(401,4.95658439993){\circle*{0.1}}
\put(402,275){\circle*{0.1}}
\put(402,4.95658439993){\circle*{0.1}}
\put(403,275){\circle*{0.1}}
\put(403,4.95658439993){\circle*{0.1}}
\put(404,275){\circle*{0.1}}
\put(404,4.95658439993){\circle*{0.1}}
\put(405,275){\circle*{0.1}}
\put(405,4.95658439993){\circle*{0.1}}
\put(406,275){\circle*{0.1}}
\put(406,4.95658439993){\circle*{0.1}}
\put(407,275){\circle*{0.1}}
\put(407,4.95658439993){\circle*{0.1}}
\put(408,275){\circle*{0.1}}
\put(408,4.95658439993){\circle*{0.1}}
\put(409,275){\circle*{0.1}}
\put(409,4.95658439993){\circle*{0.1}}
\put(410,275){\circle*{0.1}}
\put(410,4.95658439993){\circle*{0.1}}
\put(411,275){\circle*{0.1}}
\put(411,4.95658439993){\circle*{0.1}}
\put(412,275){\circle*{0.1}}
\put(412,4.95658439993){\circle*{0.1}}
\put(413,275){\circle*{0.1}}
\put(413,4.95658439993){\circle*{0.1}}
\put(414,275){\circle*{0.1}}
\put(414,4.95658439993){\circle*{0.1}}
\put(415,275){\circle*{0.1}}
\put(415,4.95658439993){\circle*{0.1}}
\put(416,275){\circle*{0.1}}
\put(416,4.95658439993){\circle*{0.1}}
\put(417,275){\circle*{0.1}}
\put(417,4.95658439993){\circle*{0.1}}
\put(418,275){\circle*{0.1}}
\put(418,4.95658439993){\circle*{0.1}}
\put(419,275){\circle*{0.1}}
\put(419,4.95658439993){\circle*{0.1}}
\put(420,275){\circle*{0.1}}
\put(420,4.95658439993){\circle*{0.1}}
\put(421,275){\circle*{0.1}}
\put(421,4.95658439993){\circle*{0.1}}
\put(422,270){\circle*{0.1}}
\put(422,4.7736212843){\circle*{0.1}}
\put(423,275){\circle*{0.1}}
\put(423,4.95658439993){\circle*{0.1}}
\put(424,275){\circle*{0.1}}
\put(424,4.95658439993){\circle*{0.1}}
\put(425,275){\circle*{0.1}}
\put(425,4.95658439993){\circle*{0.1}}
\put(426,275){\circle*{0.1}}
\put(426,4.95658439993){\circle*{0.1}}
\put(427,275){\circle*{0.1}}
\put(427,4.95658439993){\circle*{0.1}}
\put(428,275){\circle*{0.1}}
\put(428,4.95658439993){\circle*{0.1}}
\put(429,280){\circle*{0.1}}
\put(429,5.14893425389){\circle*{0.1}}
\put(430,280){\circle*{0.1}}
\put(430,5.14893425389){\circle*{0.1}}
\put(431,280){\circle*{0.1}}
\put(431,5.14893425389){\circle*{0.1}}
\put(432,280){\circle*{0.1}}
\put(432,5.14893425389){\circle*{0.1}}
\put(433,280){\circle*{0.1}}
\put(433,5.14893425389){\circle*{0.1}}
\put(434,275){\circle*{0.1}}
\put(434,4.95658439993){\circle*{0.1}}
\put(435,275){\circle*{0.1}}
\put(435,4.95658439993){\circle*{0.1}}
\put(436,275){\circle*{0.1}}
\put(436,4.95658439993){\circle*{0.1}}
\put(437,275){\circle*{0.1}}
\put(437,4.95658439993){\circle*{0.1}}
\put(438,270){\circle*{0.1}}
\put(438,4.7736212843){\circle*{0.1}}
\put(439,270){\circle*{0.1}}
\put(439,4.7736212843){\circle*{0.1}}
\put(440,270){\circle*{0.1}}
\put(440,4.7736212843){\circle*{0.1}}
\put(441,270){\circle*{0.1}}
\put(441,4.7736212843){\circle*{0.1}}
\put(442,270){\circle*{0.1}}
\put(442,4.7736212843){\circle*{0.1}}
\put(443,270){\circle*{0.1}}
\put(443,4.7736212843){\circle*{0.1}}
\put(444,265){\circle*{0.1}}
\put(444,4.59917122567){\circle*{0.1}}
\put(445,265){\circle*{0.1}}
\put(445,4.59917122567){\circle*{0.1}}
\put(446,265){\circle*{0.1}}
\put(446,4.59917122567){\circle*{0.1}}
\put(447,265){\circle*{0.1}}
\put(447,4.59917122567){\circle*{0.1}}
\put(448,265){\circle*{0.1}}
\put(448,4.59917122567){\circle*{0.1}}
\put(449,265){\circle*{0.1}}
\put(449,4.59917122567){\circle*{0.1}}
\put(450,265){\circle*{0.1}}
\put(450,4.59917122567){\circle*{0.1}}
\put(451,265){\circle*{0.1}}
\put(451,4.59917122567){\circle*{0.1}}
\put(452,265){\circle*{0.1}}
\put(452,4.59917122567){\circle*{0.1}}
\put(453,265){\circle*{0.1}}
\put(453,4.59917122567){\circle*{0.1}}
\put(454,270){\circle*{0.1}}
\put(454,4.7736212843){\circle*{0.1}}
\put(455,270){\circle*{0.1}}
\put(455,4.7736212843){\circle*{0.1}}
\put(456,270){\circle*{0.1}}
\put(456,4.7736212843){\circle*{0.1}}
\put(457,270){\circle*{0.1}}
\put(457,4.7736212843){\circle*{0.1}}
\put(458,270){\circle*{0.1}}
\put(458,4.7736212843){\circle*{0.1}}
\put(459,270){\circle*{0.1}}
\put(459,4.7736212843){\circle*{0.1}}
\put(460,270){\circle*{0.1}}
\put(460,4.7736212843){\circle*{0.1}}
\put(461,270){\circle*{0.1}}
\put(461,4.7736212843){\circle*{0.1}}
\put(462,270){\circle*{0.1}}
\put(462,4.7736212843){\circle*{0.1}}
\put(463,270){\circle*{0.1}}
\put(463,4.7736212843){\circle*{0.1}}
\put(464,275){\circle*{0.1}}
\put(464,4.95658439993){\circle*{0.1}}
\put(465,275){\circle*{0.1}}
\put(465,4.95658439993){\circle*{0.1}}
\put(466,270){\circle*{0.1}}
\put(466,4.7736212843){\circle*{0.1}}
\put(467,270){\circle*{0.1}}
\put(467,4.7736212843){\circle*{0.1}}
\put(468,270){\circle*{0.1}}
\put(468,4.7736212843){\circle*{0.1}}
\put(469,270){\circle*{0.1}}
\put(469,4.7736212843){\circle*{0.1}}
\put(470,270){\circle*{0.1}}
\put(470,4.7736212843){\circle*{0.1}}
\put(471,270){\circle*{0.1}}
\put(471,4.7736212843){\circle*{0.1}}
\put(472,270){\circle*{0.1}}
\put(472,4.7736212843){\circle*{0.1}}
\put(473,270){\circle*{0.1}}
\put(473,4.7736212843){\circle*{0.1}}
\put(474,270){\circle*{0.1}}
\put(474,4.7736212843){\circle*{0.1}}
\put(475,270){\circle*{0.1}}
\put(475,4.7736212843){\circle*{0.1}}
\put(476,270){\circle*{0.1}}
\put(476,4.7736212843){\circle*{0.1}}
\put(477,270){\circle*{0.1}}
\put(477,4.7736212843){\circle*{0.1}}
\put(478,270){\circle*{0.1}}
\put(478,4.7736212843){\circle*{0.1}}
\put(479,270){\circle*{0.1}}
\put(479,4.7736212843){\circle*{0.1}}
\put(480,270){\circle*{0.1}}
\put(480,4.7736212843){\circle*{0.1}}
\put(481,275){\circle*{0.1}}
\put(481,4.95658439993){\circle*{0.1}}
\put(482,275){\circle*{0.1}}
\put(482,4.95658439993){\circle*{0.1}}
\put(483,280){\circle*{0.1}}
\put(483,5.14893425389){\circle*{0.1}}
\put(484,280){\circle*{0.1}}
\put(484,5.14893425389){\circle*{0.1}}
\put(485,280){\circle*{0.1}}
\put(485,5.14893425389){\circle*{0.1}}
\put(486,280){\circle*{0.1}}
\put(486,5.14893425389){\circle*{0.1}}
\put(487,280){\circle*{0.1}}
\put(487,5.14893425389){\circle*{0.1}}
\put(488,285){\circle*{0.1}}
\put(488,5.35168633365){\circle*{0.1}}
\put(489,285){\circle*{0.1}}
\put(489,5.35168633365){\circle*{0.1}}
\put(490,285){\circle*{0.1}}
\put(490,5.35168633365){\circle*{0.1}}
\put(491,285){\circle*{0.1}}
\put(491,5.35168633365){\circle*{0.1}}
\put(492,285){\circle*{0.1}}
\put(492,5.35168633365){\circle*{0.1}}
\put(493,285){\circle*{0.1}}
\put(493,5.35168633365){\circle*{0.1}}
\put(494,285){\circle*{0.1}}
\put(494,5.35168633365){\circle*{0.1}}
\put(495,285){\circle*{0.1}}
\put(495,5.35168633365){\circle*{0.1}}
\put(496,285){\circle*{0.1}}
\put(496,5.35168633365){\circle*{0.1}}
\put(497,285){\circle*{0.1}}
\put(497,5.35168633365){\circle*{0.1}}
\put(498,285){\circle*{0.1}}
\put(498,5.35168633365){\circle*{0.1}}
\put(499,285){\circle*{0.1}}
\put(499,5.35168633365){\circle*{0.1}}
\put(500,285){\circle*{0.1}}
\put(500,5.35168633365){\circle*{0.1}}
\put(501,285){\circle*{0.1}}
\put(501,5.35168633365){\circle*{0.1}}
\put(502,285){\circle*{0.1}}
\put(502,5.35168633365){\circle*{0.1}}
\put(503,295){\circle*{0.1}}
\put(503,5.79337271736){\circle*{0.1}}
\put(504,295){\circle*{0.1}}
\put(504,5.79337271736){\circle*{0.1}}
\put(505,295){\circle*{0.1}}
\put(505,5.79337271736){\circle*{0.1}}
\put(506,295){\circle*{0.1}}
\put(506,5.79337271736){\circle*{0.1}}
\put(507,295){\circle*{0.1}}
\put(507,5.79337271736){\circle*{0.1}}
\put(508,295){\circle*{0.1}}
\put(508,5.79337271736){\circle*{0.1}}
\put(509,295){\circle*{0.1}}
\put(509,5.79337271736){\circle*{0.1}}
\put(510,295){\circle*{0.1}}
\put(510,5.79337271736){\circle*{0.1}}
\put(511,295){\circle*{0.1}}
\put(511,5.79337271736){\circle*{0.1}}
\put(512,300){\circle*{0.1}}
\put(512,6.03539217163){\circle*{0.1}}
\put(513,300){\circle*{0.1}}
\put(513,6.03539217163){\circle*{0.1}}
\put(514,295){\circle*{0.1}}
\put(514,5.79337271736){\circle*{0.1}}
\put(515,295){\circle*{0.1}}
\put(515,5.79337271736){\circle*{0.1}}
\put(516,295){\circle*{0.1}}
\put(516,5.79337271736){\circle*{0.1}}
\put(517,295){\circle*{0.1}}
\put(517,5.79337271736){\circle*{0.1}}
\put(518,295){\circle*{0.1}}
\put(518,5.79337271736){\circle*{0.1}}
\put(519,295){\circle*{0.1}}
\put(519,5.79337271736){\circle*{0.1}}
\put(520,290){\circle*{0.1}}
\put(520,5.56603038555){\circle*{0.1}}
\put(521,290){\circle*{0.1}}
\put(521,5.56603038555){\circle*{0.1}}
\put(522,290){\circle*{0.1}}
\put(522,5.56603038555){\circle*{0.1}}
\put(523,290){\circle*{0.1}}
\put(523,5.56603038555){\circle*{0.1}}
\put(524,290){\circle*{0.1}}
\put(524,5.56603038555){\circle*{0.1}}
\put(525,295){\circle*{0.1}}
\put(525,5.79337271736){\circle*{0.1}}
\put(526,295){\circle*{0.1}}
\put(526,5.79337271736){\circle*{0.1}}
\put(527,295){\circle*{0.1}}
\put(527,5.79337271736){\circle*{0.1}}
\put(528,295){\circle*{0.1}}
\put(528,5.79337271736){\circle*{0.1}}
\put(529,295){\circle*{0.1}}
\put(529,5.79337271736){\circle*{0.1}}
\put(530,295){\circle*{0.1}}
\put(530,5.79337271736){\circle*{0.1}}
\put(531,295){\circle*{0.1}}
\put(531,5.79337271736){\circle*{0.1}}
\put(532,295){\circle*{0.1}}
\put(532,5.79337271736){\circle*{0.1}}
\put(533,295){\circle*{0.1}}
\put(533,5.79337271736){\circle*{0.1}}
\put(534,295){\circle*{0.1}}
\put(534,5.79337271736){\circle*{0.1}}
\put(535,295){\circle*{0.1}}
\put(535,5.79337271736){\circle*{0.1}}
\put(536,290){\circle*{0.1}}
\put(536,5.56603038555){\circle*{0.1}}
\put(537,290){\circle*{0.1}}
\put(537,5.56603038555){\circle*{0.1}}
\put(538,295){\circle*{0.1}}
\put(538,5.79337271736){\circle*{0.1}}
\put(539,295){\circle*{0.1}}
\put(539,5.79337271736){\circle*{0.1}}
\put(540,295){\circle*{0.1}}
\put(540,5.79337271736){\circle*{0.1}}
\put(541,295){\circle*{0.1}}
\put(541,5.79337271736){\circle*{0.1}}
\put(542,295){\circle*{0.1}}
\put(542,5.79337271736){\circle*{0.1}}
\put(543,300){\circle*{0.1}}
\put(543,6.03539217163){\circle*{0.1}}
\put(544,300){\circle*{0.1}}
\put(544,6.03539217163){\circle*{0.1}}
\put(545,300){\circle*{0.1}}
\put(545,6.03539217163){\circle*{0.1}}
\put(546,300){\circle*{0.1}}
\put(546,6.03539217163){\circle*{0.1}}
\put(547,300){\circle*{0.1}}
\put(547,6.03539217163){\circle*{0.1}}
\put(548,300){\circle*{0.1}}
\put(548,6.03539217163){\circle*{0.1}}
\put(549,300){\circle*{0.1}}
\put(549,6.03539217163){\circle*{0.1}}
\put(550,300){\circle*{0.1}}
\put(550,6.03539217163){\circle*{0.1}}
\put(551,300){\circle*{0.1}}
\put(551,6.03539217163){\circle*{0.1}}
\put(552,300){\circle*{0.1}}
\put(552,6.03539217163){\circle*{0.1}}
\put(553,300){\circle*{0.1}}
\put(553,6.03539217163){\circle*{0.1}}
\put(554,305){\circle*{0.1}}
\put(554,6.2941154397){\circle*{0.1}}
\put(555,305){\circle*{0.1}}
\put(555,6.2941154397){\circle*{0.1}}
\put(556,305){\circle*{0.1}}
\put(556,6.2941154397){\circle*{0.1}}
\put(557,305){\circle*{0.1}}
\put(557,6.2941154397){\circle*{0.1}}
\put(558,305){\circle*{0.1}}
\put(558,6.2941154397){\circle*{0.1}}
\put(559,305){\circle*{0.1}}
\put(559,6.2941154397){\circle*{0.1}}
\put(560,305){\circle*{0.1}}
\put(560,6.2941154397){\circle*{0.1}}
\put(561,305){\circle*{0.1}}
\put(561,6.2941154397){\circle*{0.1}}
\put(562,305){\circle*{0.1}}
\put(562,6.2941154397){\circle*{0.1}}
\put(563,305){\circle*{0.1}}
\put(563,6.2941154397){\circle*{0.1}}
\put(564,305){\circle*{0.1}}
\put(564,6.2941154397){\circle*{0.1}}
\put(565,305){\circle*{0.1}}
\put(565,6.2941154397){\circle*{0.1}}
\put(566,305){\circle*{0.1}}
\put(566,6.2941154397){\circle*{0.1}}
\put(567,305){\circle*{0.1}}
\put(567,6.2941154397){\circle*{0.1}}
\put(568,305){\circle*{0.1}}
\put(568,6.2941154397){\circle*{0.1}}
\put(569,305){\circle*{0.1}}
\put(569,6.2941154397){\circle*{0.1}}
\put(570,305){\circle*{0.1}}
\put(570,6.2941154397){\circle*{0.1}}
\put(571,305){\circle*{0.1}}
\put(571,6.2941154397){\circle*{0.1}}
\put(572,305){\circle*{0.1}}
\put(572,6.2941154397){\circle*{0.1}}
\put(573,300){\circle*{0.1}}
\put(573,6.03539217163){\circle*{0.1}}
\put(574,300){\circle*{0.1}}
\put(574,6.03539217163){\circle*{0.1}}
\put(575,300){\circle*{0.1}}
\put(575,6.03539217163){\circle*{0.1}}
\put(576,300){\circle*{0.1}}
\put(576,6.03539217163){\circle*{0.1}}
\put(577,300){\circle*{0.1}}
\put(577,6.03539217163){\circle*{0.1}}
\put(578,300){\circle*{0.1}}
\put(578,6.03539217163){\circle*{0.1}}
\put(579,300){\circle*{0.1}}
\put(579,6.03539217163){\circle*{0.1}}
\put(580,300){\circle*{0.1}}
\put(580,6.03539217163){\circle*{0.1}}
\put(581,300){\circle*{0.1}}
\put(581,6.03539217163){\circle*{0.1}}
\put(582,300){\circle*{0.1}}
\put(582,6.03539217163){\circle*{0.1}}
\put(583,300){\circle*{0.1}}
\put(583,6.03539217163){\circle*{0.1}}
\put(584,300){\circle*{0.1}}
\put(584,6.03539217163){\circle*{0.1}}
\put(585,300){\circle*{0.1}}
\put(585,6.03539217163){\circle*{0.1}}
\put(586,300){\circle*{0.1}}
\put(586,6.03539217163){\circle*{0.1}}
\put(587,300){\circle*{0.1}}
\put(587,6.03539217163){\circle*{0.1}}
\put(588,305){\circle*{0.1}}
\put(588,6.2941154397){\circle*{0.1}}
\put(589,305){\circle*{0.1}}
\put(589,6.2941154397){\circle*{0.1}}
\put(590,305){\circle*{0.1}}
\put(590,6.2941154397){\circle*{0.1}}
\put(591,305){\circle*{0.1}}
\put(591,6.2941154397){\circle*{0.1}}
\put(592,305){\circle*{0.1}}
\put(592,6.2941154397){\circle*{0.1}}
\put(593,305){\circle*{0.1}}
\put(593,6.2941154397){\circle*{0.1}}
\put(594,305){\circle*{0.1}}
\put(594,6.2941154397){\circle*{0.1}}
\put(595,305){\circle*{0.1}}
\put(595,6.2941154397){\circle*{0.1}}
\put(596,305){\circle*{0.1}}
\put(596,6.2941154397){\circle*{0.1}}
\put(597,305){\circle*{0.1}}
\put(597,6.2941154397){\circle*{0.1}}
\put(598,305){\circle*{0.1}}
\put(598,6.2941154397){\circle*{0.1}}
\put(599,305){\circle*{0.1}}
\put(599,6.2941154397){\circle*{0.1}}
\put(600,305){\circle*{0.1}}
\put(600,6.2941154397){\circle*{0.1}}
\put(601,305){\circle*{0.1}}
\put(601,6.2941154397){\circle*{0.1}}
\put(602,305){\circle*{0.1}}
\put(602,6.2941154397){\circle*{0.1}}
\put(603,305){\circle*{0.1}}
\put(603,6.2941154397){\circle*{0.1}}
\put(604,305){\circle*{0.1}}
\put(604,6.2941154397){\circle*{0.1}}
\put(605,305){\circle*{0.1}}
\put(605,6.2941154397){\circle*{0.1}}
\put(606,310){\circle*{0.1}}
\put(606,6.57202033528){\circle*{0.1}}
\put(607,310){\circle*{0.1}}
\put(607,6.57202033528){\circle*{0.1}}
\put(608,310){\circle*{0.1}}
\put(608,6.57202033528){\circle*{0.1}}
\put(609,310){\circle*{0.1}}
\put(609,6.57202033528){\circle*{0.1}}
\put(610,310){\circle*{0.1}}
\put(610,6.57202033528){\circle*{0.1}}
\put(611,310){\circle*{0.1}}
\put(611,6.57202033528){\circle*{0.1}}
\put(612,310){\circle*{0.1}}
\put(612,6.57202033528){\circle*{0.1}}
\put(613,310){\circle*{0.1}}
\put(613,6.57202033528){\circle*{0.1}}
\put(614,310){\circle*{0.1}}
\put(614,6.57202033528){\circle*{0.1}}
\put(615,310){\circle*{0.1}}
\put(615,6.57202033528){\circle*{0.1}}
\put(616,310){\circle*{0.1}}
\put(616,6.57202033528){\circle*{0.1}}
\put(617,310){\circle*{0.1}}
\put(617,6.57202033528){\circle*{0.1}}
\put(618,310){\circle*{0.1}}
\put(618,6.57202033528){\circle*{0.1}}
\put(619,310){\circle*{0.1}}
\put(619,6.57202033528){\circle*{0.1}}
\put(620,310){\circle*{0.1}}
\put(620,6.57202033528){\circle*{0.1}}
\put(621,315){\circle*{0.1}}
\put(621,6.87218048906){\circle*{0.1}}
\put(622,315){\circle*{0.1}}
\put(622,6.87218048906){\circle*{0.1}}
\put(623,315){\circle*{0.1}}
\put(623,6.87218048906){\circle*{0.1}}
\put(624,315){\circle*{0.1}}
\put(624,6.87218048906){\circle*{0.1}}
\put(625,310){\circle*{0.1}}
\put(625,6.57202033528){\circle*{0.1}}
\put(626,310){\circle*{0.1}}
\put(626,6.57202033528){\circle*{0.1}}
\put(627,310){\circle*{0.1}}
\put(627,6.57202033528){\circle*{0.1}}
\put(628,310){\circle*{0.1}}
\put(628,6.57202033528){\circle*{0.1}}
\put(629,310){\circle*{0.1}}
\put(629,6.57202033528){\circle*{0.1}}
\put(630,305){\circle*{0.1}}
\put(630,6.2941154397){\circle*{0.1}}
\put(631,310){\circle*{0.1}}
\put(631,6.57202033528){\circle*{0.1}}
\put(632,305){\circle*{0.1}}
\put(632,6.2941154397){\circle*{0.1}}
\put(633,305){\circle*{0.1}}
\put(633,6.2941154397){\circle*{0.1}}
\put(634,305){\circle*{0.1}}
\put(634,6.2941154397){\circle*{0.1}}
\put(635,305){\circle*{0.1}}
\put(635,6.2941154397){\circle*{0.1}}
\put(636,305){\circle*{0.1}}
\put(636,6.2941154397){\circle*{0.1}}
\put(637,305){\circle*{0.1}}
\put(637,6.2941154397){\circle*{0.1}}
\put(638,305){\circle*{0.1}}
\put(638,6.2941154397){\circle*{0.1}}
\put(639,300){\circle*{0.1}}
\put(639,6.03539217163){\circle*{0.1}}
\put(640,300){\circle*{0.1}}
\put(640,6.03539217163){\circle*{0.1}}
\put(641,300){\circle*{0.1}}
\put(641,6.03539217163){\circle*{0.1}}
\put(642,300){\circle*{0.1}}
\put(642,6.03539217163){\circle*{0.1}}
\put(643,300){\circle*{0.1}}
\put(643,6.03539217163){\circle*{0.1}}
\put(644,300){\circle*{0.1}}
\put(644,6.03539217163){\circle*{0.1}}
\put(645,300){\circle*{0.1}}
\put(645,6.03539217163){\circle*{0.1}}
\put(646,300){\circle*{0.1}}
\put(646,6.03539217163){\circle*{0.1}}
\put(647,300){\circle*{0.1}}
\put(647,6.03539217163){\circle*{0.1}}
\put(648,300){\circle*{0.1}}
\put(648,6.03539217163){\circle*{0.1}}
\put(649,295){\circle*{0.1}}
\put(649,5.79337271736){\circle*{0.1}}
\put(650,295){\circle*{0.1}}
\put(650,5.79337271736){\circle*{0.1}}
\put(651,295){\circle*{0.1}}
\put(651,5.79337271736){\circle*{0.1}}
\put(652,295){\circle*{0.1}}
\put(652,5.79337271736){\circle*{0.1}}
\put(653,295){\circle*{0.1}}
\put(653,5.79337271736){\circle*{0.1}}
\put(654,295){\circle*{0.1}}
\put(654,5.79337271736){\circle*{0.1}}
\put(655,295){\circle*{0.1}}
\put(655,5.79337271736){\circle*{0.1}}
\put(656,295){\circle*{0.1}}
\put(656,5.79337271736){\circle*{0.1}}
\put(657,295){\circle*{0.1}}
\put(657,5.79337271736){\circle*{0.1}}
\put(658,295){\circle*{0.1}}
\put(658,5.79337271736){\circle*{0.1}}
\put(659,295){\circle*{0.1}}
\put(659,5.79337271736){\circle*{0.1}}
\put(660,295){\circle*{0.1}}
\put(660,5.79337271736){\circle*{0.1}}
\put(661,295){\circle*{0.1}}
\put(661,5.79337271736){\circle*{0.1}}
\put(662,295){\circle*{0.1}}
\put(662,5.79337271736){\circle*{0.1}}
\put(663,300){\circle*{0.1}}
\put(663,6.03539217163){\circle*{0.1}}
\put(664,300){\circle*{0.1}}
\put(664,6.03539217163){\circle*{0.1}}
\put(665,300){\circle*{0.1}}
\put(665,6.03539217163){\circle*{0.1}}
\put(666,305){\circle*{0.1}}
\put(666,6.2941154397){\circle*{0.1}}
\put(667,305){\circle*{0.1}}
\put(667,6.2941154397){\circle*{0.1}}
\put(668,305){\circle*{0.1}}
\put(668,6.2941154397){\circle*{0.1}}
\put(669,305){\circle*{0.1}}
\put(669,6.2941154397){\circle*{0.1}}
\put(670,305){\circle*{0.1}}
\put(670,6.2941154397){\circle*{0.1}}
\put(671,305){\circle*{0.1}}
\put(671,6.2941154397){\circle*{0.1}}
\put(672,305){\circle*{0.1}}
\put(672,6.2941154397){\circle*{0.1}}
\put(673,310){\circle*{0.1}}
\put(673,6.57202033528){\circle*{0.1}}
\put(674,310){\circle*{0.1}}
\put(674,6.57202033528){\circle*{0.1}}
\put(675,310){\circle*{0.1}}
\put(675,6.57202033528){\circle*{0.1}}
\put(676,310){\circle*{0.1}}
\put(676,6.57202033528){\circle*{0.1}}
\put(677,310){\circle*{0.1}}
\put(677,6.57202033528){\circle*{0.1}}
\put(678,315){\circle*{0.1}}
\put(678,6.87218048906){\circle*{0.1}}
\put(679,315){\circle*{0.1}}
\put(679,6.87218048906){\circle*{0.1}}
\put(680,315){\circle*{0.1}}
\put(680,6.87218048906){\circle*{0.1}}
\put(681,320){\circle*{0.1}}
\put(681,7.19847315277){\circle*{0.1}}
\put(682,320){\circle*{0.1}}
\put(682,7.19847315277){\circle*{0.1}}
\put(683,320){\circle*{0.1}}
\put(683,7.19847315277){\circle*{0.1}}
\put(684,315){\circle*{0.1}}
\put(684,6.87218048906){\circle*{0.1}}
\put(685,315){\circle*{0.1}}
\put(685,6.87218048906){\circle*{0.1}}
\put(686,315){\circle*{0.1}}
\put(686,6.87218048906){\circle*{0.1}}
\put(687,320){\circle*{0.1}}
\put(687,7.19847315277){\circle*{0.1}}
\put(688,320){\circle*{0.1}}
\put(688,7.19847315277){\circle*{0.1}}
\put(689,320){\circle*{0.1}}
\put(689,7.19847315277){\circle*{0.1}}
\put(690,320){\circle*{0.1}}
\put(690,7.19847315277){\circle*{0.1}}
\put(691,320){\circle*{0.1}}
\put(691,7.19847315277){\circle*{0.1}}
\put(692,320){\circle*{0.1}}
\put(692,7.19847315277){\circle*{0.1}}
\put(693,320){\circle*{0.1}}
\put(693,7.19847315277){\circle*{0.1}}
\put(694,320){\circle*{0.1}}
\put(694,7.19847315277){\circle*{0.1}}
\put(695,320){\circle*{0.1}}
\put(695,7.19847315277){\circle*{0.1}}
\put(696,320){\circle*{0.1}}
\put(696,7.19847315277){\circle*{0.1}}
\put(697,320){\circle*{0.1}}
\put(697,7.19847315277){\circle*{0.1}}
\put(698,320){\circle*{0.1}}
\put(698,7.19847315277){\circle*{0.1}}
\put(699,320){\circle*{0.1}}
\put(699,7.19847315277){\circle*{0.1}}
\put(700,315){\circle*{0.1}}
\put(700,6.87218048906){\circle*{0.1}}
\put(701,315){\circle*{0.1}}
\put(701,6.87218048906){\circle*{0.1}}
\put(702,315){\circle*{0.1}}
\put(702,6.87218048906){\circle*{0.1}}
\put(703,315){\circle*{0.1}}
\put(703,6.87218048906){\circle*{0.1}}
\put(704,315){\circle*{0.1}}
\put(704,6.87218048906){\circle*{0.1}}
\put(705,315){\circle*{0.1}}
\put(705,6.87218048906){\circle*{0.1}}
\put(706,315){\circle*{0.1}}
\put(706,6.87218048906){\circle*{0.1}}
\put(707,315){\circle*{0.1}}
\put(707,6.87218048906){\circle*{0.1}}
\put(708,310){\circle*{0.1}}
\put(708,6.57202033528){\circle*{0.1}}
\put(709,310){\circle*{0.1}}
\put(709,6.57202033528){\circle*{0.1}}
\put(710,310){\circle*{0.1}}
\put(710,6.57202033528){\circle*{0.1}}
\put(711,315){\circle*{0.1}}
\put(711,6.87218048906){\circle*{0.1}}
\put(712,315){\circle*{0.1}}
\put(712,6.87218048906){\circle*{0.1}}
\put(713,315){\circle*{0.1}}
\put(713,6.87218048906){\circle*{0.1}}
\put(714,315){\circle*{0.1}}
\put(714,6.87218048906){\circle*{0.1}}
\put(715,315){\circle*{0.1}}
\put(715,6.87218048906){\circle*{0.1}}
\put(716,315){\circle*{0.1}}
\put(716,6.87218048906){\circle*{0.1}}
\put(717,315){\circle*{0.1}}
\put(717,6.87218048906){\circle*{0.1}}
\put(718,315){\circle*{0.1}}
\put(718,6.87218048906){\circle*{0.1}}
\put(719,315){\circle*{0.1}}
\put(719,6.87218048906){\circle*{0.1}}
\put(720,315){\circle*{0.1}}
\put(720,6.87218048906){\circle*{0.1}}
\put(721,315){\circle*{0.1}}
\put(721,6.87218048906){\circle*{0.1}}
\put(722,315){\circle*{0.1}}
\put(722,6.87218048906){\circle*{0.1}}
\put(723,315){\circle*{0.1}}
\put(723,6.87218048906){\circle*{0.1}}
\put(724,315){\circle*{0.1}}
\put(724,6.87218048906){\circle*{0.1}}
\put(725,315){\circle*{0.1}}
\put(725,6.87218048906){\circle*{0.1}}
\put(726,315){\circle*{0.1}}
\put(726,6.87218048906){\circle*{0.1}}
\put(727,315){\circle*{0.1}}
\put(727,6.87218048906){\circle*{0.1}}
\put(728,315){\circle*{0.1}}
\put(728,6.87218048906){\circle*{0.1}}
\put(729,315){\circle*{0.1}}
\put(729,6.87218048906){\circle*{0.1}}
\put(730,315){\circle*{0.1}}
\put(730,6.87218048906){\circle*{0.1}}
\put(731,315){\circle*{0.1}}
\put(731,6.87218048906){\circle*{0.1}}
\put(732,315){\circle*{0.1}}
\put(732,6.87218048906){\circle*{0.1}}
\put(733,310){\circle*{0.1}}
\put(733,6.57202033528){\circle*{0.1}}
\put(734,310){\circle*{0.1}}
\put(734,6.57202033528){\circle*{0.1}}
\put(735,315){\circle*{0.1}}
\put(735,6.87218048906){\circle*{0.1}}
\put(736,315){\circle*{0.1}}
\put(736,6.87218048906){\circle*{0.1}}
\put(737,315){\circle*{0.1}}
\put(737,6.87218048906){\circle*{0.1}}
\put(738,315){\circle*{0.1}}
\put(738,6.87218048906){\circle*{0.1}}
\put(739,315){\circle*{0.1}}
\put(739,6.87218048906){\circle*{0.1}}
\put(740,315){\circle*{0.1}}
\put(740,6.87218048906){\circle*{0.1}}
\put(741,315){\circle*{0.1}}
\put(741,6.87218048906){\circle*{0.1}}
\put(742,315){\circle*{0.1}}
\put(742,6.87218048906){\circle*{0.1}}
\put(743,315){\circle*{0.1}}
\put(743,6.87218048906){\circle*{0.1}}
\put(744,315){\circle*{0.1}}
\put(744,6.87218048906){\circle*{0.1}}
\put(745,315){\circle*{0.1}}
\put(745,6.87218048906){\circle*{0.1}}
\put(746,315){\circle*{0.1}}
\put(746,6.87218048906){\circle*{0.1}}
\put(747,315){\circle*{0.1}}
\put(747,6.87218048906){\circle*{0.1}}
\put(748,315){\circle*{0.1}}
\put(748,6.87218048906){\circle*{0.1}}
\put(749,315){\circle*{0.1}}
\put(749,6.87218048906){\circle*{0.1}}
\put(750,315){\circle*{0.1}}
\put(750,6.87218048906){\circle*{0.1}}
\put(751,315){\circle*{0.1}}
\put(751,6.87218048906){\circle*{0.1}}
\put(752,315){\circle*{0.1}}
\put(752,6.87218048906){\circle*{0.1}}
\put(753,315){\circle*{0.1}}
\put(753,6.87218048906){\circle*{0.1}}
\put(754,315){\circle*{0.1}}
\put(754,6.87218048906){\circle*{0.1}}
\put(755,315){\circle*{0.1}}
\put(755,6.87218048906){\circle*{0.1}}
\put(756,315){\circle*{0.1}}
\put(756,6.87218048906){\circle*{0.1}}
\put(757,315){\circle*{0.1}}
\put(757,6.87218048906){\circle*{0.1}}
\put(758,315){\circle*{0.1}}
\put(758,6.87218048906){\circle*{0.1}}
\put(759,315){\circle*{0.1}}
\put(759,6.87218048906){\circle*{0.1}}
\put(760,315){\circle*{0.1}}
\put(760,6.87218048906){\circle*{0.1}}
\put(761,315){\circle*{0.1}}
\put(761,6.87218048906){\circle*{0.1}}
\put(762,315){\circle*{0.1}}
\put(762,6.87218048906){\circle*{0.1}}
\put(763,315){\circle*{0.1}}
\put(763,6.87218048906){\circle*{0.1}}
\put(764,315){\circle*{0.1}}
\put(764,6.87218048906){\circle*{0.1}}
\put(765,315){\circle*{0.1}}
\put(765,6.87218048906){\circle*{0.1}}
\put(766,315){\circle*{0.1}}
\put(766,6.87218048906){\circle*{0.1}}
\put(767,315){\circle*{0.1}}
\put(767,6.87218048906){\circle*{0.1}}
\put(768,315){\circle*{0.1}}
\put(768,6.87218048906){\circle*{0.1}}
\put(769,320){\circle*{0.1}}
\put(769,7.19847315277){\circle*{0.1}}
\put(770,320){\circle*{0.1}}
\put(770,7.19847315277){\circle*{0.1}}
\put(771,320){\circle*{0.1}}
\put(771,7.19847315277){\circle*{0.1}}
\put(772,320){\circle*{0.1}}
\put(772,7.19847315277){\circle*{0.1}}
\put(773,320){\circle*{0.1}}
\put(773,7.19847315277){\circle*{0.1}}
\put(774,315){\circle*{0.1}}
\put(774,6.87218048906){\circle*{0.1}}
\put(775,315){\circle*{0.1}}
\put(775,6.87218048906){\circle*{0.1}}
\put(776,315){\circle*{0.1}}
\put(776,6.87218048906){\circle*{0.1}}
\put(777,315){\circle*{0.1}}
\put(777,6.87218048906){\circle*{0.1}}
\put(778,315){\circle*{0.1}}
\put(778,6.87218048906){\circle*{0.1}}
\put(779,320){\circle*{0.1}}
\put(779,7.19847315277){\circle*{0.1}}
\put(780,320){\circle*{0.1}}
\put(780,7.19847315277){\circle*{0.1}}
\put(781,320){\circle*{0.1}}
\put(781,7.19847315277){\circle*{0.1}}
\put(782,325){\circle*{0.1}}
\put(782,7.55588632703){\circle*{0.1}}
\put(783,330){\circle*{0.1}}
\put(783,7.95098826075){\circle*{0.1}}
\put(784,330){\circle*{0.1}}
\put(784,7.95098826075){\circle*{0.1}}
\put(785,330){\circle*{0.1}}
\put(785,7.95098826075){\circle*{0.1}}
\put(786,330){\circle*{0.1}}
\put(786,7.95098826075){\circle*{0.1}}
\put(787,330){\circle*{0.1}}
\put(787,7.95098826075){\circle*{0.1}}
\put(788,330){\circle*{0.1}}
\put(788,7.95098826075){\circle*{0.1}}
\put(789,330){\circle*{0.1}}
\put(789,7.95098826075){\circle*{0.1}}
\put(790,330){\circle*{0.1}}
\put(790,7.95098826075){\circle*{0.1}}
\put(791,330){\circle*{0.1}}
\put(791,7.95098826075){\circle*{0.1}}
\put(792,330){\circle*{0.1}}
\put(792,7.95098826075){\circle*{0.1}}
\put(793,330){\circle*{0.1}}
\put(793,7.95098826075){\circle*{0.1}}
\put(794,330){\circle*{0.1}}
\put(794,7.95098826075){\circle*{0.1}}
\put(795,330){\circle*{0.1}}
\put(795,7.95098826075){\circle*{0.1}}
\put(796,335){\circle*{0.1}}
\put(796,8.39267464446){\circle*{0.1}}
\put(797,335){\circle*{0.1}}
\put(797,8.39267464446){\circle*{0.1}}
\put(798,335){\circle*{0.1}}
\put(798,8.39267464446){\circle*{0.1}}
\put(799,335){\circle*{0.1}}
\put(799,8.39267464446){\circle*{0.1}}
\put(800,335){\circle*{0.1}}
\put(800,8.39267464446){\circle*{0.1}}
\put(801,335){\circle*{0.1}}
\put(801,8.39267464446){\circle*{0.1}}
\put(802,335){\circle*{0.1}}
\put(802,8.39267464446){\circle*{0.1}}
\put(803,335){\circle*{0.1}}
\put(803,8.39267464446){\circle*{0.1}}
\put(804,335){\circle*{0.1}}
\put(804,8.39267464446){\circle*{0.1}}
\put(805,340){\circle*{0.1}}
\put(805,8.8934173668){\circle*{0.1}}
\put(806,340){\circle*{0.1}}
\put(806,8.8934173668){\circle*{0.1}}
\put(807,345){\circle*{0.1}}
\put(807,9.47148241616){\circle*{0.1}}
\put(808,345){\circle*{0.1}}
\put(808,9.47148241616){\circle*{0.1}}
\put(809,345){\circle*{0.1}}
\put(809,9.47148241616){\circle*{0.1}}
\put(810,345){\circle*{0.1}}
\put(810,9.47148241616){\circle*{0.1}}
\put(811,345){\circle*{0.1}}
\put(811,9.47148241616){\circle*{0.1}}
\put(812,345){\circle*{0.1}}
\put(812,9.47148241616){\circle*{0.1}}
\put(813,345){\circle*{0.1}}
\put(813,9.47148241616){\circle*{0.1}}
\put(814,345){\circle*{0.1}}
\put(814,9.47148241616){\circle*{0.1}}
\put(815,340){\circle*{0.1}}
\put(815,8.8934173668){\circle*{0.1}}
\put(816,340){\circle*{0.1}}
\put(816,8.8934173668){\circle*{0.1}}
\put(817,340){\circle*{0.1}}
\put(817,8.8934173668){\circle*{0.1}}
\put(818,340){\circle*{0.1}}
\put(818,8.8934173668){\circle*{0.1}}
\put(819,345){\circle*{0.1}}
\put(819,9.47148241616){\circle*{0.1}}
\put(820,345){\circle*{0.1}}
\put(820,9.47148241616){\circle*{0.1}}
\put(821,345){\circle*{0.1}}
\put(821,9.47148241616){\circle*{0.1}}
\put(822,345){\circle*{0.1}}
\put(822,9.47148241616){\circle*{0.1}}
\put(823,345){\circle*{0.1}}
\put(823,9.47148241616){\circle*{0.1}}
\put(824,350){\circle*{0.1}}
\put(824,10.1551882541){\circle*{0.1}}
\put(825,350){\circle*{0.1}}
\put(825,10.1551882541){\circle*{0.1}}
\put(826,345){\circle*{0.1}}
\put(826,9.47148241616){\circle*{0.1}}
\put(827,345){\circle*{0.1}}
\put(827,9.47148241616){\circle*{0.1}}
\put(828,345){\circle*{0.1}}
\put(828,9.47148241616){\circle*{0.1}}
\put(829,345){\circle*{0.1}}
\put(829,9.47148241616){\circle*{0.1}}
\put(830,345){\circle*{0.1}}
\put(830,9.47148241616){\circle*{0.1}}
\put(831,345){\circle*{0.1}}
\put(831,9.47148241616){\circle*{0.1}}
\put(832,345){\circle*{0.1}}
\put(832,9.47148241616){\circle*{0.1}}
\put(833,345){\circle*{0.1}}
\put(833,9.47148241616){\circle*{0.1}}
\put(834,345){\circle*{0.1}}
\put(834,9.47148241616){\circle*{0.1}}
\put(835,345){\circle*{0.1}}
\put(835,9.47148241616){\circle*{0.1}}
\put(836,345){\circle*{0.1}}
\put(836,9.47148241616){\circle*{0.1}}
\put(837,345){\circle*{0.1}}
\put(837,9.47148241616){\circle*{0.1}}
\put(838,340){\circle*{0.1}}
\put(838,8.8934173668){\circle*{0.1}}
\put(839,345){\circle*{0.1}}
\put(839,9.47148241616){\circle*{0.1}}
\put(840,340){\circle*{0.1}}
\put(840,8.8934173668){\circle*{0.1}}
\put(841,345){\circle*{0.1}}
\put(841,9.47148241616){\circle*{0.1}}
\put(842,345){\circle*{0.1}}
\put(842,9.47148241616){\circle*{0.1}}
\put(843,350){\circle*{0.1}}
\put(843,10.1551882541){\circle*{0.1}}
\put(844,350){\circle*{0.1}}
\put(844,10.1551882541){\circle*{0.1}}
\put(845,350){\circle*{0.1}}
\put(845,10.1551882541){\circle*{0.1}}
\put(846,355){\circle*{0.1}}
\put(846,10.9919765716){\circle*{0.1}}
\put(847,350){\circle*{0.1}}
\put(847,10.1551882541){\circle*{0.1}}
\put(848,350){\circle*{0.1}}
\put(848,10.1551882541){\circle*{0.1}}
\put(849,350){\circle*{0.1}}
\put(849,10.1551882541){\circle*{0.1}}
\put(850,350){\circle*{0.1}}
\put(850,10.1551882541){\circle*{0.1}}
\put(851,350){\circle*{0.1}}
\put(851,10.1551882541){\circle*{0.1}}
\put(852,350){\circle*{0.1}}
\put(852,10.1551882541){\circle*{0.1}}
\put(853,350){\circle*{0.1}}
\put(853,10.1551882541){\circle*{0.1}}
\put(854,345){\circle*{0.1}}
\put(854,9.47148241616){\circle*{0.1}}
\put(855,345){\circle*{0.1}}
\put(855,9.47148241616){\circle*{0.1}}
\put(856,345){\circle*{0.1}}
\put(856,9.47148241616){\circle*{0.1}}
\put(857,345){\circle*{0.1}}
\put(857,9.47148241616){\circle*{0.1}}
\put(858,345){\circle*{0.1}}
\put(858,9.47148241616){\circle*{0.1}}
\put(859,345){\circle*{0.1}}
\put(859,9.47148241616){\circle*{0.1}}
\put(860,345){\circle*{0.1}}
\put(860,9.47148241616){\circle*{0.1}}
\put(861,345){\circle*{0.1}}
\put(861,9.47148241616){\circle*{0.1}}
\put(862,345){\circle*{0.1}}
\put(862,9.47148241616){\circle*{0.1}}
\put(863,345){\circle*{0.1}}
\put(863,9.47148241616){\circle*{0.1}}
\put(864,340){\circle*{0.1}}
\put(864,8.8934173668){\circle*{0.1}}
\put(865,340){\circle*{0.1}}
\put(865,8.8934173668){\circle*{0.1}}
\put(866,340){\circle*{0.1}}
\put(866,8.8934173668){\circle*{0.1}}
\put(867,340){\circle*{0.1}}
\put(867,8.8934173668){\circle*{0.1}}
\put(868,340){\circle*{0.1}}
\put(868,8.8934173668){\circle*{0.1}}
\put(869,340){\circle*{0.1}}
\put(869,8.8934173668){\circle*{0.1}}
\put(870,340){\circle*{0.1}}
\put(870,8.8934173668){\circle*{0.1}}
\put(871,340){\circle*{0.1}}
\put(871,8.8934173668){\circle*{0.1}}
\put(872,340){\circle*{0.1}}
\put(872,8.8934173668){\circle*{0.1}}
\put(873,340){\circle*{0.1}}
\put(873,8.8934173668){\circle*{0.1}}
\put(874,340){\circle*{0.1}}
\put(874,8.8934173668){\circle*{0.1}}
\put(875,340){\circle*{0.1}}
\put(875,8.8934173668){\circle*{0.1}}
\put(876,340){\circle*{0.1}}
\put(876,8.8934173668){\circle*{0.1}}
\put(877,340){\circle*{0.1}}
\put(877,8.8934173668){\circle*{0.1}}
\put(878,340){\circle*{0.1}}
\put(878,8.8934173668){\circle*{0.1}}
\put(879,340){\circle*{0.1}}
\put(879,8.8934173668){\circle*{0.1}}
\put(880,340){\circle*{0.1}}
\put(880,8.8934173668){\circle*{0.1}}
\put(881,340){\circle*{0.1}}
\put(881,8.8934173668){\circle*{0.1}}
\put(882,340){\circle*{0.1}}
\put(882,8.8934173668){\circle*{0.1}}
\put(883,340){\circle*{0.1}}
\put(883,8.8934173668){\circle*{0.1}}
\put(884,340){\circle*{0.1}}
\put(884,8.8934173668){\circle*{0.1}}
\put(885,340){\circle*{0.1}}
\put(885,8.8934173668){\circle*{0.1}}
\put(886,340){\circle*{0.1}}
\put(886,8.8934173668){\circle*{0.1}}
\put(887,340){\circle*{0.1}}
\put(887,8.8934173668){\circle*{0.1}}
\put(888,340){\circle*{0.1}}
\put(888,8.8934173668){\circle*{0.1}}
\put(889,340){\circle*{0.1}}
\put(889,8.8934173668){\circle*{0.1}}
\put(890,340){\circle*{0.1}}
\put(890,8.8934173668){\circle*{0.1}}
\put(891,340){\circle*{0.1}}
\put(891,8.8934173668){\circle*{0.1}}
\put(892,340){\circle*{0.1}}
\put(892,8.8934173668){\circle*{0.1}}
\put(893,345){\circle*{0.1}}
\put(893,9.47148241616){\circle*{0.1}}
\put(894,350){\circle*{0.1}}
\put(894,10.1551882541){\circle*{0.1}}
\put(895,350){\circle*{0.1}}
\put(895,10.1551882541){\circle*{0.1}}
\put(896,350){\circle*{0.1}}
\put(896,10.1551882541){\circle*{0.1}}
\put(897,350){\circle*{0.1}}
\put(897,10.1551882541){\circle*{0.1}}
\put(898,350){\circle*{0.1}}
\put(898,10.1551882541){\circle*{0.1}}
\put(899,350){\circle*{0.1}}
\put(899,10.1551882541){\circle*{0.1}}
\put(900,350){\circle*{0.1}}
\put(900,10.1551882541){\circle*{0.1}}
\put(901,345){\circle*{0.1}}
\put(901,9.47148241616){\circle*{0.1}}
\put(902,345){\circle*{0.1}}
\put(902,9.47148241616){\circle*{0.1}}
\put(903,345){\circle*{0.1}}
\put(903,9.47148241616){\circle*{0.1}}
\put(904,340){\circle*{0.1}}
\put(904,8.8934173668){\circle*{0.1}}
\put(905,340){\circle*{0.1}}
\put(905,8.8934173668){\circle*{0.1}}
\put(906,345){\circle*{0.1}}
\put(906,9.47148241616){\circle*{0.1}}
\put(907,345){\circle*{0.1}}
\put(907,9.47148241616){\circle*{0.1}}
\put(908,350){\circle*{0.1}}
\put(908,10.1551882541){\circle*{0.1}}
\put(909,350){\circle*{0.1}}
\put(909,10.1551882541){\circle*{0.1}}
\put(910,350){\circle*{0.1}}
\put(910,10.1551882541){\circle*{0.1}}
\put(911,350){\circle*{0.1}}
\put(911,10.1551882541){\circle*{0.1}}
\put(912,350){\circle*{0.1}}
\put(912,10.1551882541){\circle*{0.1}}
\put(913,350){\circle*{0.1}}
\put(913,10.1551882541){\circle*{0.1}}
\put(914,350){\circle*{0.1}}
\put(914,10.1551882541){\circle*{0.1}}
\put(915,355){\circle*{0.1}}
\put(915,10.9919765716){\circle*{0.1}}
\put(916,355){\circle*{0.1}}
\put(916,10.9919765716){\circle*{0.1}}
\put(917,355){\circle*{0.1}}
\put(917,10.9919765716){\circle*{0.1}}
\put(918,355){\circle*{0.1}}
\put(918,10.9919765716){\circle*{0.1}}
\put(919,355){\circle*{0.1}}
\put(919,10.9919765716){\circle*{0.1}}
\put(920,355){\circle*{0.1}}
\put(920,10.9919765716){\circle*{0.1}}
\put(921,355){\circle*{0.1}}
\put(921,10.9919765716){\circle*{0.1}}
\put(922,355){\circle*{0.1}}
\put(922,10.9919765716){\circle*{0.1}}
\put(923,355){\circle*{0.1}}
\put(923,10.9919765716){\circle*{0.1}}
\put(924,350){\circle*{0.1}}
\put(924,10.1551882541){\circle*{0.1}}
\put(925,350){\circle*{0.1}}
\put(925,10.1551882541){\circle*{0.1}}
\put(926,350){\circle*{0.1}}
\put(926,10.1551882541){\circle*{0.1}}
\put(927,350){\circle*{0.1}}
\put(927,10.1551882541){\circle*{0.1}}
\put(928,350){\circle*{0.1}}
\put(928,10.1551882541){\circle*{0.1}}
\put(929,350){\circle*{0.1}}
\put(929,10.1551882541){\circle*{0.1}}
\put(930,350){\circle*{0.1}}
\put(930,10.1551882541){\circle*{0.1}}
\put(931,350){\circle*{0.1}}
\put(931,10.1551882541){\circle*{0.1}}
\put(932,350){\circle*{0.1}}
\put(932,10.1551882541){\circle*{0.1}}
\put(933,350){\circle*{0.1}}
\put(933,10.1551882541){\circle*{0.1}}
\put(934,355){\circle*{0.1}}
\put(934,10.9919765716){\circle*{0.1}}
\put(935,360){\circle*{0.1}}
\put(935,12.0707843433){\circle*{0.1}}
\put(936,360){\circle*{0.1}}
\put(936,12.0707843433){\circle*{0.1}}
\put(937,360){\circle*{0.1}}
\put(937,12.0707843433){\circle*{0.1}}
\put(938,360){\circle*{0.1}}
\put(938,12.0707843433){\circle*{0.1}}
\put(939,360){\circle*{0.1}}
\put(939,12.0707843433){\circle*{0.1}}
\put(940,360){\circle*{0.1}}
\put(940,12.0707843433){\circle*{0.1}}
\put(941,360){\circle*{0.1}}
\put(941,12.0707843433){\circle*{0.1}}
\put(942,355){\circle*{0.1}}
\put(942,10.9919765716){\circle*{0.1}}
\put(943,355){\circle*{0.1}}
\put(943,10.9919765716){\circle*{0.1}}
\put(944,360){\circle*{0.1}}
\put(944,12.0707843433){\circle*{0.1}}
\put(945,360){\circle*{0.1}}
\put(945,12.0707843433){\circle*{0.1}}
\put(946,360){\circle*{0.1}}
\put(946,12.0707843433){\circle*{0.1}}
\put(947,360){\circle*{0.1}}
\put(947,12.0707843433){\circle*{0.1}}
\put(948,360){\circle*{0.1}}
\put(948,12.0707843433){\circle*{0.1}}
\put(949,360){\circle*{0.1}}
\put(949,12.0707843433){\circle*{0.1}}
\put(950,360){\circle*{0.1}}
\put(950,12.0707843433){\circle*{0.1}}
\put(951,360){\circle*{0.1}}
\put(951,12.0707843433){\circle*{0.1}}
\put(952,360){\circle*{0.1}}
\put(952,12.0707843433){\circle*{0.1}}
\put(953,360){\circle*{0.1}}
\put(953,12.0707843433){\circle*{0.1}}
\put(954,360){\circle*{0.1}}
\put(954,12.0707843433){\circle*{0.1}}
\put(955,360){\circle*{0.1}}
\put(955,12.0707843433){\circle*{0.1}}
\put(956,360){\circle*{0.1}}
\put(956,12.0707843433){\circle*{0.1}}
\put(957,360){\circle*{0.1}}
\put(957,12.0707843433){\circle*{0.1}}
\put(958,360){\circle*{0.1}}
\put(958,12.0707843433){\circle*{0.1}}
\put(959,360){\circle*{0.1}}
\put(959,12.0707843433){\circle*{0.1}}
\put(960,360){\circle*{0.1}}
\put(960,12.0707843433){\circle*{0.1}}
\put(961,360){\circle*{0.1}}
\put(961,12.0707843433){\circle*{0.1}}
\put(962,360){\circle*{0.1}}
\put(962,12.0707843433){\circle*{0.1}}
\put(963,360){\circle*{0.1}}
\put(963,12.0707843433){\circle*{0.1}}
\put(964,360){\circle*{0.1}}
\put(964,12.0707843433){\circle*{0.1}}
\put(965,360){\circle*{0.1}}
\put(965,12.0707843433){\circle*{0.1}}
\put(966,360){\circle*{0.1}}
\put(966,12.0707843433){\circle*{0.1}}
\put(967,360){\circle*{0.1}}
\put(967,12.0707843433){\circle*{0.1}}
\put(968,360){\circle*{0.1}}
\put(968,12.0707843433){\circle*{0.1}}
\put(969,360){\circle*{0.1}}
\put(969,12.0707843433){\circle*{0.1}}
\put(970,360){\circle*{0.1}}
\put(970,12.0707843433){\circle*{0.1}}
\put(971,360){\circle*{0.1}}
\put(971,12.0707843433){\circle*{0.1}}
\put(972,360){\circle*{0.1}}
\put(972,12.0707843433){\circle*{0.1}}
\put(973,360){\circle*{0.1}}
\put(973,12.0707843433){\circle*{0.1}}
\put(974,360){\circle*{0.1}}
\put(974,12.0707843433){\circle*{0.1}}
\put(975,360){\circle*{0.1}}
\put(975,12.0707843433){\circle*{0.1}}
\put(976,360){\circle*{0.1}}
\put(976,12.0707843433){\circle*{0.1}}
\put(977,360){\circle*{0.1}}
\put(977,12.0707843433){\circle*{0.1}}
\put(978,360){\circle*{0.1}}
\put(978,12.0707843433){\circle*{0.1}}
\put(979,360){\circle*{0.1}}
\put(979,12.0707843433){\circle*{0.1}}
\put(980,360){\circle*{0.1}}
\put(980,12.0707843433){\circle*{0.1}}
\put(981,360){\circle*{0.1}}
\put(981,12.0707843433){\circle*{0.1}}
\put(982,360){\circle*{0.1}}
\put(982,12.0707843433){\circle*{0.1}}
\put(983,360){\circle*{0.1}}
\put(983,12.0707843433){\circle*{0.1}}
\put(984,360){\circle*{0.1}}
\put(984,12.0707843433){\circle*{0.1}}
\put(985,360){\circle*{0.1}}
\put(985,12.0707843433){\circle*{0.1}}
\put(986,355){\circle*{0.1}}
\put(986,10.9919765716){\circle*{0.1}}
\put(987,355){\circle*{0.1}}
\put(987,10.9919765716){\circle*{0.1}}
\put(988,355){\circle*{0.1}}
\put(988,10.9919765716){\circle*{0.1}}
\put(989,355){\circle*{0.1}}
\put(989,10.9919765716){\circle*{0.1}}
\put(990,355){\circle*{0.1}}
\put(990,10.9919765716){\circle*{0.1}}
\put(991,355){\circle*{0.1}}
\put(991,10.9919765716){\circle*{0.1}}
\put(992,355){\circle*{0.1}}
\put(992,10.9919765716){\circle*{0.1}}
\put(993,350){\circle*{0.1}}
\put(993,10.1551882541){\circle*{0.1}}
\put(994,350){\circle*{0.1}}
\put(994,10.1551882541){\circle*{0.1}}
\put(995,350){\circle*{0.1}}
\put(995,10.1551882541){\circle*{0.1}}
\put(996,350){\circle*{0.1}}
\put(996,10.1551882541){\circle*{0.1}}
\put(997,350){\circle*{0.1}}
\put(997,10.1551882541){\circle*{0.1}}
\put(998,350){\circle*{0.1}}
\put(998,10.1551882541){\circle*{0.1}}
\put(999,350){\circle*{0.1}}
\put(999,10.1551882541){\circle*{0.1}}
\end{picture}

\end{center}

Auf der x-Achse ist dabei die Zeit, auf der y-Achse die Hamming-Distanz (obere
Kurve) bzw. der geschätzte evolutionäre Abstand (untere Kurve) aufgetragen.
Dieses Beispiel wurde durch den Aufruf mit folgenden Parametern erzeugt:

\texttt{\$ python albi\_ueb02\_a7.py -L 100 -TE 9 -dtE 5 -aE -4 -sE 6}

Dabei fallen mehrere Dinge auf. Einerseits war es uns nicht möglich, ein
aussagekräftiges Diagramm für einen realistischeren Wert von zum Beispiel
$\alpha = 10^{-7}$ zu erhalten, da dann beide Kurven auf Null fielen. Dies
könnte auf Rundungsfehler zurückzuführen sein.

Weiterhin zeigt sich, dass die Kurve der Zeitschätzung relativ linear ansteigt,
währenddessen die Hamming-Distanz anfangs stark ansteigt, dann abflacht und
gegen $75$ konvergiert.

Diejenigen Stellen, an denen die Zeitschätzung Null zeigt, resultieren aus einer
Entscheidung bei der Implementierung. Es kann nämlich vorkommen, dass das
Argument für den Logarithmus in
$- \frac{3}{4} \cdot \log(1 - \frac{4}{3} \cdot p)$
negativ und dafür der Logarithmus nicht definiert ist. In diesem Fall wählen wir
$0$ als Ergebnis der Schätzung:

\begin{lstlisting}[language=python]
def jcest(dist, len):
	""" returns estimated evolutionary distance for proportion dist/len of differences """
	p = float(dist) / float(len)
	critical = 4.0 * p / 3.0
	if critical < 1.0:
		return (-3.0) * math.log(1.0 - critical) / 4.0
	else:
		return 0.0
\end{lstlisting}

\item[c)]

\end{enumerate}

\aufgabe{Markov-Ketten}{60}
\begin{enumerate}
\item $$P = \begin{pmatrix}
0,8 & 0,1 & 0,1\\
0,2 & 0,05 & 0,75\\
0,3 & 0,02 & 0,95
\end{pmatrix}$$

\begin{center}
\setlength{\unitlength}{0.75cm}
\begin{picture}(5.0,5.0)

\put(0.5,1.5){\circle{1}} \put(0.2,1.25){M}
\put(2.5,4.5){\circle{1}} \put(2.25,4.25){F}
\put(4.5,1.5){\circle{1}} \put(4.25,1.25){V}

%\put(2.5,){0,03}
%\put(2.5,){0,1}
%\put(1.5,3.0){0,2}
%\put(1.5,3.0){0,1}
%\put(3.5,3.0){0,02}
%\put(3.5,3.0){0,75}
%\put(0.5,1.5){0,8}
%\put(2.5,4.5){0,05}
%\put(4.5,1.5){0,95}

\end{picture}
\end{center}

\item Für die stationäre Verteilung $\pi$ muss gelten $\pi \cdot P = \pi$.
\begin{eqnarray*}
(\pi_M, \pi_F, \pi_V) \cdot \begin{pmatrix}
0,8 & 0,1 & 0,1\\
0,2 & 0,05 & 0,75\\
0,3 & 0,02 & 0,95
\end{pmatrix}
= & (\pi_M, \pi_F, \pi_V)
& | \quad \text{Falk-Schema}\\
\begin{pmatrix}
0,8 \pi_M & 0,1 \pi_M & 0,1 \pi_M\\
+ 0,2 \pi_F & + 0,05 \pi_F & + 0,75 \pi_F\\
+ 0,3 \pi_V & + 0,02 \pi_V & + 0,95 \pi_V\\
= \pi_M' & = \pi_F' & = \pi_V'
\end{pmatrix}
= & (\pi_M, \pi_F, \pi_V)
& | \quad \square - \pi\\
(\pi_M' - \pi_M, \pi_F' - \pi_F, \pi_V' - \pi_V)
= & 0
\end{eqnarray*}

Aus der letzten Gleichung ergibt sich ein Gleichungssystem, welches z.B. mit
Hilfe des Gauß-Algorithmus gelöst werden kann:

$$\left(\begin{array}{ccc|c}
-0,2	& 0,2	& 0,03	& 0\\
0,1		& -0,95	& 0,02	& 0\\
0,1		& 0,75	& -0,05	& 0
\end{array}\right)$$

$$\pi_M \approx 4,6429 \phi \qquad \pi_F = \phi \qquad \pi_V \approx 24,2857 \phi$$

Es zeigt sich, dass das Gleichungssystem nicht eindeutig lösbar ist. Da wir aber
wissen, dass gelten muss $\pi_M + \pi_F + \pi_V = 1$, können wir $\phi \approx 0,0334$
eindeutig bestimmen und erhalten dadurch

$$\pi \approx (0,1551, \quad 0,0334, \quad 0,8115)$$

\item Wir wissen zwar nicht, in welchem Zustand wir uns zuletzt befanden, gehen
aber davon aus, dass das System bereits ausreichend lange läuft, um akkummuliert
betrachtet die stationäre Verteilung erreicht zu haben. Daraus können wir eine
Wahrscheinlichkeit von $81,15\%$ ablesen.

\item Wir gehen wieder davon aus, dass sich das System anfangs im stationären
Zustand befindet und erhalten eine Startwahrscheinlichkeit von $81,14\%$ dafür,
dass wir im Zustand $V$ starten. Anschließend vollziehen wir die Ereignisfolge
im Zustandsdiagramm nach und bilden das Produkt aus der Startwahrscheinlichkeit
und den jeweiligen Wahrscheinlichkeiten für die Zustandsübergänge. Wir erhalten
für die gegebene Zustandsfolge eine Wahrscheinlichkeit von $0,2197\%$:

$$0,8115 \cdot 0,95 \cdot 0,95 \cdot 0,03 \cdot 0,1 = 2,197 \cdot 10^{-3}$$

\item Wenn wir wissen, dass es an einem Tag Fleisch gab, so befinden wir uns in
dem reinen Zustand $(1, 0, 0)$. Multipliziert mit $P^4$ erhalten wir eine
Wahrscheinlichkeitsverteilung für die verschiedenen Speisen von etwa $42,3722\%$
für Fleisch, $0,4821\%$ für Fisch und $27,5153\%$ für Gemüse.
\end{enumerate}

\end{enumerate}
\end{document}
