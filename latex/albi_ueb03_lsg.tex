\documentclass{homework}
\usepackage{marvosym}
\usepackage{multicol}
%\usepackage{xyling}

\course{Algorithmische Bioinformatik}
\semester{Wintersemester 2012 / 2013}
\no{3}
\date{Montag, dem 5. November 2012}
\author{Stefan Meißner (4279113) und Niels Hoppe (4356370)}
\tutorial{Dienstag 08:00 - 10:00}
\tutor{Alena van Bömmel (Übungsgruppe 3)}

\begin{document}
\maketitle
\begin{enumerate} 

\aufgabe{Distanzmatrizen}{15 + 15}
\begin{enumerate}
\item Das Kriterium einer aditiven Metrik ist erfüllt, wenn es eine Belegung der
Variablen $x, y, u$ und $v$ mit $a, b, c$ und $d$ gibt (4-Punkt Bedingung), sodass gilt:
$$d(x,y) + d(u,v) \leq d(x,u) + d(y,v) = d(x,v) + d(y,u)$$

Das Kriterium einer Ultrametrik ist erfüllt, wenn es eine Belegung der Variablen
$x, y, z$ aus jedem 3-Tupel aus $a, b, c$ und $d$ gibt, sodass gilt:
$$d(x,y) \leq d(x,z) = d(y,z)$$

Außerdem ist jede Ultrametrik additiv, sodass eine nicht additive Metrik auch
keine Ultrametrik sein kann.

\begin{enumerate}
\item Das Kriterium einer additiven Metrik ist erfüllt, da gilt:
\begin{eqnarray*}
d(a,b) + d(c,d) \leq & d(a,c) + d(b,d) & = d(a,d) + d(b,c)\\
9 + 2 \leq & 9 + 5 & = 9 + 5
\end{eqnarray*}
Das Kriterium einer Ultrametrik ist erfüllt, da gilt:
\begin{description}
\item[$(a,b,c)$] \begin{eqnarray*}
d(b,c) \leq & d(b,a) & = d(c,a)\\
5 \leq & 9 & = 9
\end{eqnarray*}
\item[$(a,b,d)$] \begin{eqnarray*}
d(b,d) \leq & d(b,a) & = d(d,a)\\
5 \leq & 9 & = 9
\end{eqnarray*}
\item[$(a,c,d)$] \begin{eqnarray*}
d(c,d) \leq & d(c,a) & = d(d,a)\\
2 \leq & 9 & = 9
\end{eqnarray*}
\item[$(b,c,d)$] \begin{eqnarray*}
d(c,d) \leq & d(c,b) & = d(d,b)\\
2 \leq & 5 & = 5
\end{eqnarray*}
\end{description}

\item Das Kriterium einer additiven Metrik ist erfüllt, da gilt:
\begin{eqnarray*}
d(a,b) + d(c,d) \leq & d(a,c) + d(b,d) & = d(a,d) + d(b,c)\\
5 + 2 \leq & 7 + 9 & = 8 + 8
\end{eqnarray*}
Das Kriterium einer Ultrametrik ist nicht erfüllt, da kein 3-Tupel zwei gleiche
Distanzen enthält und der zweite Teil der Ungleichung damit nicht erfüllbar ist.

\end{enumerate}
\item \textbf{Bonus:}\\
Die Daten eines additiven Baumes brauchen nicht mehr ultrametrisch sein, d.h.
die Annahme der Molekular Uhr kann verletzt werden. Wir nehmen uns also eine
ultrametrische Distanzmatrix und formen diese um, sodass noch folgende
Eigenschaften erhalten bleiben:
\begin{enumerate}
	\item symmetrisch
	\item alle Werte größer 0, bis auf Diagonale
	\item 4-Punkt-Bedingung
\end{enumerate}

Die folgende erste Abbildung zeigt eine einfache ultrametrische Distanzmatrix
für den Baum $(((a,b),(c,d)),(e,f))$. Verkürzen wir den Abstand von $e$ und
verlängern den Abstand von $f$ zu ihren gemeinsamen inneren Knoten
(hypotetischer Vorgänger), so erhalten wir beispielsweise die Distanzmatrix in
der folgenden zweiten Abbildung.

\begin{multicols}{2}
\begin{tabular}{c|cccccc}
 & a & b & c & d & e & f \\\hline
a & 0 & 2 & 4 & 4 & 6 & 6 \\ 
b &  & 0 & 4 & 4 & 6 & 6 \\ 
c &  &  & 0 & 2 & 6 & 6 \\ 
d &  &  &  & 0 & 6 & 6 \\ 
e &  &  &  &  & 0 & 2 \\ 
f &  &  &  &  &  & 0
\end{tabular}

\begin{tabular}{c|cccccc}
 & a & b & c & d & e & f \\\hline
a & 0 & 2 & 4 & 4 & 5.5 & 6.5 \\ 
b &  & 0 & 4 & 4 & 5.5 & 6.5 \\ 
c &  &  & 0 & 2 & 5.5 & 6.5 \\ 
d &  &  &  & 0 & 5.5 & 6.5 \\ 
e &  &  &  &  & 0 & 2 \\ 
f &  &  &  &  &  & 0
\end{tabular}
\end{multicols}

Die ultrametrische Eigenschaft ist hier nicht mehr gegeben, da beispielsweise:
\begin{description}
\item[$(a,e,f)$] \begin{eqnarray*}
d(e,f) \leq & d(a,e) & \neq d(a,f)\\
2 \leq & 5.5 & \neq 6.5\\
d(e,f) \leq & d(a,f) & \neq d(a,e)\\
2 \leq & 6.5 & \neq 5.5
\end{eqnarray*}
\end{description}
Jedoch ist klar, dass die Metrik additiv bleibt, da die Distanz vom $(e,f)$ Paar
zu den anderen Paaren wie bei der ultrametrischen Distanzmatrix immernoch gleich
ist.
\end{enumerate}

\aufgabe{Phylogenetische Bäume}{30}
\begin{enumerate}
\item
\item
\item
\end{enumerate}

\aufgabe{Implementierung Clustering-Verfahren}{60}
\begin{enumerate}
\item
\item
\item
\end{enumerate}

\aufgabe{Maximum-Likelihood-Schätzung}{60}
\begin{enumerate}
\item
\item
\item
\item
\item
\end{enumerate}

\end{enumerate}
\end{document}
