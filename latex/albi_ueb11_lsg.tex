\documentclass{homework}
\usepackage{marvosym}
\usepackage{hyperref}
\usepackage{color}
\usepackage{caption}
\usepackage{float}

\course{Algorithmische Bioinformatik}
\semester{Wintersemester 2012 / 2013}
\no{11}
\date{Montag, dem 14. Januar 2013}
\author{Stefan Meißner (4279113) und Niels Hoppe (4356370)}
\tutorial{Dienstag 08:00 - 10:00}
\tutor{Alena van Bömmel (Übungsgruppe 3)}

\begin{document}
\maketitle
\begin{enumerate} 

\aufgabe{Statistische Verteilungen}{30}
\begin{enumerate}
\item
\item
\item
\end{enumerate}

\aufgabe{Gibbs Sampler}{30}

\aufgabe{Sequencing by Hybridisation}{30}
\begin{enumerate}
\item
\item
\end{enumerate}

\aufgabe{Lowess Normalisierung}{40}
\begin{enumerate}
\item
\item
\item
\end{enumerate}

\aufgabe{RNA-Struktur}{30}
\begin{enumerate}
\item
\item
Die Struktur C kann so nicht vom Zucker-Algorithmus bestimmt werden,
da sie der Bedingung \ldots % TODO
widerspricht. Das zeigt sich deutlich im Kreisdiagramm:

\end{enumerate}

\aufgabe{Clustering}{40}
\begin{enumerate}
\item
\item
\item
\end{enumerate}

\end{enumerate}
\end{document}
